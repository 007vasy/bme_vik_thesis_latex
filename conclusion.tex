%----------------------------------------------------------------------------
\chapter{Conclusions}
%----------------------------------------------------------------------------
\section{CAN fingerprint dataset}
%----------------------------------------------------------------------------
\subsection{Phase 1: Small files}
%----------------------------------------------------------------------------
In conclusion, the predefined approach and the designed process were buoyant. The experts' knowledge were useful, but more like a direction not like a path, every fact has to rechecked and validated.

Target value is needed for the RUL calculation, without that the whole process is like a thought experiment. 
%----------------------------------------------------------------------------
\subsection{Phase 2: Large files with tire measurement}
%----------------------------------------------------------------------------
In this second iteration, the given data was nearly enough to finish piloting and testing, more target value. Unfortunately, more target value and more frequent tire measurements are required for precise prediction, without that the whole process is just a resource heavy estimation.

With more frequent tire measurements these processes could be product ready and help the pre-maintance pursuits.

\section{Electrical failure prediction dataset}
%----------------------------------------------------------------------------

Having seen the evidences, the graph-like data representation approach is resource consuming but worth the future development. With enough optimization this process also could be product ready and help the pre-maintance pursuits of the industry, not just the data provider company, but all situations were there is rather states and phases than values and observations.

%----------------------------------------------------------------------------
\section{Summary on gained experience and knowledge}
%----------------------------------------------------------------------------
The experiences gained from this journey are:
\begin{itemize}
	\item{Do not take anything for granted.} All of the heard facts, solutions, predictions from anyone have to be fact checked and data proven.
	\item{Always dig down to first principles.} When a process is has to be optimized, it is the best solution to dig down to first principles, and from that solid rock bottom build up and develop the whole theory and then the workflow.
	\item{To search for the opportunities of enhancement and advancement.} There is always room for improvement. As it seems from the previous chapter, optimization section the is no perfect solution exists just globally optimal. 
\end{itemize}
%----------------------------------------------------------------------------
\section{Advices to future similar projects}
%----------------------------------------------------------------------------
After this project there are some advices for future pre-maintanance process builders, data-scientist, and graph-like data representers:
\begin{itemize}
	\item{Have an eye on the management.} When there is a channel, there is noise. Moreover, if the desired outcomes of a project or task are not clear, the consequences could not be qualitative. Keep in mind when a favor in both direction is asked or a task is set.
	\item{Build from first principles, and question everything.} With that method the most of future firefightings could be prevented.
\end{itemize}