%----------------------------------------------------------------------------
\chapter{Design}
%----------------------------------------------------------------------------
%TODO Electrix fault pred
%----------------------------------------------------------------------------
\section{Requirement specification}
%----------------------------------------------------------------------------
	\noindent
The final aim was to create data processing methods and calculated attributes from measured values for future prediction applications.
%----------------------------------------------------------------------------
	\subsection{Starting state}
The start state was a raw CAN bus physical sensory data in a *.mat format with different measurement frequencies by sensor types.
%----------------------------------------------------------------------------
	\subsection{Final state}
		\noindent
To achive the final state, data sets require to transform to an CSV (Comma Separate Value) or RDS (R Data Structure) format for more convenient handling and calculating attributes.
		\noindent
The attributes are aggregations and summarisation of connection between key values originating from expert knowledge, exploratory data analysis and several unsupervised learning technique.
%----------------------------------------------------------------------------
\subsection{Abstract Workflow}
To accomplish the final state from the starting state, the process was partitioned from on start-to-end into smaller steps. 

\begin{enumerate}
	\item {Cleaning:} To solve different measurement frequency problem by making a one-row/one-observation data frame.
	\item {Exploring:} Investigating the data with exploratory data analysis tools and unsupervised learning supported by the STILL workers's, engineers's and my acquaintances's expert knowledge.
 	\item {Specifying Attributes:} To introduce sustainable attribute calculating algorithms.
 	\item {Calculating RUL (Remaining usefull life):} From received goal values (few tire measurement) and from the attributes calculated beforehand, present a RUL calculating algoritm for the MANTIS project.
\end{enumerate}

% TODO workflow figure
%----------------------------------------------------------------------------
\section{Data processing workflow design}
%----------------------------------------------------------------------------
	\subsection{Cleaning}
		\subsubsection{Idea One}
		\subsubsection{Idea Two}
		\subsubsection{Idea Synthesis}
	\subsection{Exploring}
		\subsubsection{Exploratory data analysis}
			\paragraph{Data summaries}
			\paragraph{Visualisation}
		\subsubsection{Unsupervised learning}
	\subsection{Specifying Attributes}
	\subsection{Calculating RUL (Remaining usefull life)}
	


