%----------------------------------------------------------------------------
\chapter{Design}
%----------------------------------------------------------------------------
\section{CAN fingerprint data set processing method}
%----------------------------------------------------------------------------
\subsection{Requirement specification}
%----------------------------------------------------------------------------
	\noindent
The final aim was to create data processing methods and calculated attributes from measured values for future prediction applications.
%----------------------------------------------------------------------------
	\subsubsection{Starting state}
The start state was a raw CAN bus physical sensory data in a *.mat format with different measurement frequencies by sensor types.
%----------------------------------------------------------------------------
	\subsubsection{Final state}
		\noindent
To achieve the final state, data sets require to transform to an CSV (Comma Separate Value) or RDS (R Data Structure) format for more convenient handling and calculating attributes.
		\noindent
The attributes are aggregations and summarization of connection between key values originating from expert knowledge, exploratory data analysis and several unsupervised learning technique.
%----------------------------------------------------------------------------
\subsubsection{Abstract Workflow}
To accomplish the final state from the starting state, the process was partitioned from on start-to-end into smaller steps. 

\begin{enumerate}
	\item {Cleaning:} To solve different measurement frequency problem by making a one-row/one-observation data frame.
	\item {Exploring:} Investigating the data with exploratory data analysis tools and unsupervised learning supported by the STILL workers's, engineers's and my acquaintances's expert knowledge.
 	\item {Specifying Attributes:} To introduce sustainable attribute calculating algorithms.
 	\item {Calculating RUL (Remaining useful life):} From received goal values (few tire measurement) and from the attributes calculated beforehand, present a RUL calculating algorithm for the MANTIS project.
\end{enumerate}

\begin{figure}[!ht]
\centering
\includegraphics[width=150mm, keepaspectratio]{figures/abstract_workflow.png}
\caption{Abstract workflow from raw data to final objective} 
\end{figure}

%----------------------------------------------------------------------------
\subsection{Expert knowledge and facts}
%----------------------------------------------------------------------------
\noindent
When some kind of industrial data set is examined and processed to make educated guess about the future principles and insights from expert knowledge are really helpful to make sense and understand the data in a higher sense.\footnote{In reality the vast majority of the insight came from exploratory data analysis and unsupervised learning}

\noindent
In summary, the main indicators of tire abrasion are between two tire measurement:
\begin{itemize}
	\item{traveled distance} 
	\item{elapsed time between}
	\item{elapsed time in high torque low speed state}
	\item{changing x direction in elapsed time}
	\item{changing y direction in elapsed time}
\end{itemize}

These aggregated values can be calculated from the raw sensor data, and from data with enough tire measurement point the RUL can be calculated with a help of a built statistical model.

%----------------------------------------------------------------------------
\subsection{Data processing workflow design}
%----------------------------------------------------------------------------
As it seems from the starting and final state, in first step the *mat files has to be transformed into usable data frame, and meanwhile the not useful columns has to be dropped to save precious processor time and data storage space, especially at handling the longer files. After that the attributes has to be calculated, then the remaining useful life.
To plan that using planning principles is necessary\footnote{Used bibliography \cite{CSDISTILLED}\cite{DATACAMP}\cite{LeanThinking}}.
	\subsubsection{Cleaning}
		\noindent
	As it has been described in the previous section, the *.mat files has to bee transformed to a data frame with only the useful columns remaining to the further calculations.
		\noindent
	To come up with a sustainable process, one has to iterate through more than one ideas, for the sake clarity, all ideas would be enumerated.
		\paragraph{Idea One}
			The first data shape transforming idea was to boxshort the data. In more detail:
\begin{enumerate}
	\item{examine the longest key value pair vector maximal timestamp} 
	\item{make a data frame with timestamp keys from zero to the max timestamp key got from the file in step of hundredth of a second}
	\item{search every value-s place with the help of the timestamp key}
	\item{interpolate for the missing values}
	\item{do it for all columns in the file}
	\item{do it for all files}
\end{enumerate}
			\begin{figure}[!ht]
			\centering
			\includegraphics[width=150mm, keepaspectratio]{figures/cleaning_idea_one.png}
			\caption{First idea to clean the raw data} 
			\end{figure}
		\noindent
This is time consuming and slow even on the server.
		\paragraph{Idea Two}
To get rid of the speed problem, the second idea was to aggregate the values mean by second, this solution have to be more faster than the previous one, but if there time skip in the data\footnote{there is a lot of missing rows, sometimes minutes}
			\begin{figure}[!ht]
			\centering
			\includegraphics[width=150mm, keepaspectratio]{figures/cleaning_idea_two.png}
			\caption{Second idea to read the raw data} 
			\end{figure}
		\paragraph{Idea Synthesis}
To get the advantage of the two methods it would be useful to merge them.
\begin{enumerate}
	\item{examine the longest key value pair vector maximal timestamp} 
	\item{make a data frame with timestamp keys from zero to the max timestamp key got from the file in step of a second}
	\item{aggregate the measurement average by second to second}
	\item{search every value-s place with the help of the timestamp key}
	\item{interpolate for the missing values}
	\item{do it for all columns in the file}
	\item{do it for all files}
\end{enumerate} 
			\begin{figure}[!ht]
			\centering
			\includegraphics[width=150mm, keepaspectratio]{figures/cleaning_idea_synthesis.png}
			\caption{To optimize the cleaning workflow, merged the to processes together} 
			\end{figure}
\noindent
With this solution design, the high quality data frame, what can be used later is possible and with this idea synthesis, the problems with the previous ones are worked around.
%----------------------------------------------------------------------------
	\subsubsection{Exploring}
	The data exploring and data examination, is a required step for correctly interpret the data and confirm the facts coming from expert knowledge. Without this "null hypothesis check" step, lot of human labor and energy can be thrown out.
		\paragraph{Exploratory data analysis}
			With this steps (data summaries and visualization) one can make himself sure, the data set has some connection with the real world, the values are close to reality and further step are worth a shot.
			\subparagraph{Data summaries}
			There are a lot of data summary methods (mean, median, SQRT, min, max and quarters). With these aggregations ran on the data set, it's get a first impression. It's advisable to use these resource sparing methods.
			\subparagraph{Visualization}
			The human mind can easy interpret large amount of data with correct visual representation and to go further checked the connections and facts derived from expert knowledge. \footnote{Not to mention, amend that knowledge.} 
		\paragraph{Unsupervised learning}
		The "unsupervised learning" is a machine learning method especially to unlabeled data, it's useful to find connections and correlation between the values in a higher perspective and confirm or question expert knowledge.

		\subparagraph{k-means clustering}
		\cite{k-means} The k-means clustering help to estimate groups between the observation, with norming all the values to one and calculate the multi-dimensional distance between them.
		\subparagraph{hierarchical clustering}
		\cite{h-clust} Is a k-means with a twist, because it clusters clusters hierarchically, step by step with various distance calculation method and top-down or bottom-up approach.
		\subparagraph{Principal component analysis}
		\cite{PCA}
		This algorithm make an orthogonal, multi dimensional space from the values and place there the observations. This algorithm can tell the most significant values from the data set.
%----------------------------------------------------------------------------
	\subsubsection{Specifying Attributes}
	From expert knowledge \footnote{and exploratory data analysis} the main components and indicators of the tire abrasion can be specified.

	These attributes are:
	\begin{itemize}
		\item{elapsed time} 
		\item{traveled distance} 
		\item{count in various speed and torque state}
		\item{changing x direction}
		\item{changing y direction}
		\item{is there weight counter (max 3600 in a hour)}
		\item{steering wheel degree change derived by time aggregated by average}
	\end{itemize}
	All of them will be aggregated hour by hour\footnote{when the tire measurement interval goes down the aggregation window could and should too}, from tire change or measurement.
%----------------------------------------------------------------------------
	\subsubsection{Calculating RUL (Remaining useful life)}
		\noindent
	After the specified attributes are calculated and aggregated and enough \footnote{more than one} tire measurement is available with accurate time stamp, the RUL can be calculated in the following way:

	\begin{enumerate}
		\item{join the tire measurement and the hourly aggregated attributes by timestamp}
		\item{interpolate from the tire measurements to the hourly aggregations}
		\item{compute the tire diameter change on all the given measurement points}
		\item{calculate the average diameter change by hourly aggregations} with that the tire diameter change by hourly attribute change is given
		\item{produce the attributes hourly for a new measurement}
		\item{calculate the tire diameter change for all attributes}
		\item{compute the mean tire diameter change by hour}
		\item{summarize the tire diameter change from last know diameter}
		\item{if the summary is greater than a given constant, the tire should be changed}
		\item{compute reaming useful life on the last hours abrasion rate}
	\end{enumerate}
	\begin{figure}[!ht]
		\centering
		\includegraphics[width=150mm, keepaspectratio]{figures/CAN_fingerprint_RUL_calculation.png}
		\caption{Calculating RUL from cleaned data} 
	\end{figure}
		\noindent
	Other not implemented solutions:
	\begin{itemize}
		\item{calculate RUL with machine learning} There was not enough data for this solution, the model will be over-fitted based on DataCamp case studies \cite{DataCamp_CaseStudies} 
		\item{calculate RUL with deep learning} Similar then above.	
	\end{itemize}
	
%----------------------------------------------------------------------------
%----------------------------------------------------------------------------
\section{Electrical fail prediction data set processing method}
%----------------------------------------------------------------------------
	\subsection{Requirement specification}
%----------------------------------------------------------------------------
From given CAN bus warning collection form the STILL forklifts electrical parts and SAP technical database, which contains the company's technical workers comments on all repair sessions, and occasionally exchanged parts ID and name, and later described warning sequence from a random truck, the model have to estimate the remaining useful life to a next repair session.

In the scope of this project iteration, the exchange part ID-s is the superior priority, the technical workers comments should be categorized with some natural language processing method. \cite{nlp}
%----------------------------------------------------------------------------
	\subsection{Data processing workflow design}
%----------------------------------------------------------------------------
\cite{BALNWSCBOOK}
\cite{CSDISTILLED}
\cite{DATACAMP}
\cite{LeanThinking}
		\subsubsection{Cleaning}
		get rid of NA-s
		\begin{figure}[!ht]
		\centering
		\includegraphics[width=150mm, keepaspectratio]{figures/Electrical_fail_prediction_preprocessing_workflow.png}
		\caption{To optimize the cleaning workflow, merged the to processes together} 
		\end{figure}
		One have get rid of the resource wasting, fiddling values, and extract the can be useful columns. After that the graph building values can be selected.

		There is two group of the data:
		Nodes
		Edges

%----------------------------------------------------------------------------
		\subsubsection{Exploring}
		With visualization, make sense about the data 
		confirm expert knowledge
%----------------------------------------------------------------------------
		\subsubsection{Calculating RUL}
explain and reason the validity of the whole method
		explain red flag in short
		\begin{figure}[!ht]
		\centering
		\includegraphics[width=150mm, keepaspectratio]{figures/Calc_Fault_pred_RUL.png}
		\caption{To optimize the cleaning workflow, merged the to processes together} 
		\end{figure}
		Nodes: possible warnings and exchange part ID-s
		Edges truck ID, timestamp warning or exchange part ID
		form that we build a graph and with the given sequence we looking for the truck has similar path after that, from the last node of the sequence calculate possibility paths all of the exchange part ID-s. We display it with avg, min max mean time frame and calculated path possibility.