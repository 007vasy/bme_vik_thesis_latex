%----------------------------------------------------------------------------
\chapter{Design}
%----------------------------------------------------------------------------
\section{CAN fingerprint data set processing method}
%----------------------------------------------------------------------------
\subsection{Requirement specification}
%----------------------------------------------------------------------------
	\noindent
The final aim was to create data processing methods and calculated attributes from measured values for future prediction applications.
%----------------------------------------------------------------------------
	\subsubsection{Starting state}
The start state was a raw CAN bus physical sensory data in a *.mat format with different measurement frequencies by sensor types.
%----------------------------------------------------------------------------
	\subsubsection{Final state}
		\noindent
To achive the final state, data sets require to transform to an CSV (Comma Separate Value) or RDS (R Data Structure) format for more convenient handling and calculating attributes.
		\noindent
The attributes are aggregations and summarisation of connection between key values originating from expert knowledge, exploratory data analysis and several unsupervised learning technique.
%----------------------------------------------------------------------------
\subsubsection{Abstract Workflow}
To accomplish the final state from the starting state, the process was partitioned from on start-to-end into smaller steps. 

\begin{enumerate}
	\item {Cleaning:} To solve different measurement frequency problem by making a one-row/one-observation data frame.
	\item {Exploring:} Investigating the data with exploratory data analysis tools and unsupervised learning supported by the STILL workers's, engineers's and my acquaintances's expert knowledge.
 	\item {Specifying Attributes:} To introduce sustainable attribute calculating algorithms.
 	\item {Calculating RUL (Remaining usefull life):} From received goal values (few tire measurement) and from the attributes calculated beforehand, present a RUL calculating algoritm for the MANTIS project.
\end{enumerate}

\begin{figure}[!ht]
\centering
\includegraphics[width=150mm, keepaspectratio]{figures/abstract_workflow.png}
\caption{Abstract workflow from raw data to final objective} 
\end{figure}

%----------------------------------------------------------------------------
\subsection{Expert knowledge and facts}
%----------------------------------------------------------------------------
(Sidenote: reality)
%----------------------------------------------------------------------------
\subsection{Data processing workflow design}
%----------------------------------------------------------------------------
	\subsubsection{Cleaning}
		\paragraph{Idea One}
			\begin{figure}[!ht]
			\centering
			\includegraphics[width=150mm, keepaspectratio]{figures/cleaning_idea_one.png}
			\caption{First idea to clean the raw data} 
			\end{figure}
		\paragraph{Idea Two}
			\begin{figure}[!ht]
			\centering
			\includegraphics[width=150mm, keepaspectratio]{figures/cleaning_idea_two.png}
			\caption{Second idea to read the raw data} 
			\end{figure}
		\paragraph{Idea Synthesis}
			\begin{figure}[!ht]
			\centering
			\includegraphics[width=150mm, keepaspectratio]{figures/cleaning_idea_synthesis.png}
			\caption{To optimise the cleaning workflow, merged the to processes together} 
			\end{figure}
	\subsubsection{Exploring}
	explain why is it usefull to do in a broder sense
		\paragraph{Exploratory data analysis}
			explain why is it usefull to do
			\subparagraph{Data summaries}
			enumerate data summary methods
			\subparagraph{Visualisation}
			enumerat data visualisation methods
		\paragraph{Unsupervised learning}
		explain what is it 
		enumarate some types
	\subsubsection{Specifying Attributes}
	from exprt knowledge
	\subsubsection{Calculating RUL (Remaining usefull life)}
	calc what from what and why
		\begin{figure}[!ht]
		\centering
		\includegraphics[width=150mm, keepaspectratio]{figures/CAN_fingerprint_RUL_calculation.png}
		\caption{To optimise the cleaning workflow, merged the to processes together} 
		\end{figure}	
%----------------------------------------------------------------------------
\section{Electrical fail prediction data set processing method}
%----------------------------------------------------------------------------
	\subsection{Requirement specification}
%----------------------------------------------------------------------------
	\subsection{Data processing workflow design}
%----------------------------------------------------------------------------
		\subsubsection{Cleaning}
		\begin{figure}[!ht]
		\centering
		\includegraphics[width=150mm, keepaspectratio]{figures/Electrical_fail_prediction_preprocessing_workflow.png}
		\caption{To optimise the cleaning workflow, merged the to processes together} 
		\end{figure}
%----------------------------------------------------------------------------
		\subsubsection{Exploring}
%----------------------------------------------------------------------------
		\subsubsection{Calculating RUL}
		\begin{figure}[!ht]
		\centering
		\includegraphics[width=150mm, keepaspectratio]{figures/Calc_Fault_pred_RUL.png}
		\caption{To optimise the cleaning workflow, merged the to processes together} 
		\end{figure}