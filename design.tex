%----------------------------------------------------------------------------
\chapter{Design}
%----------------------------------------------------------------------------
%TODO Electrix fault pred
%----------------------------------------------------------------------------
\section{Requirement specification}
%----------------------------------------------------------------------------
	\noindent
The end goal is to come up with data processing methods and calculated attributes from the measured values for the future prediction applications.
	\subsection{Start state}
The start state is a raw CAN bus physical sensory data in a *.mat format, different measurement frequencies by sensor types.

	\subsection{End state}
		\noindent
To reach the end state one has to transform that to an CSV (Comma Separate Value) or RDS (R Data Structure) format, for more convienent handling and calculate attributes.
		\noindent
The attributes are aggregations and summarisation of the connection between key values, it's comes from expert knowledge, exploratory data analysis and some unsupervised learning technique.	 \subsection{Abstract Workflow}
To reach the end state from the start state, we have to break up the process from one (start-to-end) to smaller steps.

\begin{enumerate}
	\item {Cleaning:} Solve the different measurement frequency problem, by making a one-row/one-observation data frame.
	\item {Exploring:} Explore the data with exploratory data analysis tool and unsupervised learning, backed up with the STILL workers's, engineers's and my acquaintances's expert knowledge.
 	\item {Specifying Attributes:} Come up with sustainable attribute calculating algorithms.
 	\item {Implemention:} Implementing the cleaning process and attribute calculation.
\end{enumerate}

% TODO workflow figure
%----------------------------------------------------------------------------
\section{Data processing workflow design}
%----------------------------------------------------------------------------
	\subsection{Cleaning}
		\subsubsection{Idea One}
		\subsubsection{Idea Two}
		\subsubsection{Idea Synthesis}
	\subsection{Exploring}
		\subsubsection{Exploratory data analysis}
			\paragraph{Data summaries}
			\paragraph{Visualisation}
		\subsubsection{Unsupervised learning}
	\subsection{Specifying Attributes}
	\subsection{Implemention}
		\subsubsection{Cleaning process}
		\subsubsection{Attribute calculating algorithms}		


