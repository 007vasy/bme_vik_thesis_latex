\documentclass[11pt,a4paper,oneside]{report}             % Single-side
%\documentclass[11pt,a4paper,twoside,openright]{report}  % Duplex

%\PassOptionsToPackage{chapternumber=Huordinal}{magyar.ldf}
\usepackage{t1enc}
\usepackage[latin2]{inputenc}
\usepackage{amsmath}
\usepackage{amssymb}
\usepackage{enumerate}
\usepackage[thmmarks]{ntheorem}
\usepackage{graphics}
\usepackage{epsfig}
\usepackage{listings}
\usepackage{color}
%\usepackage{fancyhdr}
\usepackage{lastpage}
\usepackage{anysize}
\usepackage[magyar]{babel}
\usepackage{sectsty}
\usepackage{setspace}  % Ettol a tablazatok, abrak, labjegyzetek maradnak 1-es sorkozzel!
\usepackage[hang]{caption}
\usepackage{hyperref}

%--------------------------------------------------------------------------------------
% Main variables
%--------------------------------------------------------------------------------------
\newcommand{\vikszerzo}{B�dis-Szomor� Andr�s}
\newcommand{\vikkonzulens}{dr.~Konzulens Elem�r}
\newcommand{\vikcim}{Elektronikus terel�k}
\newcommand{\viktanszek}{M�r�stechnika �s Inform�ci�s Rendszerek Tansz�k}
\newcommand{\vikdoktipus}{Diplomaterv}
\newcommand{\vikdepartmentr}{B�dis-Szomor� Andr�s}

%--------------------------------------------------------------------------------------
% Page layout setup
%--------------------------------------------------------------------------------------
% we need to redefine the pagestyle plain
% another possibility is to use the body of this command without \fancypagestyle
% and use \pagestyle{fancy} but in that case the special pages
% (like the ToC, the References, and the Chapter pages)remain in plane style

\pagestyle{plain}
%\setlength{\parindent}{0pt} % �ttekinthet�bb, angol nyelv� dokumentumokban jellemz�
%\setlength{\parskip}{8pt plus 3pt minus 3pt} % �ttekinthet�bb, angol nyelv� dokumentumokban jellemz�
\setlength{\parindent}{12pt} % magyar nyelv� dokumentumokban jellemz�
\setlength{\parskip}{0pt}    % magyar nyelv� dokumentumokban jellemz�

\marginsize{35mm}{25mm}{15mm}{15mm} % anysize package
\setcounter{secnumdepth}{0}
\sectionfont{\large\upshape\bfseries}
\setcounter{secnumdepth}{2}
\singlespacing
\frenchspacing

%--------------------------------------------------------------------------------------
%	Setup hyperref package
%--------------------------------------------------------------------------------------
\hypersetup{
    bookmarks=true,            % show bookmarks bar?
    unicode=false,             % non-Latin characters in Acrobat�s bookmarks
    pdftitle={\vikcim},        % title
    pdfauthor={\vikszerzo},    % author
    pdfsubject={\vikdoktipus}, % subject of the document
    pdfcreator={\vikszerzo},   % creator of the document
    pdfproducer={Producer},    % producer of the document
    pdfkeywords={keywords},    % list of keywords
    pdfnewwindow=true,         % links in new window
    colorlinks=true,           % false: boxed links; true: colored links
    linkcolor=black,           % color of internal links
    citecolor=black,           % color of links to bibliography
    filecolor=black,           % color of file links
    urlcolor=black             % color of external links
}

%--------------------------------------------------------------------------------------
% Set up listings
%--------------------------------------------------------------------------------------
\lstset{
	basicstyle=\scriptsize\ttfamily, % print whole listing small
	keywordstyle=\color{black}\bfseries\underbar, % underlined bold black keywords
	identifierstyle=, 					% nothing happens
	commentstyle=\color{white}, % white comments
	stringstyle=\scriptsize\sffamily, 			% typewriter type for strings
	showstringspaces=false,     % no special string spaces
	aboveskip=3pt,
	belowskip=3pt,
	columns=fixed,
	backgroundcolor=\color{lightgray},
} 		
\def\lstlistingname{lista}	

%--------------------------------------------------------------------------------------
%	Some new commands and declarations
%--------------------------------------------------------------------------------------
\newcommand{\code}[1]{{\upshape\ttfamily\scriptsize\indent #1}}

% define references
\newcommand{\figref}[1]{\ref{fig:#1}.}
\renewcommand{\eqref}[1]{(\ref{eq:#1})}
\newcommand{\listref}[1]{\ref{listing:#1}.}
\newcommand{\sectref}[1]{\ref{sect:#1}}
\newcommand{\tabref}[1]{\ref{tab:#1}.}

\DeclareMathOperator*{\argmax}{arg\,max}
%\DeclareMathOperator*[1]{\floor}{arg\,max}
\DeclareMathOperator{\sign}{sgn}
\DeclareMathOperator{\rot}{rot}
\definecolor{lightgray}{rgb}{0.95,0.95,0.95}

\author{\vikszerzo}
\title{\viktitle}
\includeonly{
	guideline,%
	project,%
	titlepage,%
	declaration,%
	abstract,%
	introduction,%
	chapter1,%
	chapter2,%
	chapter3,%
	acknowledgement,%
	appendices,%
}
%--------------------------------------------------------------------------------------
%	Setup captions
%--------------------------------------------------------------------------------------
\captionsetup[figure]{
%labelsep=none,
%font={footnotesize,it},
%justification=justified,
width=.75\textwidth,
aboveskip=10pt}

\renewcommand{\captionlabelfont}{\small\bf}
\renewcommand{\captionfont}{\footnotesize\it}

%--------------------------------------------------------------------------------------
% Table of contents and the main text
%--------------------------------------------------------------------------------------
\begin{document}
\singlespacing
%--------------------------------------------------------------------------------------
% Rovid formai es tartalmi tajekoztato
%--------------------------------------------------------------------------------------

\footnotesize
\begin{center}
\large
\textbf{\Large �ltal�nos inform�ci�k, a diplomaterv szerkezete}\\
\end{center}

A diplomaterv szerkezete a BME Villamosm�rn�ki �s Informatikai Kar�n:
\begin{enumerate}
\item	Diplomaterv feladatki�r�s
\item	C�moldal
\item	Tartalomjegyz�k
\item	A diplomatervez� nyilatkozata az �n�ll� munk�r�l �s az elektronikus adatok kezel�s�r�l
\item	Tartalmi �sszefoglal� magyarul �s angolul
\item	Bevezet�s: a feladat �rtelmez�se, a tervez�s c�lja, a feladat indokolts�ga, a diplomaterv fel�p�t�s�nek r�vid �sszefoglal�sa
\item	A feladatki�r�s pontos�t�sa �s r�szletes elemz�se
\item	El�zm�nyek (irodalomkutat�s, hasonl� alkot�sok), az ezekb�l levonhat� k�vetkeztet�sek
\item	A tervez�s r�szletes le�r�sa, a d�nt�si lehet�s�gek �rt�kel�se �s a v�lasztott megold�sok indokl�sa
\item	A megtervezett m�szaki alkot�s �rt�kel�se, kritikai elemz�se, tov�bbfejleszt�si lehet�s�gek
\item	Esetleges k�sz�netnyilv�n�t�sok
\item	R�szletes �s pontos irodalomjegyz�k
\item	F�ggel�k(ek)
\end{enumerate}

Felhaszn�lhat� a k�vetkez� oldalt�l kezd�d� \LaTeX-Diplomaterv sablon dokumentum tartalma. 

A diplomaterv szabv�nyos m�ret� A4-es lapokra ker�lj�n. Az oldalak t�k�rmarg�val k�sz�ljenek (mindenhol 2.5cm, baloldalon 1cm-es k�t�ssel). Az alap�rtelmezett bet�k�szlet a 12 pontos Times New Roman, m�sfeles sork�zzel.

Minden oldalon - az els� n�gy szerkezeti elem kiv�tel�vel - szerepelnie kell az oldalsz�mnak.

A fejezeteket decim�lis beoszt�ssal kell ell�tni. Az �br�kat a megfelel� helyre be kell illeszteni, fejezetenk�nt decim�lis sz�mmal �s kifejez� c�mmel kell ell�tni. A fejezeteket decim�lis al�oszt�ssal sz�mozzuk, maxim�lisan 3 al�oszt�s m�lys�gben (pl. 2.3.4.1.). Az �br�kat, t�bl�zatokat �s k�pleteket c�lszer� fejezetenk�nt k�l�n sz�mozni (pl. 2.4. �bra, 4.2 t�bl�zat vagy k�pletn�l (3.2)). A fejezetc�meket igaz�tsuk balra, a norm�l sz�vegn�l viszont haszn�ljunk sorkiegyenl�t�st. Az �br�kat, t�bl�zatokat �s a hozz�juk tartoz� c�met igaz�tsuk k�z�pre. A c�m a jel�lt r�sz alatt helyezkedjen el.

A k�peket lehet�leg rajzol� programmal k�sz�ts�k el, az egyenleteket egyenlet-szerkeszt� seg�ts�g�vel �rj�k le (A \LaTeX~ehhez k�zenfekv� megold�sokat ny�jt).

Az irodalomjegyz�k sz�vegk�zi hivatkoz�sa t�rt�nhet a Harvard-rendszerben (a szerz� �s az �vsz�m megad�s�val) vagy sorsz�mozva. A teljes lista n�vsor szerinti sorrendben a sz�veg v�g�n szerepeljen (sorsz�mozott irodalmi hivatkoz�sok eset�n hivatkoz�si sorrendben). A szakirodalmi forr�sok c�meit azonban mindig az eredeti nyelven kell megadni, esetleg z�r�jelben a ford�t�ssal. A list�ban szerepl� valamennyi publik�ci�ra hivatkozni kell a sz�vegben (a \LaTeX-sablon a Bib\TeX~seg�ts�g�vel mindezt automatikusan kezeli). Minden publik�ci� a szerz�k ut�n a k�vetkez� adatok szerepelnek: foly�irat cikkekn�l a pontos c�m, a foly�irat c�me, �vfolyam, sz�m, oldalsz�m t�l-ig. A foly�irat c�meket csak akkor r�vid�ts�k, ha azok nagyon k�zismertek vagy nagyon hossz�ak. Internet hivatkoz�sok megad�sakor fontos, hogy az el�r�si �t el�tt megadjuk az oldal tulajdonos�t �s tartalm�t (mivel a link egy id� ut�n ak�r el�rhetetlenn� is v�lhat), valamint az el�r�s id�pontj�t.

\vspace{5mm}
Fontos:
\begin{itemize}
	\item A szakdolgozat k�sz�t� / diplomatervez� nyilatkozata (a jelen sablonban szerepl� sz�vegtartalommal) k�telez� el��r�s Karunkon ennek hi�ny�ban a szakdolgozat/diplomaterv nem b�r�lhat� �s nem v�dhet� !
	\item Mind a dolgozat, mind a mell�klet maxim�lisan 15 MB m�ret� lehet !
\end{itemize}

\vspace{5mm}
\begin{center}
J� munk�t, sikeres szakdolgozat k�sz�t�st ill. diplomatervez�st k�v�nunk !
\end{center}

\normalsize

%--------------------------------------------------------------------------------------
% Feladatkiiras (a tanszeken atveheto, kinyomtatott valtozat)
%--------------------------------------------------------------------------------------
\clearpage
\begin{center}
\large
\textbf{FELADATKI�R�S}\\
\end{center}

A feladatki�r�st a tansz�ki adminisztr�ci�ban lehet �tvenni, �s a leadott munk�ba eredeti, tansz�ki pecs�ttel ell�tott �s a tansz�kvezet� �ltal al��rt lapot kell belef�zni (ezen oldal \emph{helyett}, ez az oldal csak �tmutat�s). Az elektronikusan felt�lt�tt dolgozatban m�r nem kell beleszerkeszteni ezt a feladatki�r�st.





\pagenumbering{arabic}
\onehalfspacing
%--------------------------------------------------------------------------------------
%	The title page
%--------------------------------------------------------------------------------------
\begin{titlepage}
\begin{center}
\includegraphics[width=60mm,keepaspectratio]{figures/BMElogo.png}\\
\vspace{0.3cm}
\textbf{BUDAPEST UNIVERSITY OF TECHNOLOGY AND ECONOMICS}\\
\textmd{Faculty of Electrical Engineering and Informatics}\\
\textmd{\viktanszek}\\[5cm]

\vspace{0.4cm}
{\huge \bfseries \vikcim}\\[0.8cm]
\vspace{0.5cm}
\textsc{\Large \vikdoktipus}\\[4cm]

\begin{tabular}{cc}
 \makebox[7cm]{\emph{Created by}} & \makebox[7cm]{\emph{Supervisor}} \\
 \makebox[7cm]{\vikszerzo} & \makebox[7cm]{\vikkonzulens}
\end{tabular}

\vfill
{\large \today}
\end{center}
\end{titlepage}



\tableofcontents\vfill
%--------------------------------------------------------------------------------------
% Nyilatkozat
%--------------------------------------------------------------------------------------
\begin{center}
\large
\textbf{HALLGAT�I NYILATKOZAT}\\
\end{center}

Alul�rott \emph{\vikszerzo}, szigorl� hallgat� kijelentem, hogy ezt a szakdolgozatot/ diplomatervet \textcolor{blue}{(nem k�v�nt t�rlend�)} meg nem engedett seg�ts�g n�lk�l, saj�t magam k�sz�tettem, csak a megadott forr�sokat (szakirodalom, eszk�z�k stb.) haszn�ltam fel. Minden olyan r�szt, melyet sz� szerint, vagy azonos �rtelemben, de �tfogalmazva m�s forr�sb�l �tvettem, egy�rtelm�en, a forr�s megad�s�val megjel�ltem.

Hozz�j�rulok, hogy a jelen munk�m alapadatait (szerz�(k), c�m, angol �s magyar nyelv� tartalmi kivonat, k�sz�t�s �ve, konzulens(ek) neve) a BME VIK nyilv�nosan hozz�f�rhet� elektronikus form�ban, a munka teljes sz�veg�t pedig az egyetem bels� h�l�zat�n kereszt�l (vagy autentik�lt felhaszn�l�k sz�m�ra) k�zz�tegye. Kijelentem, hogy a beny�jtott munka �s annak elektronikus verzi�ja megegyezik. D�k�ni enged�llyel titkos�tott diplomatervek eset�n a dolgozat sz�vege csak 3 �v eltelte ut�n v�lik hozz�f�rhet�v�.

\begin{flushleft}
\vspace*{1cm}
Budapest, \today
\end{flushleft}

\begin{flushright}
 \vspace*{1cm}
 \makebox[7cm]{\rule{6cm}{.4pt}}\\
 \makebox[7cm]{\emph{\vikszerzo}}\\
 \makebox[7cm]{hallgat�}
\end{flushright}
\thispagestyle{empty}

\vfill
\clearpage
\thispagestyle{empty} % an empty page


%----------------------------------------------------------------------------
% Abstract in english
%----------------------------------------------------------------------------
\chapter*{Abstract}\addcontentsline{toc}{chapter}{Abstract}

The purpose of my thesis is studying data science in detail and beside studying its benefits, to develop data-based proactive maintenance methods for industrial devices, tools and machines.

In case of this thesis, two dataset has to be preprocessed, analyzed and with that information, two proactive maintenance method has to be designed, implemented and verified.

For educational and research purposes I used two technologies (R,Python) to reach these ambitious objectives and implemented various data processing and analyzing methods, borrowed from the statistical or data science field. Armed with that knowledge I come up with special processes to help to indicate various type of failure in the examined machines and applications.

%----------------------------------------------------------------------------
\chapter*{Introduction}\addcontentsline{toc}{chapter}{Introduction}
%----------------------------------------------------------------------------
\section{Antendent}
%----------------------------------------------------------------------------
In this current era the interconnectivity and interoperability is transform our environtment, and has ernormous effect on the all industries too. We use more and more sensors, and connect them to collect usefull data on purpose to predict the future.

In this thesis i would like to develop some future predicting method for proactive maintenance, to cut down human and financial cost.

The data what i have get to work with is a CAN-Bus data collected from the STILL Company's forklifts.

My main task is to prototype method and reason for specific data types for better results to improve the Industry 4.0 MANTIS project.

To fulfill this purpose, i have worked with two kinds of datasets.
%----------------------------------------------------------------------------
\section{My Solution}
%----------------------------------------------------------------------------
My task was with the "CAN fingerprint" data set to make it to a tidy format, explore what can one predict with no goal value providet, figure out requirement for future better prediction, visualise the connection between the given variables, and with the second phase data wave and small amount and low resolution goal value made a pilot predict algorithm to iterate the solution searching process. 

%(Electrical fault pred)

\section{Chapter description}
%----------------------------------------------------------------------------
		\paragraph{Technical background}
%----------------------------------------------------------------------------

Brief description: Technical background

%----------------------------------------------------------------------------
		\paragraph{Design}
%----------------------------------------------------------------------------

Brief description: Design

%----------------------------------------------------------------------------
		\paragraph{Implementation}
%----------------------------------------------------------------------------

Brief description: Implementation

%----------------------------------------------------------------------------
		\paragraph{Verification}
%----------------------------------------------------------------------------

Brief description: Verification

%----------------------------------------------------------------------------
		\paragraph{Conclusion}
%----------------------------------------------------------------------------

Brief description: Conclusion


%----------------------------------------------------------------------------
\chapter{\LaTeX-eszk�z�k}\label{sect:LatexTools}
%----------------------------------------------------------------------------
\section{A szerkeszt�shez haszn�latos, Windows alap� eszk�z�k}
%----------------------------------------------------------------------------
Ez a sablon Windows oper�ci�s rendszer alatt k�sz�lt TeXnicCenter 1 Beta 7.01 szerkeszt�vel. A TeXnicCenter egy \LaTeX-szerkeszt�program sz�mtalan hasznos -- �s r�ad�sul j�l m�k�d� -- szolg�ltat�ssal (\figref{TexnicCenter} �bra). A szoftver ingyenesen let�lthet� a\\\url{http://www.texniccenter.org/} c�mr�l.

\begin{figure}[!ht]
\centering
\includegraphics[width=150mm, keepaspectratio]{figures/TeXnicCenter.png}
\caption{A TeXnicCenter Windows alap� \LaTeX-szerkeszt�.} 
\label{fig:TexnicCenter}
\end{figure}

Egy m�sik haszn�lhat� Windows alap� szerkeszt�program a LEd (LaTeX Editor,\\\url{http://www.latexeditor.org/}), a TeXnicCenter azonban stabilabb, gyorsabb, �s jobban haszn�lhat�.

%----------------------------------------------------------------------------
\section{A dokumentum leford�t�sa Windows alatt}
%----------------------------------------------------------------------------
A TeXnicCenter �s a LEd kiz�r�lag szerkeszt�program (b�r az ut�bbiban DVI-n�zeget� is van), �gy a dokumentum ford�t�s�hoz sz�ks�ges eszk�z�ket nem tartalmazza. Windows alatt alapvet�en k�t lehet�s�g k�z�l �rdemes v�lasztani: MiKTeX (\url{http://miktex.org/}) �s TeXLive (\url{http://www.tug.org/texlive/}) programcsomag. Az ut�bbi m�k�dik Mac OS X, GNU/Linux alatt �s Unix-sz�rmaz�kokon is. A MiKTeX egy alapcsomag telep�t�se ut�n mindig let�lti a haszn�lt funkci�khoz sz�ks�ges, de lok�lisan hi�nyz� \TeX-csomagokat, m�g a TeXLive DVD ISO verz�ban f�rhet� hozz�. Ez a dokumentum TeXLive 2008 programcsomag seg�ts�g�vel fordult, amelynek DVD ISO verzi�ja a megadott oldalr�l let�lthet�. A sablon leford�t�s�hoz a disztrib�ci�ban szerepl� \verb+magyar.ldf+ f�jlt a \verb+http://www.math.bme.hu/latex/+ v�ltozatra kell cser�lni, vagy az ut�bbi v�ltozatot be kell m�solni a projekt-k�nyvt�rba (ahogy ezt meg is tett�k a sablonban) k�l�nben anom�li�k tapasztalhat�k a dokumentumban (pl. az �bra- �s t�bl�zat-al��r�sok form�tuma nem a be�ll�tott lesz, vagy bizonyos oldalakon megjelenik alap�telmez�sben egy fejl�c). A TeXLive 2008-at m�g nem kell k�l�n telep�teni a g�pre, elegend� DVD-r�l (vagy az ISO f�jlb�l k�zvetlen�l, pl. DaemonTools-szal) haszn�lni. 

A \TeX-eszk�z�ket tartalmaz� programcsomag bin�risainak el�r�si �tj�t minden esetben be kell �ll�tani a szerkeszt�programban, p�ld�ul TeXnicCenter eset�n legegyszer�bben a \verb+Build / Define output profiles...+ men�ponttal el�h�vott dial�gusablakban a \verb+Wizard...+ gombra kattintva tehetj�k ezt meg.

A PDF-\LaTeX~haszn�lata eset�n a gener�lt dokumentum k�zvetlen�l PDF-form�tumban �ll rendelkez�sre. Amennyiben a PDF-f�jl egy PDF-n�z�ben (pl. Adobe Acrobat Reader vagy Foxit PDF Reader) meg van nyitva, akkor a f�jlle�r�t a PDF-n�z� program tipikusan lefoglalja. Ilyen esetben a dokumentum �jraford�t�sa hiba�zenettel kil�p. Ha bez�rjuk �s �jra megnyitjuk a PDF dokumentumot, akkor pedig a PDF-n�z�k t�bbs�ge az els� oldalon nyitja meg a dokumentumot, nem a legut�bb olvasott oldalon. Ezzel szemben p�ld�ul az egyszer� �s ingyenes \textcolor{blue}{Sumatra PDF} nev� program k�pes arra, hogy a megnyitott dokumentum megv�ltoz�s�t detekt�lja, �s friss�tse a n�zetet az aktu�lis oldal megtart�s�val.

%----------------------------------------------------------------------------
\section{Eszk�z�k Linuxhoz}
%----------------------------------------------------------------------------
Linux oper�ci�s rendszer alatt is rengeteg szerkeszt�program van, pl. a KDE alap� Kile j�l haszn�lhat�. Ez ingyenesen let�lthet�, vagy �ppens�ggel az adott Linux-disztrib�ci� eleve tartalmazza, ahogyan a dokumentum ford�t�s�hoz sz�ks�ges csomagokat is. Az Ubuntu Linux disztrib�ci�k alatt p�ld�ul legt�bbsz�r a \verb+texlive-base+ csomag telep�t�s�vel haszn�lhat�k a \LaTeX-eszk�z�k.

%----------------------------------------------------------------------------
\chapter{A dolgozat formai kivitele}
%----------------------------------------------------------------------------
Az itt tal�lhat� inform�ci�k egy r�sze a BME VIK Hallgat�i K�pviselet �ltal k�sz�tett ,,Utols� f�l�v a villanykaron'' c. munk�b�l lett kis v�ltoztat�sokkal �temelve. Az eredeti dokumentum az al�bbi linken �rhet� el: \url{http://vik-hk.bme.hu/diplomafelev-howto-2009}.

%----------------------------------------------------------------------------
\section{A dolgozat kim�rete}
%----------------------------------------------------------------------------
A minim�lis 50, az optim�lis kim�ret 60-70 oldal (f�ggel�kkel egy�tt). A b�r�l�k �s a z�r�vizsga bizotts�g sem szereti kifejezetten a t�l hossz� dolgozatokat, �gy a brutt� 90 oldalt m�r nem �rdemes t�lsz�rnyalni. Egy�bk�nt f�ggetlen�l a dolgozat kim�ret�t�l, ha a dolgozat nem �rdekfesz�t�, akkor az olvas� m�r az elej�n a v�g�t fogja v�rni. �rdemes z�rt, �nmag�ban is �rthet� m�vet alkotni.

ex
Some of the \textbf{greatest} 
discoveries in \underline{science} 
were made by \textbf{\textit{accident}}.

%----------------------------------------------------------------------------
\section{A dolgozat nyelve}
%----------------------------------------------------------------------------
Mivel Magyarorsz�gon a hivatalos nyelv a magyar, ez�rt alap�rtelmez�sben magyarul kell meg�rni a dolgozatot. Aki k�lf�ldi posztgradu�lis k�pz�sben akar r�szt venni, nemzetk�zi szint� tudom�nyos kutat�st szeretne v�gezni, vagy multinacion�lis c�gn�l akar elhelyezkedni, annak c�lszer� angolul meg�rnia diplomadolgozat�t. Miel�tt a hallgat� az angol nyelv� verzi� mellett d�nt, er�sen aj�nlott m�rlegelni, hogy ez mennyi t�bbletmunk�t fog a hallgat�nak jelenteni fogalmaz�s �s nyelvhelyess�g ter�n, valamint - nem utols� sorban - hogy ez mennyi t�bbletmunk�t fog jelenteni a konzulens illetve b�r�l� sz�m�ra. Egy nehezen olvashat�, netal�n �rthetetlen sz�veg teher minden j�t�kos sz�m�ra.

%----------------------------------------------------------------------------
\section{A dokumentum nyomdatechnikai kivitele}
%----------------------------------------------------------------------------
A dolgozatot A4-es feh�r lapra nyomtatva, 2,5 centim�teres marg�val (+1~cm k�t�sbeni), 11-12 pontos bet�m�rettel, talpas bet�t�pussal �s m�sfeles sork�zzel c�lszer� elk�sz�teni.



%----------------------------------------------------------------------------
\chapter{A \LaTeX-sablon haszn�lata}
%----------------------------------------------------------------------------
Ebben a fejezetben r�viden, implicit m�don bemutatjuk a sablon haszn�lat�nak m�dj�t, ami azt jelenti, hogy sablon haszn�lata ennek a dokumentumnak a forr�sk�dj�t tanulm�nyozva v�lik teljesen vil�goss�. Amennyiben a szoftver-keretrendszer telep�tve van, a sablon alkalmaz�sa �s a dolgozat szerkeszt�se \LaTeX-ben a sablon seg�ts�g�vel tapasztalataink szerint j�val hat�konyabb, mint egy WYSWYG (\emph{What You See is What You Get}) t�pus� sz�vegszerkeszt� eset�n (pl. Microsoft Word, OpenOffice).

%----------------------------------------------------------------------------
\section{C�mk�k �s hivatkoz�sok}
%----------------------------------------------------------------------------
A \LaTeX~dokumentumban c�mk�ket (\verb+\label+) rendelhet�nk �br�khoz, t�bl�zatokhoz, fejezetekhez, list�khoz, k�pletekhez stb. Ezekre a dokumentum b�rmely r�sz�ben hivatkozhatunk, a hivatkoz�sok automatikusan felold�sra ker�lnek.

A sablonban makr�kat defini�ltunk a hivatkoz�sok megk�nny�t�s�hez. Ennek megfelel�en minden �bra (\emph{figure}) c�mk�je \verb+fig:+ kulcssz�val kezd�dik, m�g minden t�bl�zat (\emph{table}), k�plet (\emph{equation}), fejezet (\emph{section}) �s lista (\emph{listing}) rendre a \verb+tab:+, \verb+eq:+, \verb+sect:+ �s \verb+listing:+ kulcssz�val kezd�dik, �s a kulcsszavak ut�n tetsz�legesen v�lasztott c�mke haszn�lhat�. Ha ezt a konvenci�t betartjuk, akkor az el�bbi objektumok sz�m�ra rendre a \verb+\figref+, \verb+\tabref+, \verb+\eqref+, \verb+\sectref+ �s \verb+\listref+ makr�kkal hivatkozhatunk. A makr�k param�tere a c�mke, amelyre hivatkozunk (a kulcssz� n�lk�l). Az �sszes eml�tett hivatkoz�st�pus, bele�rtve az \verb+\url+ kulcssz�val bevezetett web-hivatkoz�sokat is a  \verb+hyperref+\footnote{Seg�ts�g�vel a dokumentumban megjelen� hivatkoz�sok nem csak dinamikuss� v�lnak, de sz�nezhet�k is, b�vebbet err�l a csomag dokument�ci�j�ban tal�lunk. Ez egy�ttal egy p�lda l�bjegyzet �r�s�ra.} csomagnak k�sz�nhet�en akt�vak a legt�bb PDF-n�zeget�ben, r�juk kattintva a dokumentum megfelel� oldal�ra ugrik a PDF-n�z� vagy a megfelel� linket megnyitja az alap�rtelmezett b�ng�sz�vel. A \verb+hyperref+ csomag a kimeneti PDF-dokumentumba k�nyvjelz�ket is k�sz�t a tartalomjegyz�kb�l. Ez egy szint�n akt�v tartalomjegyz�k, amelynek elemeire kattintva a n�zeget� behozza a kiv�lasztott fejezetet.

%----------------------------------------------------------------------------
\section{�br�k �s t�bl�zatok}
%----------------------------------------------------------------------------
A k�peket PDFLaTeX eset�n a vesztes�gmentes PNG, valamint a vesztes�ges JPEG form�tumban �rdemes elmenteni. Az EPS (PostScript) vektorgrafikus k�pform�tum beilleszt�s�t a PDFLatex k�zvetlen�l nem t�mogatja. Ehelyett egy lehet�s�g 200 dpi, vagy ann�l nagyobb felbont�sban raszteriz�lni a k�pet, �s PNG form�tumban elmenteni. Az egyes k�pek m�rete �ltal�ban nem, de sok k�p eset�n a dokumentum �sszm�rete �gy m�r szignifik�ns is lehet. A dokumentumban felhaszn�lt k�pf�jlokat a dokumentum forr�sa mellett �rdemes tartani, archiv�lni, mivel ezek hi�ny�ban a dokumentum nem fordul �jra. Ha lehet, a vektorgrafikus k�peket vektorgrafikus form�tumban is �rdemes elmenteni az �jrafelhaszn�lhat�s�g (az �tszerkeszthet�s�g) �rdek�ben.

Kapcsol�si rajzok legt�bbsz�r kim�solhat�k egy vektorgrafikus programba (pl. CorelDraw) �s onnan nagyobb felbont�ssal raszteriz�lva kimenthat�k PNG form�tumban. Ugyanakkor kiv�l� �br�k k�sz�thet�k Microsoft Visio vagy hasonl� program haszn�lat�val is: Visio-b�l az �br�k k�zvetlen�l PNG-be is menthet�k.

Lehet�s�geink Matlab �br�k eset�n:
\begin{itemize}
	\item K�perny�lop�s (\emph{screenshot}) is elfogadhat� min�s�g� lehet a dokumentumban, de �ltal�ban jobb felbont�st is el lehet �rni m�s m�dszerrel.
	\item A Matlab �br�t a \verb+File/Save As+ opci�val lementhetj�k PNG form�tumban (ugyanaz itt is �rv�nyes, mint kor�bban, ez�rt nem javasoljuk).
	\item A Matlab �br�t az \verb+Edit/Copy figure+ opci�val kim�solhatjuk egy vektorgrafikus programba is �s onnan nagyobb felbont�ssal raszteriz�lva kimenthatj�k PNG form�tumban (nem javasolt).
	\item Javasolt megold�s: az �br�t a \verb+File/Save As+ opci�val EPS \emph{vektorgrafikus} form�tumban elmentj�k, PDF-be konvert�lva beillesztj�k a dolgozatba.
\end{itemize}
Az EPS k�p az \verb+epstopdf+ programmal\footnote{a kor�bban eml�tett \LaTeX-disztrib�ci�kban megtal�lhat�} konvert�lhat� PDF form�tumba. C�lszer� egy batch-f�jlt k�sz�teni az �sszes EPS �bra leford�t�s�ra az al�bbi m�don (ez Windows alatt m�k�dik).
\begin{lstlisting}[frame=single,float=!ht]
@echo off
for %%j in (*.eps) do (
echo converting file "%%j"
epstopdf "%%j"
)
echo done .
\end{lstlisting}

Egy ilyen parancsf�jlt (\verb+convert.cmd+) elhelyezt�k a sablon \verb+figures\eps+ k�nyvt�r�ba, �gy a felhaszn�l�nak csak annyi a dolga, hogy a \verb+figures\eps+ k�nyvt�rba kimenti az EPS form�tum� vektorgrafikus k�pet, majd lefuttatja a \verb+convert.cmd+ parancsf�jlt, ami PDF-be konvert�lja az EPS f�jlt.

Ezek ut�n a PDF-�br�t ugyan�gy lehet a dokumentumba beilleszteni, mint a PNG-t vagy a JPEG-et. A megold�s el�nye, hogy a leford�tott dokumentumban is vektorgrafikusan t�rol�dik az �bra, �gy a m�rete j�val kisebb, mintha raszteriz�ltuk volna beilleszt�s el�tt. Ez a m�dszer minden -- az EPS form�tumot ismer� -- vektorgrafikus program (pl. CorelDraw) eset�n is haszn�lhat�.

A k�pek beilleszt�s�re az \sectref{LatexTools}. fejezetben mutattunk be p�ld�t (\figref{TexnicCenter}~�bra). Az el�z� mondatban egy�ttal az automatikusan felold�d� �brahivatkoz�sra is l�thatunk p�ld�t. T�bb k�pf�jlt is beilleszthet�nk egyetlen �br�ba. Az egyes k�pek k�z�tti horizont�lis �s vertik�lis marg�t metrikusan szab�lyozhatjuk (\figref{HVSpaces}~�bra). Az �br�k elhelyez�s�t sz�mtalan tipogr�fiai szab�ly egyidej� teljes�t�s�vel a ford�t� maga v�gzi, a dokumentum �r�ja csak preferenci�it jelezheti a ford�t� fel� (olykor ez bossz�s�got is okozhat, ilyenkor pl. a k�p m�ret�vel lehet j�tszani).

\begin{figure}[!ht]
\centering
\includegraphics[width=67mm, keepaspectratio]{figures/TeXnicCenter.png}\hspace{1cm}
\includegraphics[width=67mm, keepaspectratio]{figures/TeXnicCenter.png}\\\vspace{5mm}
\includegraphics[width=67mm, keepaspectratio]{figures/TeXnicCenter.png}\hspace{1cm}
\includegraphics[width=67mm, keepaspectratio]{figures/TeXnicCenter.png}
\caption{T�bb k�pf�jl beilleszt�se eset�n t�rk�z�ket is �rdemes haszn�lni.} 
\label{fig:HVSpaces}
\end{figure}

A t�bl�zatok haszn�lat�ra a \tabref{TabularExample}~t�bl�zat mutat p�ld�t.
A t�bl�zat c�mk�je nem v�letlen�l ker�lt a t�bl�zat f�l�, ez a szokv�nyos.
\begin{table}[ht]
	\footnotesize
	\centering
	\caption{Az �rajel-gener�tor chip �rajel-kimenetei.} \label{tab:SysClocks}
	\begin{tabular}{ | l | c | c |}
	\hline
	�rajel & Frekvencia & C�l pin \\ \hline
	CLKA & 100 MHz & FPGA CLK0\\
	CLKB & 48 MHz  & FPGA CLK1\\
	CLKC & 20 MHz  & Processzor\\
	CLKD & 25 MHz  & Ethernet chip \\
	CLKE & 72 MHz  & FPGA CLK2\\
	XBUF & 20 MHz  & FPGA CLK3\\
	\hline
	\end{tabular}
	\label{tab:TabularExample}
\end{table}


%----------------------------------------------------------------------------
\section{Felsorol�sok �s list�k}
%----------------------------------------------------------------------------
Sz�mozatlan felsorol�sra mutat p�ld�t a jelenlegi bekezd�s:
\begin{itemize}
	\item \emph{els� bajusz:} ide lehetne �rni az els� elem kifej�s�t,
	\item \emph{m�sodik bajusz:} ide lehetne �rni a m�sodik elem kifej�s�t,
	\item \emph{ez meg egy szak�ll:} ide lehetne �rni a harmadik elem kifej�s�t.
\end{itemize}

Sz�mozott felsorol�st is k�sz�thet�nk az al�bbi m�don:
\begin{enumerate}
	\item \emph{els� bajusz:} ide lehetne �rni az els� elem kifej�s�t, �s ez a kifejt�s �gy n�z ki, ha t�bb sorosra sikeredik,
	\item \emph{m�sodik bajusz:} ide lehetne �rni a m�sodik elem kifej�s�t,
	\item \emph{ez meg egy szak�ll:} ide lehetne �rni a harmadik elem kifej�s�t.
\end{enumerate}
A felsorol�sokban sorok v�g�n vessz�, az utols� sor v�g�n pedig pont a szok�sos �r�sjel. Ez al�l kiv�telt k�pezhet, ha az egyes elemek t�bb teljes mondatot tartalmaznak.

List�kban a dolgozat sz�veg�t�l elk�l�n�tend� k�dr�szleteket, programsorokat, pszeudo-k�dokat jelen�thet�nk meg (\listref{Example}~lista). 
\begin{lstlisting}[frame=single,float=!ht,caption=A fenti sz�mozott felsorol�s \LaTeX- forr�sk�dja, label=listing:Example]
\begin{enumerate}
	\item \emph{els� bajusz:} ide lehetne �rni az els� elem kifej�s�t, 
	�s ez a kifejt�s �gy n�z ki, ha t�bb sorosra sikeredik,
	\item \emph{m�sodik bajusz:} ide lehetne �rni a m�sodik elem kifej�s�t,
	\item \emph{ez meg egy szak�ll:} ide lehetne �rni a harmadik elem kifej�s�t.
\end{enumerate}
\end{lstlisting}
A lista keret�t, h�tt�rsz�n�t, eg�sz st�lus�t megv�laszthatjuk. R�ad�sul k�l�nf�le programnyelveket �s a nyelveken bel�l kulcsszavakat is defini�lhatunk, ha sz�ks�ges. Err�l b�vebbet a \verb+listings+ csomag hivatalos le�r�s�ban tal�lhatunk.

%----------------------------------------------------------------------------
\section{K�pletek}
%----------------------------------------------------------------------------
Ha egy formula nem t�ls�gosan hossz�, �s nem akarjuk hivatkozni a sz�vegb�l, mint p�ld�ul a $e^{i\pi}+1=0$ k�plet, \emph{sz�vegk�zi k�pletk�nt} szok�s le�rni. Csak, hogy m�sik p�ld�t is l�ssunk, az $U_i=-d\Phi/dt$ Faraday-t�rv�ny a $\rot E=-\frac{dB}{dt}$ differenci�lis alakban adott Maxwell-egyenlet fel�letre vett integr�lj�b�l vezethet� le. L�that�, hogy a \LaTeX-ford�t� a sork�z�ket betartja, �gy a sz�veg szed�se eszt�tikus marad sz�vegk�zi k�pletek haszn�lata eset�n is.

K�pletek eset�n az �ltal�nos konvenci�, hogy a kisbet�k skal�rt, a kis f�lk�v�r bet�k ($\mathbf{v}$) oszlopvektort -- �s ennek megfelel�en $\mathbf{v}^T$ sorvektort -- a kapit�lis f�lk�v�r bet�k ($\mathbf{V}$) m�trixot jel�lnek. Ha ett�l el szeretn�nk t�rni, akkor az alkalmazni k�v�nt jel�l�sm�dot c�lszer� k�l�n alfejezetben defini�lni. Ennek megfelel�en, amennyiben $\mathbf{y}$ jel�li a m�r�sek vektor�t, $\mathbf{\vartheta}$ a param�terek vektor�t �s $\hat{\mathbf{y}}=\mathbf{X}\vartheta$ a param�terekben line�ris modellt, akkor a \emph{Least-Squares} �rtelemben optim�lis param�terbecsl� $\hat{\mathbf{\vartheta}}_{LS}=(\mathbf{X}^T\mathbf{X})^{-1}\mathbf{X}^T\mathbf{y}$ lesz.

Emellett kiemelt, sorsz�mozott k�pleteket is megadhatunk, enn�l az \verb+equation+ �s a \verb+eqnarray+ k�rnyezetek helyett a korszer�bb \verb+align+ k�rnyezet alkalmaz�s�t javasoljuk (t�bb okb�l, k�l�nf�le probl�m�k elker�l�se v�gett, amelyekre most nem t�r�nk ki). Teh�t
\begin{align}
\dot{\mathbf{x}}&=\mathbf{A}\mathbf{x}+\mathbf{B}\mathbf{u},\\
\mathbf{y}&=\mathbf{C}\mathbf{x},
\end{align}
ahol $\mathbf{x}$ az �llapotvektor, $\mathbf{y}$ a m�r�sek vektora �s $\mathbf{A}$, $\mathbf{B}$ �s $\mathbf{C}$ a rendszert le�r� param�term�trixok. Figyelj�k meg, hogy a k�t egyenletben az egyenl�s�gjelek egym�shoz igaz�tva jelennek meg, mivel a mindkett�t az \& karakter el�zi meg a k�dban. Lehet�s�g van sz�mozatlan kiemelt k�plet haszn�lat�ra is, p�ld�ul
\begin{align}
\dot{\mathbf{x}}&=\mathbf{A}\mathbf{x}+\mathbf{B}\mathbf{u},\nonumber\\
\mathbf{y}&=\mathbf{C}\mathbf{x}\nonumber.
\end{align}
M�trixok fel�r�s�ra az $\mathbf{A}\mathbf{x}=\mathbf{b}$ inhomog�n line�ris egyenlet r�szletes kifejt�s�vel mutatunk p�ld�t:
\begin{align}
\begin{bmatrix}
a_{11} & a_{12} & \dots & a_{1n}\\
a_{21} & a_{22} & \dots & a_{2n}\\
\vdots & \vdots & \ddots & \vdots\\
a_{m1} & a_{m2} & \dots & a_{mn}
\end{bmatrix}
\begin{pmatrix}x_1\\x_2\\\vdots\\x_n\end{pmatrix}=
\begin{pmatrix}b_1\\b_2\\\vdots\\b_m\end{pmatrix}.
\end{align}
A \verb+\frac+ utas�t�s hat�konys�g�t egy �ltal�nos m�sodfok� tag �tviteli f�ggv�ny�n kereszt�l mutatjuk be, azaz
\begin{align}
W(s)=\frac{A}{1+2T\xi s+s^2T^2}.
\end{align}
A matematikai m�d minden szimb�lum�nak �s k�pess�g�nek a bemutat�s�ra term�szetesen itt nincs lehet�s�g, de gyors referenciak�nt hat�konyan haszn�lhat�k a k�vetkez� linkek:\\
\indent\url{http://www.artofproblemsolving.com/LaTeX/AoPS_L_GuideSym.php},\\
\indent\url{http://www.ctan.org/tex-archive/info/symbols/comprehensive/symbols-a4.pdf},\\
\indent\url{ftp://ftp.ams.org/pub/tex/doc/amsmath/short-math-guide.pdf}.\\
Ez pedig itt egy magyar�zat, hogy mi�rt �rdemes \verb+align+ k�rnyezetet haszn�lni:\\
\indent\url{http://texblog.net/latex-archive/maths/eqnarray-align-environment/}.

%----------------------------------------------------------------------------
\section{Irodalmi hivatkoz�sok}\label{sect:HowtoReference}
%----------------------------------------------------------------------------
Egy \LaTeX dokumentumban az irodalmi hivatkoz�sok defin�ci�j�nak k�t m�dja van. Az egyik a \verb+\thebibliograhy+ k�rnyezet haszn�lata a dokumentum v�g�n, az \verb+\end{document}+ lez�r�s el�tt.
\begin{lstlisting}[frame=single,float=!ht]
\begin{thebibliography}{9}

\bibitem{Lamport94} Leslie Lamport, \emph{\LaTeX: A Document Preparation System}. 
Addison Wesley, Massachusetts, 2nd Edition, 1994.

\end{thebibliography}
\end{lstlisting}

Ezek ut�n a dokumentumban a \verb+\cite{Lamport94}+ utas�t�ssal hivatkozhatunk a forr�sra. A fenti megad�s viszonylag k�tetlen, a szerz� maga form�zza az irodalomjegyz�ket. 

Egy sokkal professzion�lisabb m�dszer a BiB\TeX~haszn�lata, ez�rt ez a sablon is ezt t�mogatja. Ebben az esetben egy k�l�n sz�veges adatb�zisban defini�ljuk a forr�smunk�kat, �s egy k�l�n st�lusf�jl hat�rozza meg az irodalomjegyz�k kin�zet�t. Ez, �sszhangban azzal, hogy k�l�n form�tumkonvenci� hat�rozza meg a foly�irat-, a k�nyv-, a konferenciacikk- stb. hivatkoz�sok kin�zet�t az irodalomjegyz�kben (a sablon haszn�lata eset�n ezzel nem is kell foglalkoznia a hallgat�nak, de az eredm�nyt c�lszer� ellen�rizni). A felhaszn�lt hivatkoz�sok adatb�zisa egy \verb+.bib+ kiterjeszt�s� sz�veges f�jl, amelynek szerkezet�t a \listref{Bibtex} k�dr�szlet demonstr�lja. A forr�smunk�k bevitelekor a sor v�gi vessz�k k�l�n figyelmet ig�nyelnek, mert hi�nyuk a BiB\TeX-ford�t� hiba�zenet�t eredm�nyezi. A forr�smunk�kat t�pus szerinti kulcssz� vezeti be (\verb+@book+ k�nyv, \verb+@inproceedings+ konferenciakiadv�nyban megjelent cikk, \verb+@article+ foly�iratban megjelent cikk, \verb+@techreport+ valamelyik egyetem gondoz�s�ban megjelent m�szaki tanulm�ny, \verb+@manual+ m�szaki dokument�ci� eset�n stb.). Nemcsak a megjelen�s st�lusa, de a k�telez�en megadand� mez�k is t�pusr�l-t�pusra v�ltoznak. Egy j�l haszn�lhat� referencia a \url{http://en.wikipedia.org/wiki/BibTeX} oldalon tal�lhat�.
\begin{lstlisting}[frame=single,float=!ht,caption=P�lda sz�veges irodalomjegyz�k-adatb�zisra BiBTeX haszn�lata eset�n., label=listing:Bibtex]
@BOOK{Wettl04,
  author="Ferenc Wettl and Gyula Mayer and P�ter Szab�",
  title="\LaTeX~k�zik�nyv",
  publisher="Panem K�nyvkiad�",
  year=2004
}
@ARTICLE{Candy86,
  author ="James C. Candy",
  title  ="Decimation for Sigma Delta Modulation",
  journal="{IEEE} Trans.\ on Communications",
  volume =34,
  number =1,
  pages  ="72--76",
  month  =jan,
  year   =1986,
}
@INPROCEEDINGS{Lee87,
  author =       "Wai L. Lee and Charles G. Sodini",
  title =        "A Topology for Higher Order Interpolative Coders",
  booktitle =    "Proc.\ of the IEEE International Symposium on 
  Circuits and Systems",
  year =         1987,
  vol =          2,
  month =        may # "~4--7",
  address =      "Philadelphia, PA, USA",
  pages =        "459--462"
}
@PHDTHESIS{KissPhD,
  author =   "Peter Kiss",
  title =    "Adaptive Digital Compensation of Analog Circuit Imperfections 
  for Cascaded Delta-Sigma Analog-to-Digital Converters",
  school =   "Technical University of Timi\c{s}oara, Romania",
  month =    apr,
  year =     2000
}
@MANUAL{Schreier00,
  author = "Richard Schreier",
  title  = "The Delta-Sigma Toolbox v5.2",
  organization = "Oregon State University",
  year   = 2000,
  month  = jan,
  note   ="\newline URL: http://www.mathworks.com/matlabcentral/fileexchange/"
}
@MISC{DipPortal,
	author="Budapesti {M}�szaki �s {G}azdas�gtudom�nyi {E}gyetem 
	{V}illamosm�rn�ki �s {I}nformatikai {K}ar",
  title="{D}iplomaterv port�l (2011 febru�r 26.)",
  howpublished="\url{http://diplomaterv.vik.bme.hu/}",
}}
\end{lstlisting}

A st�lusf�jl egy \verb+.sty+ kiterjeszt�s� f�jl, de ezzel l�nyeg�ben nem kell foglalkozni, mert vannak be�p�tett st�lusok, amelyek j�l haszn�lhat�k. Ez a sablon a BiB\TeX-et haszn�lja, a hozz� tartoz� adatb�zisf�jl a \verb+mybib.bib+ f�jl. Megfigyelhet�, hogy az irodalomjegyz�ket a dokumentum v�g�re (a \verb+\end{document}+ utas�t�s el�) beillesztett \verb+\bibliography{mybib}+ utas�t�ssal hozhatjuk l�tre, a st�lus�t pedig ugyanitt a  \verb+\bibliographystyle{plain}+ utas�t�ssal adhatjuk meg. Ebben az esetben a \verb+plain+ el�re defini�lt st�lust haszn�ljuk (a sablonban is ezt �ll�tottuk be). A \verb+plain+ st�luson k�v�l term�szetesen sz�mtalan m�s el�re defini�lt st�lus is l�tezik. Mivel a \verb+.bib+ adatb�zisban ezeket megadtuk, a BiB\TeX-ford�t� is meg tudja k�l�nb�ztetni a szerz�t a c�mt�l �s a kiad�t�l, �s ez alapj�n automatikusan gener�l�dik az irodalomjegyz�k a st�lusf�jl �ltal meghat�rozott st�lusban.

Az egyes forr�smunk�kra a sz�vegb�l tov�bbra is a \verb+\cite+ paranccsal tudunk hivatkozni, �gy a \listref{Bibtex} k�dr�szlet eset�n a hivatkoz�sok rendre \verb+\cite{Wettl04}+, \verb+\cite{Candy86}+, \verb+\cite{Lee87}+, \verb+\cite{KissPhD}+, \verb+\cite{Schreirer00}+ �s \verb+\cite{DipPortal}+. Az irodalomjegyz�kben alap�rtelmez�sben csak azok a forr�smunk�k jelennek meg, amelyekre tal�lhat� hivatkoz�s a sz�vegben, �s ez �gy alapvet�en helyes is, hiszen olyan forr�smunk�kat nem illik az irodalomjegyz�kbe �rni, amelyekre nincs hivatkoz�s.

Mivel a ford�t�si folyamat sor�n t�bb l�p�sben old�dnak fel a szimb�lumok, ez�rt gyakran t�bbsz�r (TeXLive �s TeXnicCenter eset�n 2-3-szor) is le kell ford�tani a dokumentumot. Ilyenkor ez els� 1-2 ford�t�s esetleg szimb�lum-felold�sra vonatkoz� figyelmeztet� �zenettel z�rul. Ha hiba�zenettel z�rul b�rmelyik ford�t�s, akkor nincs �rtelme megism�telni, hanem a hib�t kell megkeresni. A \verb+.bib+ f�jl megv�ltoztat�skor sokszor nincs hat�sa a v�ltoztat�snak azonnal, mivel nem mindig fut �jra a BibTeX ford�t�. Ez�rt c�lszer� a v�ltoztat�s ut�n azt manu�lisan is lefuttatni (TeXnicCenter eset�n \verb+Build/BibTeX+).

Hogy a sz�vegbe �gyazott hivatkoz�sok kin�zet�t demonstr�ljuk, itt most sorban meghivatkozzuk a \cite{Wettl04}, \cite{Candy86}, \cite{Lee87}, \cite{KissPhD} �s az \cite{Schreier00} forr�smunk�t, valamint az \cite{DipPortal} weboldalt.

Megjegyzend�, hogy az �kezetes magyar bet�ket is tartalmaz� \verb+.bib+ f�jl az \verb+inputenc+ csomaggal bet�lt�tt \verb+latin2+ bet�k�szlet miatt ford�that�. Ugyanez a \verb+.bib+ f�jl hiba�zenettel fordul egy olyan dokumentumban, ami nem tartalmazza a \verb+\usepackage[latin2]{inputenc}+ sort. Speci�lis ig�ny eset�n az irodalmi adatb�zis �ltal�nosabb �rv�ny�v� tehet�, ha az �kezetes bet�ket speci�lis latex karakterekkel helyettes�tj�k a \verb+.bib+ f�jlban, pl. � helyett \verb+\'{a}+-t vagy � helyett \verb+\H{o}+-t �runk. 

Oldalt�r�s k�vetkezik (ld. forr�s).
\newpage

%----------------------------------------------------------------------------
\section{A dolgozat szerkezete �s a forr�sf�jlok}
%----------------------------------------------------------------------------
A diplomatervsablon (a kari ir�nyelvek szerint) az al�bbi f� fejezetekb�l �ll:
\begin{enumerate}
	\item 1 oldalas \emph{t�j�koztat�} a szakdolgozat/diplomaterv szerkezet�r�l (\verb+guideline.tex+), ami a v�gs� dolgozatb�l t�rlend�,
	\item \emph{feladatki�r�s} (\verb+project.tex+), a dolgozat nyomtatott verz�j�ban ennek a hely�re ker�l a tansz�k �ltal kiadott, a tansz�kvezet� �ltal al��rt feladatki�r�s, a dolgozat elektronikus verzi�j�ba pedig a feladatki�r�s egy�ltal�n ne ker�lj�n bele, azt k�l�n t�lti fel a tansz�k a diplomaterv-honlapra,
	\item \emph{c�moldal} (\verb+titlepage.tex+),
	\item \emph{tartalomjegyz�k} (\verb+diploma.tex+),
	\item a diplomatervez� \emph{nyilatkozat}a az �n�ll� munk�r�l (\verb+declaration.tex+),
	\item 1-2 oldalas tartalmi \emph{�sszefoglal�} magyarul �s angolul, illetve elk�sz�thet� m�g tov�bbi nyelveken is (\verb+abstract.tex+),
	\item \emph{bevezet�s}: a feladat �rtelmez�se, a tervez�s c�lja, a feladat indokolts�ga, a diplomaterv fel�p�t�s�nek r�vid �sszefoglal�sa (\verb+introduction.tex+),
	\item sorsz�mmal ell�tott \emph{fejezetek}: a feladatki�r�s pontos�t�sa �s r�szletes elemz�se, el�zm�nyek (irodalomkutat�s, hasonl� alkot�sok), az ezekb�l levonhat� k�vetkeztet�sek, a tervez�s r�szletes le�r�sa, a d�nt�si lehet�s�gek �rt�kel�se �s a v�lasztott megold�sok indokl�sa, a megtervezett m�szaki alkot�s �rt�kel�se, kritikai elemz�se, tov�bbfejleszt�si lehet�s�gek (\verb+chapter{1,2..n}.tex+),
	\item esetleges \emph{k�sz�netnyilv�n�t�s}ok (\verb+acknowledgement.tex+),
	\item r�szletes �s pontos \emph{irodalomjegyz�k} (ez a sablon eset�ben automatikusan gener�l�dik a \verb+diploma.tex+ f�jlban elhelyezett \verb+\bibliography+ utas�t�s hat�s�ra, a \sectref{HowtoReference}. fejezetben le�rtak szerint),
	\item \emph{f�ggel�kek} (\verb+appendices.tex+).
\end{enumerate}

A sablonban a fejezetek a \verb+diploma.tex+ f�jlba vannak beillesztve \verb+\include+ utas�t�sok seg�ts�g�vel. Lehet�s�g van arra, hogy csak az �ppen szerkeszt�s alatt �ll� \verb+.tex+ f�jlt ford�tsuk le, ezzel ler�vid�tve a ford�t�si folyamatot. Ezt a lehet�s�get az al�bbi k�dr�szlet biztos�tja a \verb+diploma.tex+ f�jlban.
\begin{lstlisting}[frame=single,float=!ht]
\includeonly{
	guideline,%
	project,%
	titlepage,%
	declaration,%
	abstract,%
	introduction,%
	chapter1,%
	chapter2,%
	chapter3,%
	acknowledgement,%
	appendices,%
}
\end{lstlisting}

Ha az al�bbi k�dr�szletben az egyes sorokat a \verb+%+ szimb�lummal kikommentezz�k, akkor a megfelel� \verb+.tex+ f�jl nem fordul le. Az oldalsz�mok �s a tartalomjegy�k term�szetesen csak akkor billennek helyre, ha a teljes dokumentumot leford�tjuk.

%----------------------------------------------------------------------------
\newpage
\section{Alapadatok megad�sa}
%----------------------------------------------------------------------------
A diplomaterv alapadatait (c�m, szerz�, konzulens, konzulens titulusa) a \verb+diploma.tex+ f�jlban lehet megadni az al�bbi k�dr�szlet m�dos�t�s�val.
\begin{lstlisting}[frame=single,float=!ht]
\newcommand{\vikszerzo}{B�dis-Szomor� Andr�s}
\newcommand{\vikkonzulens}{dr.~Konzulens Elem�r}
\newcommand{\vikcim}{Elektronikus terel�k}
\newcommand{\viktanszek}{M�r�stechnika �s Inform�ci�s Rendszerek Tansz�k}
\newcommand{\vikdoktipus}{Diplomaterv}
\newcommand{\vikdepartmentr}{B�dis-Szomor� Andr�s}
\end{lstlisting}

%----------------------------------------------------------------------------
\section{�j fejezet �r�sa}
%----------------------------------------------------------------------------
A f�fejezetek k�l�n \verb+chapter{1..n}.tex+ f�jlban foglalnak helyet. A sablonhoz 3 fejezet k�sz�lt. Tov�bbi f�fejezeteket �gy hozhatunk l�tre, ha �j \verb+chapter{i}.tex+ f�jlt k�sz�t�nk a fejezet sz�m�ra, �s a \verb+diploma.tex+ f�jlban, a \verb+\include+ �s \verb+\includeonly+ utas�t�sok argumentum�ba felvessz�k az �j \verb+.tex+ f�jl nev�t.






%----------------------------------------------------------------------------
\chapter*{Acknowledgement}\addcontentsline{toc}{chapter}{Acknowledgement}
%----------------------------------------------------------------------------

Fanni,Pali,Sch�nherz

Ez nem k�telez�, ak�r t�r�lhet� is. Ha a szerz� sz�ks�g�t �rzi, itt lehet k�sz�netet nyilv�n�tani azoknak, akik hozz�j�rultak munk�jukkal ahhoz, hogy a hallgat� a szakdolgozatban vagy diplomamunk�ban le�rt feladatokat sikeresen elv�gezze. A konzulensnek val� k�sz�netnyilv�n�t�s sem k�telez�, a konzulensnek hivatalosan is dolga, hogy a hallgat�t konzult�lja.

%\listoffigures\addcontentsline{toc}{chapter}{�br�k jegyz�ke}
%\listoftables\addcontentsline{toc}{chapter}{T�bl�zatok jegyz�ke}

\bibliography{mybib}
\addcontentsline{toc}{chapter}{Irodalomjegyz�k}
\bibliographystyle{plain}

%----------------------------------------------------------------------------
\appendix
%----------------------------------------------------------------------------
\chapter*{Appendix}\addcontentsline{toc}{chapter}{Appendix}
\setcounter{chapter}{1}  % a fofejezet-szamlalo az angol ABC 6. betuje (F) lesz
\setcounter{equation}{0} % a fofejezet-szamlalo az angol ABC 6. betuje (F) lesz
\numberwithin{equation}{section}
\numberwithin{figure}{section}
\numberwithin{lstlisting}{section}
%\numberwithin{tabular}{section}
%----------------------------------------------------------------------------
\section{CAN fingerprint RUL calculation workflow:}
%----------------------------------------------------------------------------
\label{appendix:CANFPRULCalc}
%\begin{figure}[!ht]
%\centering
%\includegraphics[width=150mm, keepaspectratio]{figures/TeXnicCenter.png}
%\caption{A TeXnicCenter Windows alap� \LaTeX-szerkeszt�.} 
%\end{figure}
	\begin{figure}[!ht]
		\centering
		\includegraphics[width=150mm, angle =-90, keepaspectratio]{figures/CAN_fingerprint_RUL_calculation.png}
		\caption{Calculating RUL from cleaned data} 
	\end{figure}
%----------------------------------------------------------------------------
\clearpage\section{Electrical failure prediction preprocessing workflow}
%----------------------------------------------------------------------------
\label{appendix:FaultPredPreProc}
\begin{figure}[!ht]
		\centering
		\includegraphics[width=150mm, angle =-90,keepaspectratio]{figures/Electrical_fail_prediction_preprocessing_workflow.png}
		\caption{Cleaning and ordering step by step} 
\end{figure}

%A Pitagorasz-t�telb�l levezetve
%\begin{align}
%c^2=a^2+b^2=42.
%\end{align}
%A Faraday-indukci�s t�rv�nyb�l levezetve
%\begin{align}
%\rot E=-\frac{dB}{dt}\hspace{1cm}\longrightarrow \hspace{1cm}
%U_i=\oint\limits_\mathbf{L}{\mathbf{E}\mathbf{dl}}=-\frac{d}{dt}\int\limits_A{\mathbf{B}\mathbf{da}}=42.
%\end{align}
%----------------------------------------------------------------------------
\clearpage\section{Electrical failure prediction RUL calculation workflow}
%----------------------------------------------------------------------------
\label{appendix:FaultPredRULCalc}
		\#TODO explain red flags in short
		\begin{figure}[!ht]
		\centering
		\includegraphics[width=150mm, angle =-90,keepaspectratio]{figures/Electrical_failure_pred_RUL.png}
		\caption{Electrical failure prediction RUL calculation workflow} 
		\end{figure}




\label{page:last}
\end{document}
