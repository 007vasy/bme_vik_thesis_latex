\documentclass[11pt,a4paper,oneside]{report}             % Single-side
%\documentclass[11pt,a4paper,twoside,openright]{report}  % Duplex

%\PassOptionsToPackage{chapternumber=Huordinal}{magyar.ldf}
\usepackage{t1enc}
\usepackage[latin2]{inputenc}
\usepackage{amsmath}
\usepackage{amssymb}
\usepackage{enumerate}
\usepackage{enumitem}
\usepackage[thmmarks]{ntheorem}
\usepackage{graphics}
\usepackage{epsfig}
\usepackage{listings}
\usepackage{color}
%\usepackage{fancyhdr}
\usepackage{lastpage}
\usepackage{anysize}
\usepackage[english]{babel}
\usepackage{sectsty}
\usepackage{ctable}
\usepackage{setspace}  % Ettol a tablazatok, abrak, labjegyzetek maradnak 1-es sorkozzel!
\usepackage[hang]{caption}
\usepackage{hyperref}
\usepackage[nodayofweek]{datetime}
\usepackage{float}
\floatstyle{boxed} 
\restylefloat{figure}



%--------------------------------------------------------------------------------------
% Main variables
%--------------------------------------------------------------------------------------
\newcommand{\vikszerzo}{Bence Vass}
\newcommand{\vikkonzulens}{Dr.~P�l Varga}
\newcommand{\vikcim}{Data processing methods for proactive maintenance of industrial systems}
\newcommand{\viktanszek}{Department of Telecommunications and Media Informatics}
\newcommand{\vikdoktipus}{Thesis}
\newcommand{\vikdepartmentr}{Bence Vass}

%--------------------------------------------------------------------------------------
% Page layout setup
%--------------------------------------------------------------------------------------
% we need to redefine the pagestyle plain
% another possibility is to use the body of this command without \fancypagestyle
% and use \pagestyle{fancy} but in that case the special pages
% (like the ToC, the References, and the Chapter pages)remain in plane style

\pagestyle{plain}
\setlength{\parindent}{0pt} % �ttekinthet�bb, angol nyelv� dokumentumokban jellemz�
\setlength{\parskip}{8pt plus 3pt minus 3pt} % �ttekinthet�bb, angol nyelv� dokumentumokban jellemz�
%\setlength{\parindent}{12pt} % magyar nyelv� dokumentumokban jellemz�
%\setlength{\parskip}{0pt}    % magyar nyelv� dokumentumokban jellemz�

\marginsize{35mm}{25mm}{15mm}{15mm} % anysize package
\setcounter{secnumdepth}{0}
\sectionfont{\large\upshape\bfseries}
\setcounter{secnumdepth}{2}
\singlespacing
\frenchspacing

%--------------------------------------------------------------------------------------
%	Setup hyperref package
%--------------------------------------------------------------------------------------
\hypersetup{
    bookmarks=true,            % show bookmarks bar?
    unicode=false,             % non-Latin characters in Acrobat�s bookmarks
    pdftitle={\vikcim},        % title
    pdfauthor={\vikszerzo},    % author
    pdfsubject={\vikdoktipus}, % subject of the document
    pdfcreator={\vikszerzo},   % creator of the document
    pdfproducer={Producer},    % producer of the document
    pdfkeywords={keywords},    % list of keywords
    pdfnewwindow=true,         % links in new window
    colorlinks=true,           % false: boxed links; true: colored links
    linkcolor=black,           % color of internal links
    citecolor=black,           % color of links to bibliography
    filecolor=black,           % color of file links
    urlcolor=black             % color of external links
}

%--------------------------------------------------------------------------------------
% Set up listings
%--------------------------------------------------------------------------------------
\lstset{
	basicstyle=\scriptsize\ttfamily, % print whole listing small
	keywordstyle=\color{black}\bfseries\underbar, % underlined bold black keywords
	identifierstyle=, 					% nothing happens
	commentstyle=\color{white}, % white comments
	stringstyle=\scriptsize\sffamily, 			% typewriter type for strings
	showstringspaces=false,     % no special string spaces
	aboveskip=3pt,
	belowskip=3pt,
	columns=fixed,
	backgroundcolor=\color{lightgray},
} 		
\def\lstlistingname{lista}	

%--------------------------------------------------------------------------------------
%	Some new commands and declarations
%--------------------------------------------------------------------------------------
\newcommand{\code}[1]{{\upshape\ttfamily\scriptsize\indent #1}}

% define references
\newcommand{\figref}[1]{\ref{fig:#1}.}
\renewcommand{\eqref}[1]{(\ref{eq:#1})}
\newcommand{\listref}[1]{\ref{listing:#1}.}
\newcommand{\sectref}[1]{\ref{sect:#1}}
\newcommand{\tabref}[1]{\ref{tab:#1}.}

\DeclareMathOperator*{\argmax}{arg\,max}
%\DeclareMathOperator*[1]{\floor}{arg\,max}
\DeclareMathOperator{\sign}{sgn}
\DeclareMathOperator{\rot}{rot}
\definecolor{lightgray}{rgb}{0.95,0.95,0.95}

\author{\vikszerzo}
\title{\viktitle}
\includeonly{
	%guideline,%
	%project,%
	titlepage,%
	declaration,%
	abstract,%
	introduction,%
	%chapter1,%
	%chapter2,%
	%chapter3,%
	technical_background,%
	design,%
	implementation,%
	verification,%
	conclusion,%
	acknowledgement,%
	appendices,%
}
%--------------------------------------------------------------------------------------
%	Setup captions
%--------------------------------------------------------------------------------------
\captionsetup[figure]{
%labelsep=none,
%font={footnotesize,it},
%justification=justified,
width=.75\textwidth,
aboveskip=10pt}

\renewcommand{\captionlabelfont}{\small\bf}
\renewcommand{\captionfont}{\footnotesize\it}

%--------------------------------------------------------------------------------------
% Table of contents and the main text
%--------------------------------------------------------------------------------------
\begin{document}
\singlespacing
%%--------------------------------------------------------------------------------------
% Rovid formai es tartalmi tajekoztato
%--------------------------------------------------------------------------------------

\footnotesize
\begin{center}
\large
\textbf{\Large �ltal�nos inform�ci�k, a diplomaterv szerkezete}\\
\end{center}

A diplomaterv szerkezete a BME Villamosm�rn�ki �s Informatikai Kar�n:
\begin{enumerate}
\item	Diplomaterv feladatki�r�s
\item	C�moldal
\item	Tartalomjegyz�k
\item	A diplomatervez� nyilatkozata az �n�ll� munk�r�l �s az elektronikus adatok kezel�s�r�l
\item	Tartalmi �sszefoglal� magyarul �s angolul
\item	Bevezet�s: a feladat �rtelmez�se, a tervez�s c�lja, a feladat indokolts�ga, a diplomaterv fel�p�t�s�nek r�vid �sszefoglal�sa
\item	A feladatki�r�s pontos�t�sa �s r�szletes elemz�se
\item	El�zm�nyek (irodalomkutat�s, hasonl� alkot�sok), az ezekb�l levonhat� k�vetkeztet�sek
\item	A tervez�s r�szletes le�r�sa, a d�nt�si lehet�s�gek �rt�kel�se �s a v�lasztott megold�sok indokl�sa
\item	A megtervezett m�szaki alkot�s �rt�kel�se, kritikai elemz�se, tov�bbfejleszt�si lehet�s�gek
\item	Esetleges k�sz�netnyilv�n�t�sok
\item	R�szletes �s pontos irodalomjegyz�k
\item	F�ggel�k(ek)
\end{enumerate}

Felhaszn�lhat� a k�vetkez� oldalt�l kezd�d� \LaTeX-Diplomaterv sablon dokumentum tartalma. 

A diplomaterv szabv�nyos m�ret� A4-es lapokra ker�lj�n. Az oldalak t�k�rmarg�val k�sz�ljenek (mindenhol 2.5cm, baloldalon 1cm-es k�t�ssel). Az alap�rtelmezett bet�k�szlet a 12 pontos Times New Roman, m�sfeles sork�zzel.

Minden oldalon - az els� n�gy szerkezeti elem kiv�tel�vel - szerepelnie kell az oldalsz�mnak.

A fejezeteket decim�lis beoszt�ssal kell ell�tni. Az �br�kat a megfelel� helyre be kell illeszteni, fejezetenk�nt decim�lis sz�mmal �s kifejez� c�mmel kell ell�tni. A fejezeteket decim�lis al�oszt�ssal sz�mozzuk, maxim�lisan 3 al�oszt�s m�lys�gben (pl. 2.3.4.1.). Az �br�kat, t�bl�zatokat �s k�pleteket c�lszer� fejezetenk�nt k�l�n sz�mozni (pl. 2.4. �bra, 4.2 t�bl�zat vagy k�pletn�l (3.2)). A fejezetc�meket igaz�tsuk balra, a norm�l sz�vegn�l viszont haszn�ljunk sorkiegyenl�t�st. Az �br�kat, t�bl�zatokat �s a hozz�juk tartoz� c�met igaz�tsuk k�z�pre. A c�m a jel�lt r�sz alatt helyezkedjen el.

A k�peket lehet�leg rajzol� programmal k�sz�ts�k el, az egyenleteket egyenlet-szerkeszt� seg�ts�g�vel �rj�k le (A \LaTeX~ehhez k�zenfekv� megold�sokat ny�jt).

Az irodalomjegyz�k sz�vegk�zi hivatkoz�sa t�rt�nhet a Harvard-rendszerben (a szerz� �s az �vsz�m megad�s�val) vagy sorsz�mozva. A teljes lista n�vsor szerinti sorrendben a sz�veg v�g�n szerepeljen (sorsz�mozott irodalmi hivatkoz�sok eset�n hivatkoz�si sorrendben). A szakirodalmi forr�sok c�meit azonban mindig az eredeti nyelven kell megadni, esetleg z�r�jelben a ford�t�ssal. A list�ban szerepl� valamennyi publik�ci�ra hivatkozni kell a sz�vegben (a \LaTeX-sablon a Bib\TeX~seg�ts�g�vel mindezt automatikusan kezeli). Minden publik�ci� a szerz�k ut�n a k�vetkez� adatok szerepelnek: foly�irat cikkekn�l a pontos c�m, a foly�irat c�me, �vfolyam, sz�m, oldalsz�m t�l-ig. A foly�irat c�meket csak akkor r�vid�ts�k, ha azok nagyon k�zismertek vagy nagyon hossz�ak. Internet hivatkoz�sok megad�sakor fontos, hogy az el�r�si �t el�tt megadjuk az oldal tulajdonos�t �s tartalm�t (mivel a link egy id� ut�n ak�r el�rhetetlenn� is v�lhat), valamint az el�r�s id�pontj�t.

\vspace{5mm}
Fontos:
\begin{itemize}
	\item A szakdolgozat k�sz�t� / diplomatervez� nyilatkozata (a jelen sablonban szerepl� sz�vegtartalommal) k�telez� el��r�s Karunkon ennek hi�ny�ban a szakdolgozat/diplomaterv nem b�r�lhat� �s nem v�dhet� !
	\item Mind a dolgozat, mind a mell�klet maxim�lisan 15 MB m�ret� lehet !
\end{itemize}

\vspace{5mm}
\begin{center}
J� munk�t, sikeres szakdolgozat k�sz�t�st ill. diplomatervez�st k�v�nunk !
\end{center}

\normalsize

%%--------------------------------------------------------------------------------------
% Feladatkiiras (a tanszeken atveheto, kinyomtatott valtozat)
%--------------------------------------------------------------------------------------
\clearpage
\begin{center}
\large
\textbf{FELADATKI�R�S}\\
\end{center}

A feladatki�r�st a tansz�ki adminisztr�ci�ban lehet �tvenni, �s a leadott munk�ba eredeti, tansz�ki pecs�ttel ell�tott �s a tansz�kvezet� �ltal al��rt lapot kell belef�zni (ezen oldal \emph{helyett}, ez az oldal csak �tmutat�s). Az elektronikusan felt�lt�tt dolgozatban m�r nem kell beleszerkeszteni ezt a feladatki�r�st.





\pagenumbering{arabic}
\onehalfspacing
%--------------------------------------------------------------------------------------
%	The title page
%--------------------------------------------------------------------------------------
\begin{titlepage}
\begin{center}
\includegraphics[width=60mm,keepaspectratio]{figures/BMElogo.png}\\
\vspace{0.3cm}
\textbf{BUDAPEST UNIVERSITY OF TECHNOLOGY AND ECONOMICS}\\
\textmd{Faculty of Electrical Engineering and Informatics}\\
\textmd{\viktanszek}\\[5cm]

\vspace{0.4cm}
{\huge \bfseries \vikcim}\\[0.8cm]
\vspace{0.5cm}
\textsc{\Large \vikdoktipus}\\[4cm]

\begin{tabular}{cc}
 \makebox[7cm]{\emph{Created by}} & \makebox[7cm]{\emph{Supervisor}} \\
 \makebox[7cm]{\vikszerzo} & \makebox[7cm]{\vikkonzulens}
\end{tabular}

\vfill
{\large \today}
\end{center}
\end{titlepage}



%--------------------------------------------------------------------------------------
% Nyilatkozat
%--------------------------------------------------------------------------------------
\begin{center}
\large
\textbf{HALLGAT�I NYILATKOZAT}\\
\end{center}

Alul�rott \emph{\vikszerzo}, szigorl� hallgat� kijelentem, hogy ezt a szakdolgozatot/ diplomatervet \textcolor{blue}{(nem k�v�nt t�rlend�)} meg nem engedett seg�ts�g n�lk�l, saj�t magam k�sz�tettem, csak a megadott forr�sokat (szakirodalom, eszk�z�k stb.) haszn�ltam fel. Minden olyan r�szt, melyet sz� szerint, vagy azonos �rtelemben, de �tfogalmazva m�s forr�sb�l �tvettem, egy�rtelm�en, a forr�s megad�s�val megjel�ltem.

Hozz�j�rulok, hogy a jelen munk�m alapadatait (szerz�(k), c�m, angol �s magyar nyelv� tartalmi kivonat, k�sz�t�s �ve, konzulens(ek) neve) a BME VIK nyilv�nosan hozz�f�rhet� elektronikus form�ban, a munka teljes sz�veg�t pedig az egyetem bels� h�l�zat�n kereszt�l (vagy autentik�lt felhaszn�l�k sz�m�ra) k�zz�tegye. Kijelentem, hogy a beny�jtott munka �s annak elektronikus verzi�ja megegyezik. D�k�ni enged�llyel titkos�tott diplomatervek eset�n a dolgozat sz�vege csak 3 �v eltelte ut�n v�lik hozz�f�rhet�v�.

\begin{flushleft}
\vspace*{1cm}
Budapest, \today
\end{flushleft}

\begin{flushright}
 \vspace*{1cm}
 \makebox[7cm]{\rule{6cm}{.4pt}}\\
 \makebox[7cm]{\emph{\vikszerzo}}\\
 \makebox[7cm]{hallgat�}
\end{flushright}
\thispagestyle{empty}

\vfill
\clearpage
\thispagestyle{empty} % an empty page


%feladatki�r�s
%----------------------------------------------------------------------------
% Abstract in english
%----------------------------------------------------------------------------
\chapter*{Abstract}\addcontentsline{toc}{chapter}{Abstract}

The purpose of my thesis is studying data science in detail and beside studying its benefits, to develop data-based proactive maintenance methods for industrial devices, tools and machines.

In case of this thesis, two dataset has to be preprocessed, analyzed and with that information, two proactive maintenance method has to be designed, implemented and verified.

For educational and research purposes I used two technologies (R,Python) to reach these ambitious objectives and implemented various data processing and analyzing methods, borrowed from the statistical or data science field. Armed with that knowledge I come up with special processes to help to indicate various type of failure in the examined machines and applications.


\tableofcontents\vfill
%----------------------------------------------------------------------------
\chapter*{Introduction}\addcontentsline{toc}{chapter}{Introduction}
%----------------------------------------------------------------------------
\section{Antendent}
%----------------------------------------------------------------------------
In this current era the interconnectivity and interoperability is transform our environtment, and has ernormous effect on the all industries too. We use more and more sensors, and connect them to collect usefull data on purpose to predict the future.

In this thesis i would like to develop some future predicting method for proactive maintenance, to cut down human and financial cost.

The data what i have get to work with is a CAN-Bus data collected from the STILL Company's forklifts.

My main task is to prototype method and reason for specific data types for better results to improve the Industry 4.0 MANTIS project.

To fulfill this purpose, i have worked with two kinds of datasets.
%----------------------------------------------------------------------------
\section{My Solution}
%----------------------------------------------------------------------------
My task was with the "CAN fingerprint" data set to make it to a tidy format, explore what can one predict with no goal value providet, figure out requirement for future better prediction, visualise the connection between the given variables, and with the second phase data wave and small amount and low resolution goal value made a pilot predict algorithm to iterate the solution searching process. 

%(Electrical fault pred)

\section{Chapter description}
%----------------------------------------------------------------------------
		\paragraph{Technical background}
%----------------------------------------------------------------------------

Brief description: Technical background

%----------------------------------------------------------------------------
		\paragraph{Design}
%----------------------------------------------------------------------------

Brief description: Design

%----------------------------------------------------------------------------
		\paragraph{Implementation}
%----------------------------------------------------------------------------

Brief description: Implementation

%----------------------------------------------------------------------------
		\paragraph{Verification}
%----------------------------------------------------------------------------

Brief description: Verification

%----------------------------------------------------------------------------
		\paragraph{Conclusion}
%----------------------------------------------------------------------------

Brief description: Conclusion


%----------------------------------------------------------------------------
\chapter{Technical background}
%----------------------------------------------------------------------------
	\section{The MANTIS project}
%----------------------------------------------------------------------------
		\subsection{Overview of the MANTIS project}
%----------------------------------------------------------------------------
The MANTIS pilot project is a data-driven decision making system. The collected information comes from industrial sensors. Through a popular IoT communication protocol (MQTT) communicate with given company cloud service where data clarification and prediction happens. These processes are providing future insights to the company's employees for pre-maintenance leading to potential reduction of human, financial, and material resources.\cite{Mantis}

%----------------------------------------------------------------------------
		\subsection{Data sets from STILL}
%----------------------------------------------------------------------------
			\subsubsection{CAN fingerprint dataset}
%----------------------------------------------------------------------------
				\paragraph{Phase One}
				\noindent
As it was mentioned before, STILL forklifts \cite{RX20} use the CAN technology to collect data. In this project two kinds of them are utilized \footnote{Physical sensory and internal electric part logs.}. In one of the two called CAN fingerprint data set, the workers from the STILL had connected a special CAN bus data collector to one forklift and had done with them specific tasks. They have called them "fingerprints". The *.mat file names helped to understand the type of path and what the forklift has gone through:

\begin{itemize}[noitemsep]
    \item {800hTestDrive\_fast.mat}
    \item {800hTestDrive\_slow.mat}
    \item {FastTrack800h\_fast.mat}
 	\item {FastTrack800h\_slow.mat}
	\item {FastTrackEight\_wl\_fast.mat}
	\item {FastTrackEight\_wl\_slow.mat}
	\item {FastTrackEight\_wol\_fast.mat}
	\item {FastTrackEight\_wol\_slow.mat}
  	\item {FastTrackOval\_wl\_fast.mat}
	\item {FastTrackOval\_wl\_slow.mat}
	\item {FastTrackOval\_wol\_fast.mat}
	\item {FastTrackOval\_wol\_slow.mat}
	\item {Ramp\_wl\_fast.mat} 
	\item {Ramp\_wl\_slow.mat}
	\item {Ramp\_wol\_fast.mat}
	\item {Ramp\_wol\_slow.mat}
    \item {WorkCycle\_fast.mat}
    \item {WorkCycle\_slow.mat}
	\item {Shunt800h\_fast.mat}  	
  	\item {Shunt800h\_slow.mat}
 	\item {Shunt\_fast.mat} 	
 	\item {Shunt\_slow.mat}
\end{itemize}\footnote{tracktype\_(with load)/(without load)\_speed}

				\subparagraph\noindent
As it easily can be seen in the file names, the workers carried out various tests with the trucks more than one time with various speed and load on over various tracks.

However, in this first iteration, STILL did not provide any target value for tire abrasion. Thus from this first iteration I was able to make only exploratory data analysis to fuel further data collection. With enough target values in adequate resolution and quality predictive models can be trained based on that data.
				\subparagraph\noindent
In the second iteration on data provided on 10.2017 I received more information in a quite similar fashion, but from working forklifts and a top on that with some additional tire measurement. The CAN-bus sensor data contain approximately two months in time, but there were just a few measurements for tire wear from every third week. Unfortunately, this is just nearly sufficient to make predictions, but required for further iterations to develop and upgrade the model.

\#TODO enumerate column names in English.
%----------------------------------------------------------------------------
				\paragraph{Phase Two}
In the second iteration our team was given some additional data from STILL. It included 3 forklifts' CAN bus data with additional tire measurements and a little change in the columns names and values.
				\subparagraph\noindent
With this few tire measurement divided in time the interpolation and the RUL calculation become possible, despite of high risk of over-fitting the machine learning model or a statistical model, which could cause the lack of satisfying results.

\#TODO enumerate column names in English.
%----------------------------------------------------------------------------
			\subsubsection{Electrical failure prediction dataset}
%----------------------------------------------------------------------------
\paragraph\noindent
The STILL forklift's system has an electrical CAN bus warning collection subsystem. The warning comes from the part which can communicate with the bus. Engineers of STILL captured this data in a defined time frame and saved it for us in a txt file. 

\paragraph\noindent
One observation in the previously mentioned file involves a forklift unique ID, time stamp, occurred warning and other specific data what isn't in the scope of this thesis.

\paragraph\noindent
In the world of Electrical failure prediction dataset there are two additional file to work with, one is an SAP database export in a txt format and other one is and xls file all two have truck ID, exchanged part ID, and time stamp in one observation along with additional information what is not in the spectrum of this project.

\paragraph\noindent
For the sake of completeness the last two files have technical worker comments and not all observations have exchanged part ID. For that we can use some Natural Language Processing Methods but this is also not in the scope of this piece of writing.

\paragraph\noindent
From this three previously described file when it's represented in a normally workable way, the wanted remaining useful life can be calculated.

\#TODO enumerate column names in English.
%----------------------------------------------------------------------------
	\section{Data processing methods}
%----------------------------------------------------------------------------
		\subsection{CAN fingerprint}
				\paragraph\noindent
Assuming someone wants to make educated guess about the future, the neat arrangement of data is demanded, often called tidy format, as it is advised and required for preparation for most of data mining methods. And for that reasons I had chosen the R language to tackle this objective.
				\paragraph\noindent
In this case, the *.mat files involved a special treatment called box-shorting to achieve the clean and usable tidy state. During the box-shorting procedure mentioned before the researcher has to know the range of shorting key (in this data set's case the time stamp key's range) and with that one can make a large data table with a lot of "NA"-s, but it's needed to organize the multi-frequency sensory data.

Once the tidying is done, the \textit{exploratory} data analysis comes to make sense about the data mass and develop ways to achieve the best prediction possible, and short out unnecessary columns to make the future process faster.
%----------------------------------------------------------------------------
		\subsection{Electrical failure prediction}
For knowledge broadation and easy network-like data representation I had chosen Python language, and the Pycharm software \cite{PyCharm} along with NetworkX Python package\cite{NetworkX}.

This is an another coding language useful for data science related to making solutions along with an enormous web community.
%----------------------------------------------------------------------------
	\section{Development environment}
%----------------------------------------------------------------------------
		\subsection{Hardware}
%----------------------------------------------------------------------------
			\subsubsection{Personal computer}
%----------------------------------------------------------------------------
My PC is a Dell Latitude E6320 64 bit with an Intel(R) Core(TM) i5-2520M CPU @ 2.50GHz processor.
\cite{Latitude}
%----------------------------------------------------------------------------
			\subsubsection{Department Linux R and Python server}
%----------------------------------------------------------------------------
The Department's super computer called "batman" is Linux based R and Python server and it's capable of running R and Python script's on enormous data set's. It has two Intel(R) Xeon(R) CPU E5-2620 v2 @ 2.10GHz processors, each of them has 6 cores and is capable of running 12 Threads.
\cite{Batman}
%----------------------------------------------------------------------------
		\subsection{Software}
%----------------------------------------------------------------------------
			\subsubsection{Linux Ubuntu 17.04}
%----------------------------------------------------------------------------
The operation system in the title is an open source software, stable and convenient for an advanced user, and to go further, it's the easiest way to communicate with to Linux system via SSH.
\cite{Ubuntu}
%----------------------------------------------------------------------------				
			\subsubsection{RStudio}
%----------------------------------------------------------------------------
RStudio is the most common tool for developing R projects, it has good package controlling application.
\cite{RStudio}
%----------------------------------------------------------------------------	
			\subsubsection{Termius Android mobile phone application}
%----------------------------------------------------------------------------
The tidying and data preprocessing method is resource and time consuming. Thus I have optimized it for every-day life with an application via 4G or HSDPA internet access to maintain an SSH connection to the Linux R server. In addition, supervise and check the process's progress from time to time and hand the occasional errors and bugs.
\cite{Termius}
%----------------------------------------------------------------------------
			\subsubsection{Git and GitHub}
%----------------------------------------------------------------------------
As a consequence of version control and open source commitment I had chosen this common tool set \cite{Github} to save my work as it has been progressing. Including the work with the two data sets \footnote{\cite{GitHub_CAN_RUL,GitHub_FP_RUL}} and the documentation in LateX \cite{GitHub_Thesis_Text}.
%----------------------------------------------------------------------------
			\subsubsection{PyCharm}
%----------------------------------------------------------------------------
At working with the Electrical failure prediction data set, I had used python for various reasons, mainly because there is a package called \textit{NetworkX} \cite{NetworkX}. Moreover, for python developing for me the most convenient tool is PyCharm because it's high functionality.
\cite{PyCharm}	
%----------------------------------------------------------------------------
		\subsection{Management}
%----------------------------------------------------------------------------
according my demand of productivity, I use the GTD\cite{GTD} methodology concluded from 10 years of experience. It provides clarity, focus and flexible planning for me. Additionally, I implemented it in Wrike \cite{WRIKE} project management software available free for students. For more information visit the developer company's support website: \cite{WRIKE_for_students}
%----------------------------------------------------------------------------
\chapter{Design}
%----------------------------------------------------------------------------
\section{CAN fingerprint dataset processing method}
%----------------------------------------------------------------------------
\subsection{Requirement specification}
%----------------------------------------------------------------------------
	\noindent
The final aim was to create data processing methods and calculated attributes from measured values for future prediction applications. Moreover, to come up with a sustainable prediction model, what is capable for prediction even when the amount of th provided target values are low.
%----------------------------------------------------------------------------
	\subsubsection{Initial state}
The initial state was a raw CAN bus physical sensory data in a *.mat format with different measurement frequencies by sensor types.
%----------------------------------------------------------------------------
	\subsubsection{Final state}
To achieve the final state, datasets require to be transformed into a CSV (Comma Separated Value) or RDS (R Data Structure) format for more convenient handling and calculating attributes.

The attributes are aggregations and summarizations of connection between key values originating from experts' knowledge, exploratory data analysis, and several unsupervised learning techniques. Along with several data transforming functions, to achieve the demanded Reaming Useful Life prediction.
\clearpage
%----------------------------------------------------------------------------
\subsubsection{Abstract Workflow}
To accomplish the final state from the starting state, the process was partitioned from on start-to-end into smaller steps. 

\begin{enumerate}
	\item {Cleaning:} to solve different measurement frequency problems by making a one-row/one-observation data frame.
	\item {Exploring:} to investigate the data with exploratory data analysis tools and unsupervised learning supported by the STILL workers', engineers' and my acquaintances' experts' knowledge.
 	\item {Specifying Attributes:} to introduce sustainable attribute calculating algorithms.
 	\item {Calculating RUL (Remaining Useful Life):} to develop a RUL calculation algorithm for the MANTIS project. This is based on received target values\footnote{few tire measurements}, and attributes calculated beforehand.
\end{enumerate}

\begin{figure}[!ht]
\centering
\includegraphics[width=150mm, keepaspectratio]{figures/abstract_workflow.png}
\caption{Abstract workflow from raw data, to final objective.} 
\end{figure}

%----------------------------------------------------------------------------
\subsection{Expert knowledge and facts}
%----------------------------------------------------------------------------
\paragraph\noindent
When some kind of industrial dataset is examined and processed to make educated guess about the future, principles and insights from experts' knowledge are actually helpful to interpret and understand the data in a higher sense\footnote{In reality the vast majority of the insight came from exploratory data analysis and unsupervised learning}. 

For example: There is four main forklift driving style.

In summary, the main indicators of tire abrasion between two tire measurements are:
\begin{itemize}
	\item{traveled distance,} 
	\item{elapsed time between,}
	\item{elapsed time in high torque low speed state,}
	\item{changing x direction in elapsed time,}
	\item{changing y direction in elapsed time.}
\end{itemize}

These aggregated values can be calculated from the raw sensor data, and with enough tire measurement points the RUL can be calculated with a help of a previously built statistical model.

%----------------------------------------------------------------------------
\subsection{Data processing workflow design}
%----------------------------------------------------------------------------
As it can be seen from the starting and final state, at first the *.mat files have to be transformed into usable data frame. Meanwhile, the not useful columns have to be dropped to save precious processor time and data storage space, especially by handling the longer files. After that the attributes should to be calculated, then the remaining useful life has to be estimated.
Various planning principles can be utilized during this design phase \cite{CSDISTILLED}, \cite{DATACAMP}, \cite{LeanThinking}.
%----------------------------------------------------------------------------
	\subsubsection{Cleaning}
	As it has been described in the previous section, the *.mat files have to be transformed to a data frame with only the useful columns remaining to further calculations.

	To establish a sustainable process, the iteration has to go through more than one ideas. For the sake of clarity, all ideas would be enumerated.
		\paragraph{Idea One:}
			The first data shape transforming idea was to box-short the data. Item by item:
\begin{enumerate}
	\item{to examine the longest key-value pair vector maximal timestamp,} 
	\item{to make a data frame with timestamp keys from zero to the maximum timestamp key got from the file in step of 10 milliseconds,}
	\item{to search every value's place with the help of timestamp key,}
	\item{to interpolate for missing values,}
	\item{to do it for all columns in the file,}
	\item{to do it for all files.}
\end{enumerate}
			\begin{figure}[!ht]
			\centering
			\includegraphics[width=150mm, keepaspectratio]{figures/cleaning_idea_one.png}
			\caption{First idea (Idea One) to clean the raw data.} 
			\end{figure}
		\subparagraph\noindent
This is time consuming and slow even on the server, because it is not the optimal sorting algorithm for this kind of data representation \cite{CSDISTILLED}.
%----------------------------------------------------------------------------
		\paragraph{Idea Two:}
To get rid of the processing-speed problem, the second idea was to aggregate the values' mean by second. This solution has to be more faster than the previous one, but if there time skip in the data\footnote{there is a lot of missing rows, sometimes effecting minutes of operational data}, this idea would not be sufficient, either.
			\begin{figure}[!ht]
			\centering
			\includegraphics[width=150mm, keepaspectratio]{figures/cleaning_idea_two.png}
			\caption{Second idea (Idea Two) to clean the raw data.} 
			\end{figure}
%----------------------------------------------------------------------------
		\paragraph{Idea Synthesis:}
To get the advantage of the two methods, it would be useful to merge the previous ideas.
\begin{enumerate}
	\item{examine the longest key-value pair vector maximal timestamp,} 
	\item{make a data frame with timestamp keys from zero to the maximal timestamp key got from the file in step of a second,}
	\item{aggregate the measurement average second by second,}
	\item{search every values' place with the help of timestamp key,}
	\item{interpolate to eliminate values,}
	\item{do it for all columns in the file,}
	\item{do it for all files.}
\end{enumerate} 
			\begin{figure}[!ht]
			\centering
			\includegraphics[width=150mm, keepaspectratio]{figures/cleaning_idea_synthesis.png}
			\caption{To optimize the cleaning workflow, the to processes were merged together.} 
			\end{figure}
\subparagraph\noindent
With this solution design the high quality data frame, what can be used later, is possible, and with this idea synthesis the problems with the previous ones are worked around.
%----------------------------------------------------------------------------
	\subsubsection{Exploring}
	The data exploration and data examination is a required step for correct interpretation of the data and confirmation of facts coming from experts' knowledge. Without this "null hypothesis check" step, a lot of human labor and energy could be discarded.
%----------------------------------------------------------------------------
		\paragraph{Exploratory data analysis:}
			With these steps\footnote{data summaries and visualization} one can make himself ensure that dataset has connection with the real world, the values are close to reality, and further steps are worth a shot.
%----------------------------------------------------------------------------
			\subparagraph{Data summaries:}
			There are a lot of data summary methods\footnote{mean, median, SQRT, min, max and quarters}. With these aggregations ran on the dataset, the first impression is given. It is advisable to use these resource sparing methods.
%----------------------------------------------------------------------------			
			\subparagraph{Visualizations:}
			Human mind can effortlessly interpret large amount of data with correct visual representation. Furthermore, can check the connections and facts derived from experts' knowledge\footnote{Not to mention, amend that knowledge.}. 
%----------------------------------------------------------------------------		
		\paragraph{Unsupervised learning:}
		The "unsupervised learning" \cite{UnsuplearnBook} is a machine learning method used especially to unlabeled data. It is useful to find connections and correlation between the values in a higher perspective and confirm or question experts' knowledge.
%----------------------------------------------------------------------------
			\subparagraph{k-means clustering:}
			The k-means clustering \cite{k-means}, \cite{kmeans2} helps to estimate groups between observations with norming all values to one and calculating multi-dimensional distance between them.
			\subparagraph{Hierarchical clustering:}
			Hierarchical clustering \cite{h-clust}, \cite{h-clust2}, \cite{Clustering} is a k-means with a twist, because it clusters groups hierarchically, step by step with various distance calculation methods and top-down or bottom-up approach.
			\subparagraph{Principal component analysis:}
			The PCA \cite{PCA} algorithm makes an orthogonal, multi dimensional space from the values and places there the observations. This algorithm can tell the most significant values from the dataset.
%----------------------------------------------------------------------------
	\subsubsection{Specifying Attributes:}
	From experts' knowledge\footnote{and exploratory data analysis} the main components and indicators of the tire abrasion can be specified.

	These attributes are:
	\begin{itemize}
		\item{elapsed time,} 
		\item{traveled distance,} 
		\item{count in various speed and torque state,}
		\item{changing x direction,}
		\item{changing y direction,}
		\item{"is there weight" counter (max 3600 in a hour),}
		\item{steering wheel degree change derived by time aggregated by average\footnote{the torque and speed derived by time could be useful too}.}
	\end{itemize}
	All of them will be aggregated hour by hour\footnote{if the tire measurement interval goes down the aggregation window could and should too} from tire change or measurement.
%----------------------------------------------------------------------------
	\subsubsection{Calculating RUL}
	After the specified attributes are calculated, and aggregated and enough\footnote{more than one} tire measurement is available with accurate timestamp, the RUL can be calculated in the following way:

	\begin{enumerate}
		\item{to join tire measurement and hourly aggregated attributes by timestamp,}
		\item{to interpolate from tire measurements to hourly aggregations,}
		\item{to compute tire diameter change on all given measurement points,}
		\item{to calculate average diameter change by hourly aggregations,} 
		\item{to produce attributes hourly for a new measurement,}
		\item{to calculate tire diameter change for all attributes based on proportionality,}
		\item{to compute mean tire diameter change by hour,}
		\item{to summarize tire diameter change from last known diameter,}
		\item{if the summary is greater than a given constant, the tire should be changed,}
		\item{to compute reaming useful life on the last hours abrasion rate.}
	\end{enumerate}

	\paragraph\noindent
	The whole CAN fingerprint calculation workflow is in the Appendix \ref{appendix:CANFPRULCalc}.

	Other not implemented solutions:
	\begin{itemize}
		\item{calculate RUL with machine learning:} There were not enough data for this solution, the model will be over-fitted based on DataCamp case studies \cite{DataCampCaseStudies} 
		\item{calculate RUL with deep learning:} Similar to above.	
	\end{itemize}
	
%----------------------------------------------------------------------------
%----------------------------------------------------------------------------
\section{Electrical failure prediction dataset processing method}
%----------------------------------------------------------------------------
	\subsection{Requirement specification}
%----------------------------------------------------------------------------
This prediction model encompasses given CAN bus warning collection from STILL forklifts' electrical parts and SAP technical database containing technical workers' comments on each repair sessions. Moreover, it encloses occasionally exchanged parts' ID and name, and later described warning sequences from a random machine. The model has to estimate from these data the remaining useful life to a next repair session.

In the scope of this project iteration the exchange part IDs are the superior priority. The technical workers comments should be categorized with natural language processing method \cite{nlp}.
%----------------------------------------------------------------------------
	\subsection{Data processing workflow design}
%----------------------------------------------------------------------------
As it previously has been mentioned, the BAL Network science book \cite{BALNWSCBOOK} is an effective indicator of later specified solution. However, a lot of optimizing ideas come from the Computer Science Distilled book \cite{CSDISTILLED} and the usage of the Python language and the NetworkX package \cite{NetworkX} is the influence of the DataCamp e-learning platform \cite{DATACAMP}, and thinking in flows, process and value streams is caused by books such as the Lean Thinking book \cite{LeanThinking} and the Phoenix Project book \cite{PhoenixProject}.
%----------------------------------------------------------------------------
		\subsubsection{Conditions of the raw data}
		The dataset given for this subproject is consisting three files:
		\begin{itemize}
			\item{xtra\_All\_errors\_with\_metainfo.txt} size of 12.9 GB contains the warning logs of 789 forklifts, with timestamp, machine ID, the IDs of warnings, additional information which is out of the scope of this project\footnote{but in a further iteration could be necessary}, and a lot of NA columns.
			\item{ServiceReportSAPExtrakt\_FLMxFahrzeuge.txt} size of 224.7 MB contains an export from an SAP repair database with columns named for example: as "when was the machine repair session stared","when was the machine repair session ended", the machine's ID, the exchanged part ID number, some additional information\footnote{not all rows contain exchanged part ID. In these cases the technical worker's comments can be useful, but natural language processing is not the scope of this thesis.} and lot of NA columns.
			\item{Alle-SBs-ProActive\_droprows.xlsx} size of 4.5 MB nearly exactly the same columns like the one above, but form different forklifts and a little different column names.
		\end{itemize}
%----------------------------------------------------------------------------
		\subsubsection{Cleaning in two iterations}
		\begin{itemize}
			\item{Eliminating NA-s columns, and duplications:} This step is required to reduce the further resource cost of operations, and to have a filebase for further method development.
			\item{Discarding not useful columns:} For the first iteration graph building, the timestamp, machine ID, and ID of the warning is required from the first file mentioned before. From the second, and third file the repair sessions' starting and ending time the machine ID, and the exchange part ID is essential. When there is no exchange part ID just technical worker comment, for fast method's development, it could be handy to give a temporary distinct ID. The temporary ones have to differ from real exchange part IDs.\footnote{this can be a starting point of subcategorizing, with the help of NLP} 
		\end{itemize}
%----------------------------------------------------------------------------
		\subsubsection{Creating nodes' list and edges' list}
		From previously generated files the nodes and edges, the two vital components of the graph, can be generated. The nodes' list should contain all the distinct warning and exchange part ID-s, the edges' list should contain\footnote{One edge in a row.} the timestamps, the from and to nodes' ID , and the vehicles' ID.
%----------------------------------------------------------------------------
		\paragraph{Making the nodes' list:}
		The previously generated files contain a lot of duplicated warning and exchange part ID-s along with a multitudinous not needed columns.
		In this case the planned steps should be:
		\begin{enumerate}
			\item{to read in warning ID and exchange part ID columns from previously mentioned files,} 
			\item{to concatenate the two SAP originated lists,}
			\item{to drop duplications,}
			\item{to concatenate the lists,}
			\item{to save it.}
		 \end{enumerate}
%----------------------------------------------------------------------------
		\paragraph{Making the edges' list, machine by machine:}
		For easier handling, better performance, and disk space sparing reasons, it would be wise to generate the paths of all machines.
		The steps should be:
		\begin{enumerate}
			\item{reading in the largest file chunk by chunk,} 
			\item{grouping the information by machine ID,}
			\item{saving it to a corresponding machine's file,}
			\item{doing the previous 2 steps to the smaller files after reading it in as a whole,}
			\item{reading in the distinct machine paths one by one,}
			\item{ordering them by time,}
			\item{saving them in separated files.}
		 \end{enumerate}
		With these steps we get the distinct nodes and a lot of edges in different paths.

		The electrical failure prediction preprocessing workflow is in the Appendix \ref{appendix:FaultPredPreProc}
%----------------------------------------------------------------------------
		\subsubsection{Exploring:}
			Unlike the previous dataset, the cleaning steps are not as significant as they could be, because the data was given in a more efficient way. After the previously described NA and duplicated column dropping, the data exploring can begin. However, the end of data exploring is after the making of nodes and edges' list, because it is easy to check visually a correct graph representation\footnote{for example: the path is continuous or dashed}.
%----------------------------------------------------------------------------
			\subparagraph{Data summaries:}
			There are a lot of data summary methods (mean, median, SQRT, min, max and quarters). With these aggregations ran on the dataset, it is get a first impression. 
%----------------------------------------------------------------------------
			\subparagraph{Visualizations:}
			The human mind can easily interpret large amount of data with correct visual representation, and to go further, check the connections and facts derived from experts' knowledge\footnote{Not to mention, amend that knowledge.}.
%----------------------------------------------------------------------------
		\subsubsection{Calculating RUL}
%----------------------------------------------------------------------------
			\paragraph{Building the graph:}
			When the previous ordering, exploring, and fact checking are completed, the graph building can begin from generated files. Thus the base is the nodes' list, all nodes should be represented in the graph, when it comes to edges, the situation is different. For computational resource reasons not all of the paths should be included for sure. But the more the paths, the better is the prediction.
%----------------------------------------------------------------------------
\clearpage\paragraph{Computing steps:}
			If the graph has been built, the remaining useful life calculation steps should be the following:
			\begin{enumerate}
				\item{getting a warning sequence, what is not represented in the graph, from the same machine, in a chronological order,}
				\item{listing the machine' IDs between the oldest and the second oldest nodes given in the sequence}\footnote{All the nodes in the new sequence should be in the graph, as well.} 
				\item{iterating through the whole sequence and eliminating the machines which are absent from the current edges,}
				\item{eliminating the oldest node in the sequence and redo the previous search, when there is no machine shorted with the sequence,}
				\item{listing the shortest similar machine paths to the exchange part ID nodes, from the last and youngest node in the sequence.}
				\item{from the last node the possibility calculating to all exchange part ID nodes, based on the similar shortest machine path count to the specific exchange part ID nodes dividing by sum of to all the exchange part ID nodes,}
				\item{besides of possibility, computing the distance in time  to the specific exchange part ID node, to give more insights about the remaining useful life, based on the similar shortest machine's distance in time,}
				\item{the result should look with columns like: exchange part ID node, minimal, mean, median and maximal time to that node, and the computed possibility arranged by possibility and time,} 
		 		\item{the RUL is the highest possibility with the shortest remaining time.}
		 	\end{enumerate}
		
		\paragraph\noindent	
		The calculation of the whole table could be profitable because the technical workers can solve more problem at one repair sessions which could cause longer timespan on duty for the machines.

		The whole electrical failure prediction preprocessing workflow is in the Appendix \ref{appendix:FaultPredRULCalc}
%----------------------------------------------------------------------------
\chapter{Implementation}
%----------------------------------------------------------------------------
\section{STEP 1}
%----------------------------------------------------------------------------

FROM WORKFLOW

TEST
\cite{DipPortal}

%----------------------------------------------------------------------------
\section{STEP 2}
%----------------------------------------------------------------------------

FROM WORKFLOW

%----------------------------------------------------------------------------
\section{STEP 3}
%----------------------------------------------------------------------------

FROM WORKFLOW

%----------------------------------------------------------------------------
\section{STEP 4}
%----------------------------------------------------------------------------

FROM WORKFLOW

%----------------------------------------------------------------------------
\section{STEP 5}
%----------------------------------------------------------------------------

FROM WORKFLOW
%----------------------------------------------------------------------------
\chapter{Verification}
%----------------------------------------------------------------------------
\section{CAN fingerprint dataset processing method}
%----------------------------------------------------------------------------
\subsection{Functional testing}
%----------------------------------------------------------------------------
All written functions and workflows were manually unit tested before the unit assembly to a grater function or workflow. Thus, debugging were easier and following functional testing was simpler because of the trustworthy sub functions and sub workflows.

After a wanted workflow for example the cleaning was assembled, the following steps were integration tests, to ensure the solution's quality and whole planning and implementation process' validity.

Due to their large volumes in this data set all of these enormous files were tested and ran on the TMIT department's Linux server \cite{Batman}.
Afterwards, the results were copied back to the personal computer for further investigation.
%----------------------------------------------------------------------------
\subsection{Performance testing}
%----------------------------------------------------------------------------
The performance was investigated on various workflows, as they already had been presented, on two computers on a personal computer \cite{Latitude}, and on a Linux server \cite{Batman}.

The results were:
\begin{table}[H]
\centering
\begin{tabular}{ |c|c|c|  }
\hline
\multicolumn{3}{|c|}{Performance test results on small files} \\
\hline
Test workflow& PC [min] & Linux Server [min]\\
\hline
Cleaning Idea 1 one file& 20 & 12 \\
Cleaning Idea 1 all files& NA & 120 \\
Cleaning Idea 2 one file& 2 & 0.5 \\
Cleaning Idea 2 all files& 10 & 6 \\
Cleaning Synthesis 2 one file& 5 & 3 \\
Cleaning Synthesis 2 all files& 14 & 8 \\
Specifying attributes & 2 & - \\
\hline
\end{tabular}
\caption{Performance test results on CAN fingerprint's small files}
\label{table:1}
\end{table}

\begin{table}[H]
\centering
\begin{tabular}{ |c|c|c|  }
\hline
\multicolumn{3}{|c|}{Performance test results on large files\footnote{The missing data is due to the lack of quality data}} \\
\hline
Test workflow& PC [min] & Linux Server [min]\\
\hline
Cleaning Idea 2 one file& 10 & 5 \\
Cleaning Idea 2 all files& NA & 20 \\
Cleaning Synthesis 2 one file& NA & 5.5 \\
Cleaning Synthesis 2 all files& NA & 21 \\
Specifying attributes & NA & 3 \\
CAL calculating RUL & NA & 4 \\
Whole RUL calculating workflow & NA & NA \\
\hline
\end{tabular}
\caption{Performance test results on CAN fingerprint's large files}
\label{table:2}
\end{table}
%----------------------------------------------------------------------------
\subsection{Possible functional and performance optimizations}
The results in previous section were indicating the worth of future optimization. There are a lot of opportunities when the whole process are under investigation.

The whole process means from the data collection on the forklifts CAN bus, and the tire measurement, through the entire computation processes to the final RUL calculation.

The possible optimization opportunities are:
\begin{itemize}
	\item{Improving the data collection from CAN bus, and eliminating measurement gaps with that improvement.} Thus the process could skip the interpolation step. 
	\item{Changing file format, which the data is saved from the CAN bus.} If the file format could change from *.mat to CSV\footnote{Comma Separated Value} or RDS\footnote{R Data Structure}, the process can skip a step too.
	\item{Widening the aggregation window.} The optimal aggregation window is under search right now. However, when the correct window was found, the process could speed up because of fewer observations to handle.
	\item{Estimating the correct lower size of speed and torque 2D binning.} The testing size were $9*9$ but if future testing shows lower is better, it could be an acceleration too, because a lower attribute count and it means lower processing time.
	\item{Weighting the attributes.} Inducted RULs by the attributes are summarized by a casual mean function, but with future testing or some machine learning algorithm the RUL calculating attributes could be weighted for additional precision.
\end{itemize} 
%----------------------------------------------------------------------------%----------------------------------------------------------------------------
%----------------------------------------------------------------------------%----------------------------------------------------------------------------
\section{Electrical failure prediction dataset processing method}
%----------------------------------------------------------------------------
\subsection{Functional testing}
%----------------------------------------------------------------------------
Although there is an opportunity to write automated unit and integration tests in Python, all written functions and workflows were manually unit tested, before the unit assembly to a grater function or workflow. Therefore, the debugging were easier and following functional testing was simpler because of trustworthy sub functions and sub workflows.

After a wanted workflow (e.g. the cleaning) was assembled, following steps were the integration tests, to ensure the solution quality and whole planning process validity of the new approach.

Thus, the large file in this dataset, the enormous file was tested and ran on the department Linux server \cite{Batman}.
Moreover, the results were copied back to the personal computer for further investigation.
%----------------------------------------------------------------------------
\subsection{Performance testing}
%----------------------------------------------------------------------------
The performance was investigated on various workflows, as they already had been mentioned, on two computer: on a personal computer \cite{Latitude}, and on a Linux server \cite{Batman}.

The tested workflows were the following:
\begin{enumerate}
	\item{Dropping NA-s columns from the 12.9 GB electric\_errors.txt,}
	\item{Dropping NA-s columns from the 224.7 MB sap\_fails.txt,}
	\item{Selecting columns for the graph building from the electric\_errors.csv,}
	\item{Selecting columns for the graph building from the sap\_fails.csv,}
	\item{Fabricating nodes list,}
	\item{Fabricating edges list,}
	\item{Building the graph A\footnote{With 10 machines' path.},}
	\item{Building the graph B\footnote{With 100 machines' path.},}
	\item{Calculating RUL A,}
	\item{Calculating RUL B,}
	\item{Whole RUL calculating workflow A,}
	\item{Whole RUL calculating workflow B.}
\end{enumerate} 

\begin{table}[H]
\centering
\begin{tabular}{ |c|c|c|  }
\hline
\multicolumn{3}{|c|}{Performance test results\footnote{The missing data is due to the lack of correct data}} \\
\hline
Test workflow& PC [min] & Linux Server [min]\\
\hline
1.& NA & 14 \\
2.& 1 & NA \\
3.& 4 & 1 \\
4.& 1 & - \\
5.& NA & 3 \\
6.& NA & 4 \\
7.& 9 & 3 \\
8.& NA & 3 \\
9.& 20 & 2 \\
10.& NA & NA \\
11.& NA & NA \\
12.& NA & NA \\
\hline
\end{tabular}
\caption{Performance test results on failure prediction files}
\label{table:3}
\end{table}
%----------------------------------------------------------------------------
\subsection{Possible functional and performance optimizations}
Results in previous section were indicating the worth of future optimization. There are a lot of opportunities when the whole process are under investigation.

The possible optimization opportunities are:
\begin{itemize}
	\item{Allocating more resources to RUL computation process.} Thus, RUL calculation is computation heavy. However, some cloud computing solution can speed up the process.
	\item{Including technical workers' comment.} As they already had been mentioned, the technical workers' comment with NLP tokenisation, and categorization could bring more and preciser fail states which could improve the predictions' precision.
	\item{Drop paths under X path weight.} The calculation of lower paths' possibility could be probably worthless with further testing. Corresponding a dropping level could be estimated, with that in mind the wasteful paths could be dropped.
	\item{Saving by vehicle ID when making node list.} In the current solution grouping was done on the whole file, but if the edges pre-data chunk is saved to distinct files by vehicle ID, the processing computer could arrange data file by file. This could accelerate the grouping process.
\end{itemize} 

%----------------------------------------------------------------------------
\chapter{Conclusions}
%----------------------------------------------------------------------------
\section{CAN fingerprint data set}
%----------------------------------------------------------------------------
\subsection{Phase 1: Small files}
%----------------------------------------------------------------------------
In conclusion, the predefined approach and the designed process were buoyant. The experts' knowledge were useful, but like a direction not like a path, every facts has to rechecked and validated.

The target value is needed for the RUL calculation without that the whole process is like a thought experiment. 
%----------------------------------------------------------------------------
\subsection{Phase 2: Large files with tire measurement}
%----------------------------------------------------------------------------
In this second iteration, the given data was nearly enough to finish the piloting and testing, more target value, more frequent tire measurement required, for precise prediction, without that the whole process is just a resource heavy estimation.

With more frequent tire measurements, these process could be product ready and could help the pre-maintance pursuits.

\section{Electrical fail prediction dataset}
%----------------------------------------------------------------------------

In conclusion, the graph-like data representation approach is resource consuming but worth the future development, with enough optimization, this process also could be product ready and help the pre-maintance pursuits of the industry, not just the data provider company, but all situations were there is rather states than values and observations.

%----------------------------------------------------------------------------
\section{Summary on the gained experience and knowledge}
%----------------------------------------------------------------------------
The experiences gained from this journey are:
\begin{itemize}
	\item{Don't take anything for granted.} All of the heard facts, solutions, prediction from anyone have to be fact checked and data proven.
	\item{Always dig down to first principles.} When a process is have to be optimized, it's the best solution to dig down to first principles, and from that solid rock bottom build up the whole theory and then the workflow.
	\item{There is always room for improvement.} As it seems from the previous chapter optimization section the is no perfect solution, just globally optimal. 
\end{itemize}
%----------------------------------------------------------------------------
\section{Advices to future similar projects}
%----------------------------------------------------------------------------
After this project the there are some advices for future pre-maintanance process builders, data-scientist, and graph-like data representers:
\begin{itemize}
	\item{Have an eye on the management} The there is a channel, there is noise. Moreover, if the desired outcome of a project or task are not clear, the consequence couldn't be qualitative. Keep in mind when a favor in both direction is asked or a task is set.
	\item{Build from first principles, and question everything.} With that method the most of the future firefighting could be eliminated.
\end{itemize}

%%----------------------------------------------------------------------------
\chapter{\LaTeX-eszk�z�k}\label{sect:LatexTools}
%----------------------------------------------------------------------------
\section{A szerkeszt�shez haszn�latos, Windows alap� eszk�z�k}
%----------------------------------------------------------------------------
Ez a sablon Windows oper�ci�s rendszer alatt k�sz�lt TeXnicCenter 1 Beta 7.01 szerkeszt�vel. A TeXnicCenter egy \LaTeX-szerkeszt�program sz�mtalan hasznos -- �s r�ad�sul j�l m�k�d� -- szolg�ltat�ssal (\figref{TexnicCenter} �bra). A szoftver ingyenesen let�lthet� a\\\url{http://www.texniccenter.org/} c�mr�l.

\begin{figure}[!ht]
\centering
\includegraphics[width=150mm, keepaspectratio]{figures/TeXnicCenter.png}
\caption{A TeXnicCenter Windows alap� \LaTeX-szerkeszt�.} 
\label{fig:TexnicCenter}
\end{figure}

Egy m�sik haszn�lhat� Windows alap� szerkeszt�program a LEd (LaTeX Editor,\\\url{http://www.latexeditor.org/}), a TeXnicCenter azonban stabilabb, gyorsabb, �s jobban haszn�lhat�.

%----------------------------------------------------------------------------
\section{A dokumentum leford�t�sa Windows alatt}
%----------------------------------------------------------------------------
A TeXnicCenter �s a LEd kiz�r�lag szerkeszt�program (b�r az ut�bbiban DVI-n�zeget� is van), �gy a dokumentum ford�t�s�hoz sz�ks�ges eszk�z�ket nem tartalmazza. Windows alatt alapvet�en k�t lehet�s�g k�z�l �rdemes v�lasztani: MiKTeX (\url{http://miktex.org/}) �s TeXLive (\url{http://www.tug.org/texlive/}) programcsomag. Az ut�bbi m�k�dik Mac OS X, GNU/Linux alatt �s Unix-sz�rmaz�kokon is. A MiKTeX egy alapcsomag telep�t�se ut�n mindig let�lti a haszn�lt funkci�khoz sz�ks�ges, de lok�lisan hi�nyz� \TeX-csomagokat, m�g a TeXLive DVD ISO verz�ban f�rhet� hozz�. Ez a dokumentum TeXLive 2008 programcsomag seg�ts�g�vel fordult, amelynek DVD ISO verzi�ja a megadott oldalr�l let�lthet�. A sablon leford�t�s�hoz a disztrib�ci�ban szerepl� \verb+magyar.ldf+ f�jlt a \verb+http://www.math.bme.hu/latex/+ v�ltozatra kell cser�lni, vagy az ut�bbi v�ltozatot be kell m�solni a projekt-k�nyvt�rba (ahogy ezt meg is tett�k a sablonban) k�l�nben anom�li�k tapasztalhat�k a dokumentumban (pl. az �bra- �s t�bl�zat-al��r�sok form�tuma nem a be�ll�tott lesz, vagy bizonyos oldalakon megjelenik alap�telmez�sben egy fejl�c). A TeXLive 2008-at m�g nem kell k�l�n telep�teni a g�pre, elegend� DVD-r�l (vagy az ISO f�jlb�l k�zvetlen�l, pl. DaemonTools-szal) haszn�lni. 

A \TeX-eszk�z�ket tartalmaz� programcsomag bin�risainak el�r�si �tj�t minden esetben be kell �ll�tani a szerkeszt�programban, p�ld�ul TeXnicCenter eset�n legegyszer�bben a \verb+Build / Define output profiles...+ men�ponttal el�h�vott dial�gusablakban a \verb+Wizard...+ gombra kattintva tehetj�k ezt meg.

A PDF-\LaTeX~haszn�lata eset�n a gener�lt dokumentum k�zvetlen�l PDF-form�tumban �ll rendelkez�sre. Amennyiben a PDF-f�jl egy PDF-n�z�ben (pl. Adobe Acrobat Reader vagy Foxit PDF Reader) meg van nyitva, akkor a f�jlle�r�t a PDF-n�z� program tipikusan lefoglalja. Ilyen esetben a dokumentum �jraford�t�sa hiba�zenettel kil�p. Ha bez�rjuk �s �jra megnyitjuk a PDF dokumentumot, akkor pedig a PDF-n�z�k t�bbs�ge az els� oldalon nyitja meg a dokumentumot, nem a legut�bb olvasott oldalon. Ezzel szemben p�ld�ul az egyszer� �s ingyenes \textcolor{blue}{Sumatra PDF} nev� program k�pes arra, hogy a megnyitott dokumentum megv�ltoz�s�t detekt�lja, �s friss�tse a n�zetet az aktu�lis oldal megtart�s�val.

%----------------------------------------------------------------------------
\section{Eszk�z�k Linuxhoz}
%----------------------------------------------------------------------------
Linux oper�ci�s rendszer alatt is rengeteg szerkeszt�program van, pl. a KDE alap� Kile j�l haszn�lhat�. Ez ingyenesen let�lthet�, vagy �ppens�ggel az adott Linux-disztrib�ci� eleve tartalmazza, ahogyan a dokumentum ford�t�s�hoz sz�ks�ges csomagokat is. Az Ubuntu Linux disztrib�ci�k alatt p�ld�ul legt�bbsz�r a \verb+texlive-base+ csomag telep�t�s�vel haszn�lhat�k a \LaTeX-eszk�z�k.

%%----------------------------------------------------------------------------
\chapter{A dolgozat formai kivitele}
%----------------------------------------------------------------------------
Az itt tal�lhat� inform�ci�k egy r�sze a BME VIK Hallgat�i K�pviselet �ltal k�sz�tett ,,Utols� f�l�v a villanykaron'' c. munk�b�l lett kis v�ltoztat�sokkal �temelve. Az eredeti dokumentum az al�bbi linken �rhet� el: \url{http://vik-hk.bme.hu/diplomafelev-howto-2009}.

%----------------------------------------------------------------------------
\section{A dolgozat kim�rete}
%----------------------------------------------------------------------------
A minim�lis 50, az optim�lis kim�ret 60-70 oldal (f�ggel�kkel egy�tt). A b�r�l�k �s a z�r�vizsga bizotts�g sem szereti kifejezetten a t�l hossz� dolgozatokat, �gy a brutt� 90 oldalt m�r nem �rdemes t�lsz�rnyalni. Egy�bk�nt f�ggetlen�l a dolgozat kim�ret�t�l, ha a dolgozat nem �rdekfesz�t�, akkor az olvas� m�r az elej�n a v�g�t fogja v�rni. �rdemes z�rt, �nmag�ban is �rthet� m�vet alkotni.

ex
Some of the \textbf{greatest} 
discoveries in \underline{science} 
were made by \textbf{\textit{accident}}.

%----------------------------------------------------------------------------
\section{A dolgozat nyelve}
%----------------------------------------------------------------------------
Mivel Magyarorsz�gon a hivatalos nyelv a magyar, ez�rt alap�rtelmez�sben magyarul kell meg�rni a dolgozatot. Aki k�lf�ldi posztgradu�lis k�pz�sben akar r�szt venni, nemzetk�zi szint� tudom�nyos kutat�st szeretne v�gezni, vagy multinacion�lis c�gn�l akar elhelyezkedni, annak c�lszer� angolul meg�rnia diplomadolgozat�t. Miel�tt a hallgat� az angol nyelv� verzi� mellett d�nt, er�sen aj�nlott m�rlegelni, hogy ez mennyi t�bbletmunk�t fog a hallgat�nak jelenteni fogalmaz�s �s nyelvhelyess�g ter�n, valamint - nem utols� sorban - hogy ez mennyi t�bbletmunk�t fog jelenteni a konzulens illetve b�r�l� sz�m�ra. Egy nehezen olvashat�, netal�n �rthetetlen sz�veg teher minden j�t�kos sz�m�ra.

%----------------------------------------------------------------------------
\section{A dokumentum nyomdatechnikai kivitele}
%----------------------------------------------------------------------------
A dolgozatot A4-es feh�r lapra nyomtatva, 2,5 centim�teres marg�val (+1~cm k�t�sbeni), 11-12 pontos bet�m�rettel, talpas bet�t�pussal �s m�sfeles sork�zzel c�lszer� elk�sz�teni.



%%----------------------------------------------------------------------------
\chapter{A \LaTeX-sablon haszn�lata}
%----------------------------------------------------------------------------
Ebben a fejezetben r�viden, implicit m�don bemutatjuk a sablon haszn�lat�nak m�dj�t, ami azt jelenti, hogy sablon haszn�lata ennek a dokumentumnak a forr�sk�dj�t tanulm�nyozva v�lik teljesen vil�goss�. Amennyiben a szoftver-keretrendszer telep�tve van, a sablon alkalmaz�sa �s a dolgozat szerkeszt�se \LaTeX-ben a sablon seg�ts�g�vel tapasztalataink szerint j�val hat�konyabb, mint egy WYSWYG (\emph{What You See is What You Get}) t�pus� sz�vegszerkeszt� eset�n (pl. Microsoft Word, OpenOffice).

%----------------------------------------------------------------------------
\section{C�mk�k �s hivatkoz�sok}
%----------------------------------------------------------------------------
A \LaTeX~dokumentumban c�mk�ket (\verb+\label+) rendelhet�nk �br�khoz, t�bl�zatokhoz, fejezetekhez, list�khoz, k�pletekhez stb. Ezekre a dokumentum b�rmely r�sz�ben hivatkozhatunk, a hivatkoz�sok automatikusan felold�sra ker�lnek.

A sablonban makr�kat defini�ltunk a hivatkoz�sok megk�nny�t�s�hez. Ennek megfelel�en minden �bra (\emph{figure}) c�mk�je \verb+fig:+ kulcssz�val kezd�dik, m�g minden t�bl�zat (\emph{table}), k�plet (\emph{equation}), fejezet (\emph{section}) �s lista (\emph{listing}) rendre a \verb+tab:+, \verb+eq:+, \verb+sect:+ �s \verb+listing:+ kulcssz�val kezd�dik, �s a kulcsszavak ut�n tetsz�legesen v�lasztott c�mke haszn�lhat�. Ha ezt a konvenci�t betartjuk, akkor az el�bbi objektumok sz�m�ra rendre a \verb+\figref+, \verb+\tabref+, \verb+\eqref+, \verb+\sectref+ �s \verb+\listref+ makr�kkal hivatkozhatunk. A makr�k param�tere a c�mke, amelyre hivatkozunk (a kulcssz� n�lk�l). Az �sszes eml�tett hivatkoz�st�pus, bele�rtve az \verb+\url+ kulcssz�val bevezetett web-hivatkoz�sokat is a  \verb+hyperref+\footnote{Seg�ts�g�vel a dokumentumban megjelen� hivatkoz�sok nem csak dinamikuss� v�lnak, de sz�nezhet�k is, b�vebbet err�l a csomag dokument�ci�j�ban tal�lunk. Ez egy�ttal egy p�lda l�bjegyzet �r�s�ra.} csomagnak k�sz�nhet�en akt�vak a legt�bb PDF-n�zeget�ben, r�juk kattintva a dokumentum megfelel� oldal�ra ugrik a PDF-n�z� vagy a megfelel� linket megnyitja az alap�rtelmezett b�ng�sz�vel. A \verb+hyperref+ csomag a kimeneti PDF-dokumentumba k�nyvjelz�ket is k�sz�t a tartalomjegyz�kb�l. Ez egy szint�n akt�v tartalomjegyz�k, amelynek elemeire kattintva a n�zeget� behozza a kiv�lasztott fejezetet.

%----------------------------------------------------------------------------
\section{�br�k �s t�bl�zatok}
%----------------------------------------------------------------------------
A k�peket PDFLaTeX eset�n a vesztes�gmentes PNG, valamint a vesztes�ges JPEG form�tumban �rdemes elmenteni. Az EPS (PostScript) vektorgrafikus k�pform�tum beilleszt�s�t a PDFLatex k�zvetlen�l nem t�mogatja. Ehelyett egy lehet�s�g 200 dpi, vagy ann�l nagyobb felbont�sban raszteriz�lni a k�pet, �s PNG form�tumban elmenteni. Az egyes k�pek m�rete �ltal�ban nem, de sok k�p eset�n a dokumentum �sszm�rete �gy m�r szignifik�ns is lehet. A dokumentumban felhaszn�lt k�pf�jlokat a dokumentum forr�sa mellett �rdemes tartani, archiv�lni, mivel ezek hi�ny�ban a dokumentum nem fordul �jra. Ha lehet, a vektorgrafikus k�peket vektorgrafikus form�tumban is �rdemes elmenteni az �jrafelhaszn�lhat�s�g (az �tszerkeszthet�s�g) �rdek�ben.

Kapcsol�si rajzok legt�bbsz�r kim�solhat�k egy vektorgrafikus programba (pl. CorelDraw) �s onnan nagyobb felbont�ssal raszteriz�lva kimenthat�k PNG form�tumban. Ugyanakkor kiv�l� �br�k k�sz�thet�k Microsoft Visio vagy hasonl� program haszn�lat�val is: Visio-b�l az �br�k k�zvetlen�l PNG-be is menthet�k.

Lehet�s�geink Matlab �br�k eset�n:
\begin{itemize}
	\item K�perny�lop�s (\emph{screenshot}) is elfogadhat� min�s�g� lehet a dokumentumban, de �ltal�ban jobb felbont�st is el lehet �rni m�s m�dszerrel.
	\item A Matlab �br�t a \verb+File/Save As+ opci�val lementhetj�k PNG form�tumban (ugyanaz itt is �rv�nyes, mint kor�bban, ez�rt nem javasoljuk).
	\item A Matlab �br�t az \verb+Edit/Copy figure+ opci�val kim�solhatjuk egy vektorgrafikus programba is �s onnan nagyobb felbont�ssal raszteriz�lva kimenthatj�k PNG form�tumban (nem javasolt).
	\item Javasolt megold�s: az �br�t a \verb+File/Save As+ opci�val EPS \emph{vektorgrafikus} form�tumban elmentj�k, PDF-be konvert�lva beillesztj�k a dolgozatba.
\end{itemize}
Az EPS k�p az \verb+epstopdf+ programmal\footnote{a kor�bban eml�tett \LaTeX-disztrib�ci�kban megtal�lhat�} konvert�lhat� PDF form�tumba. C�lszer� egy batch-f�jlt k�sz�teni az �sszes EPS �bra leford�t�s�ra az al�bbi m�don (ez Windows alatt m�k�dik).
\begin{lstlisting}[frame=single,float=!ht]
@echo off
for %%j in (*.eps) do (
echo converting file "%%j"
epstopdf "%%j"
)
echo done .
\end{lstlisting}

Egy ilyen parancsf�jlt (\verb+convert.cmd+) elhelyezt�k a sablon \verb+figures\eps+ k�nyvt�r�ba, �gy a felhaszn�l�nak csak annyi a dolga, hogy a \verb+figures\eps+ k�nyvt�rba kimenti az EPS form�tum� vektorgrafikus k�pet, majd lefuttatja a \verb+convert.cmd+ parancsf�jlt, ami PDF-be konvert�lja az EPS f�jlt.

Ezek ut�n a PDF-�br�t ugyan�gy lehet a dokumentumba beilleszteni, mint a PNG-t vagy a JPEG-et. A megold�s el�nye, hogy a leford�tott dokumentumban is vektorgrafikusan t�rol�dik az �bra, �gy a m�rete j�val kisebb, mintha raszteriz�ltuk volna beilleszt�s el�tt. Ez a m�dszer minden -- az EPS form�tumot ismer� -- vektorgrafikus program (pl. CorelDraw) eset�n is haszn�lhat�.

A k�pek beilleszt�s�re az \sectref{LatexTools}. fejezetben mutattunk be p�ld�t (\figref{TexnicCenter}~�bra). Az el�z� mondatban egy�ttal az automatikusan felold�d� �brahivatkoz�sra is l�thatunk p�ld�t. T�bb k�pf�jlt is beilleszthet�nk egyetlen �br�ba. Az egyes k�pek k�z�tti horizont�lis �s vertik�lis marg�t metrikusan szab�lyozhatjuk (\figref{HVSpaces}~�bra). Az �br�k elhelyez�s�t sz�mtalan tipogr�fiai szab�ly egyidej� teljes�t�s�vel a ford�t� maga v�gzi, a dokumentum �r�ja csak preferenci�it jelezheti a ford�t� fel� (olykor ez bossz�s�got is okozhat, ilyenkor pl. a k�p m�ret�vel lehet j�tszani).

\begin{figure}[!ht]
\centering
\includegraphics[width=67mm, keepaspectratio]{figures/TeXnicCenter.png}\hspace{1cm}
\includegraphics[width=67mm, keepaspectratio]{figures/TeXnicCenter.png}\\\vspace{5mm}
\includegraphics[width=67mm, keepaspectratio]{figures/TeXnicCenter.png}\hspace{1cm}
\includegraphics[width=67mm, keepaspectratio]{figures/TeXnicCenter.png}
\caption{T�bb k�pf�jl beilleszt�se eset�n t�rk�z�ket is �rdemes haszn�lni.} 
\label{fig:HVSpaces}
\end{figure}

A t�bl�zatok haszn�lat�ra a \tabref{TabularExample}~t�bl�zat mutat p�ld�t.
A t�bl�zat c�mk�je nem v�letlen�l ker�lt a t�bl�zat f�l�, ez a szokv�nyos.
\begin{table}[ht]
	\footnotesize
	\centering
	\caption{Az �rajel-gener�tor chip �rajel-kimenetei.} \label{tab:SysClocks}
	\begin{tabular}{ | l | c | c |}
	\hline
	�rajel & Frekvencia & C�l pin \\ \hline
	CLKA & 100 MHz & FPGA CLK0\\
	CLKB & 48 MHz  & FPGA CLK1\\
	CLKC & 20 MHz  & Processzor\\
	CLKD & 25 MHz  & Ethernet chip \\
	CLKE & 72 MHz  & FPGA CLK2\\
	XBUF & 20 MHz  & FPGA CLK3\\
	\hline
	\end{tabular}
	\label{tab:TabularExample}
\end{table}


%----------------------------------------------------------------------------
\section{Felsorol�sok �s list�k}
%----------------------------------------------------------------------------
Sz�mozatlan felsorol�sra mutat p�ld�t a jelenlegi bekezd�s:
\begin{itemize}
	\item \emph{els� bajusz:} ide lehetne �rni az els� elem kifej�s�t,
	\item \emph{m�sodik bajusz:} ide lehetne �rni a m�sodik elem kifej�s�t,
	\item \emph{ez meg egy szak�ll:} ide lehetne �rni a harmadik elem kifej�s�t.
\end{itemize}

Sz�mozott felsorol�st is k�sz�thet�nk az al�bbi m�don:
\begin{enumerate}
	\item \emph{els� bajusz:} ide lehetne �rni az els� elem kifej�s�t, �s ez a kifejt�s �gy n�z ki, ha t�bb sorosra sikeredik,
	\item \emph{m�sodik bajusz:} ide lehetne �rni a m�sodik elem kifej�s�t,
	\item \emph{ez meg egy szak�ll:} ide lehetne �rni a harmadik elem kifej�s�t.
\end{enumerate}
A felsorol�sokban sorok v�g�n vessz�, az utols� sor v�g�n pedig pont a szok�sos �r�sjel. Ez al�l kiv�telt k�pezhet, ha az egyes elemek t�bb teljes mondatot tartalmaznak.

List�kban a dolgozat sz�veg�t�l elk�l�n�tend� k�dr�szleteket, programsorokat, pszeudo-k�dokat jelen�thet�nk meg (\listref{Example}~lista). 
\begin{lstlisting}[frame=single,float=!ht,caption=A fenti sz�mozott felsorol�s \LaTeX- forr�sk�dja, label=listing:Example]
\begin{enumerate}
	\item \emph{els� bajusz:} ide lehetne �rni az els� elem kifej�s�t, 
	�s ez a kifejt�s �gy n�z ki, ha t�bb sorosra sikeredik,
	\item \emph{m�sodik bajusz:} ide lehetne �rni a m�sodik elem kifej�s�t,
	\item \emph{ez meg egy szak�ll:} ide lehetne �rni a harmadik elem kifej�s�t.
\end{enumerate}
\end{lstlisting}
A lista keret�t, h�tt�rsz�n�t, eg�sz st�lus�t megv�laszthatjuk. R�ad�sul k�l�nf�le programnyelveket �s a nyelveken bel�l kulcsszavakat is defini�lhatunk, ha sz�ks�ges. Err�l b�vebbet a \verb+listings+ csomag hivatalos le�r�s�ban tal�lhatunk.

%----------------------------------------------------------------------------
\section{K�pletek}
%----------------------------------------------------------------------------
Ha egy formula nem t�ls�gosan hossz�, �s nem akarjuk hivatkozni a sz�vegb�l, mint p�ld�ul a $e^{i\pi}+1=0$ k�plet, \emph{sz�vegk�zi k�pletk�nt} szok�s le�rni. Csak, hogy m�sik p�ld�t is l�ssunk, az $U_i=-d\Phi/dt$ Faraday-t�rv�ny a $\rot E=-\frac{dB}{dt}$ differenci�lis alakban adott Maxwell-egyenlet fel�letre vett integr�lj�b�l vezethet� le. L�that�, hogy a \LaTeX-ford�t� a sork�z�ket betartja, �gy a sz�veg szed�se eszt�tikus marad sz�vegk�zi k�pletek haszn�lata eset�n is.

K�pletek eset�n az �ltal�nos konvenci�, hogy a kisbet�k skal�rt, a kis f�lk�v�r bet�k ($\mathbf{v}$) oszlopvektort -- �s ennek megfelel�en $\mathbf{v}^T$ sorvektort -- a kapit�lis f�lk�v�r bet�k ($\mathbf{V}$) m�trixot jel�lnek. Ha ett�l el szeretn�nk t�rni, akkor az alkalmazni k�v�nt jel�l�sm�dot c�lszer� k�l�n alfejezetben defini�lni. Ennek megfelel�en, amennyiben $\mathbf{y}$ jel�li a m�r�sek vektor�t, $\mathbf{\vartheta}$ a param�terek vektor�t �s $\hat{\mathbf{y}}=\mathbf{X}\vartheta$ a param�terekben line�ris modellt, akkor a \emph{Least-Squares} �rtelemben optim�lis param�terbecsl� $\hat{\mathbf{\vartheta}}_{LS}=(\mathbf{X}^T\mathbf{X})^{-1}\mathbf{X}^T\mathbf{y}$ lesz.

Emellett kiemelt, sorsz�mozott k�pleteket is megadhatunk, enn�l az \verb+equation+ �s a \verb+eqnarray+ k�rnyezetek helyett a korszer�bb \verb+align+ k�rnyezet alkalmaz�s�t javasoljuk (t�bb okb�l, k�l�nf�le probl�m�k elker�l�se v�gett, amelyekre most nem t�r�nk ki). Teh�t
\begin{align}
\dot{\mathbf{x}}&=\mathbf{A}\mathbf{x}+\mathbf{B}\mathbf{u},\\
\mathbf{y}&=\mathbf{C}\mathbf{x},
\end{align}
ahol $\mathbf{x}$ az �llapotvektor, $\mathbf{y}$ a m�r�sek vektora �s $\mathbf{A}$, $\mathbf{B}$ �s $\mathbf{C}$ a rendszert le�r� param�term�trixok. Figyelj�k meg, hogy a k�t egyenletben az egyenl�s�gjelek egym�shoz igaz�tva jelennek meg, mivel a mindkett�t az \& karakter el�zi meg a k�dban. Lehet�s�g van sz�mozatlan kiemelt k�plet haszn�lat�ra is, p�ld�ul
\begin{align}
\dot{\mathbf{x}}&=\mathbf{A}\mathbf{x}+\mathbf{B}\mathbf{u},\nonumber\\
\mathbf{y}&=\mathbf{C}\mathbf{x}\nonumber.
\end{align}
M�trixok fel�r�s�ra az $\mathbf{A}\mathbf{x}=\mathbf{b}$ inhomog�n line�ris egyenlet r�szletes kifejt�s�vel mutatunk p�ld�t:
\begin{align}
\begin{bmatrix}
a_{11} & a_{12} & \dots & a_{1n}\\
a_{21} & a_{22} & \dots & a_{2n}\\
\vdots & \vdots & \ddots & \vdots\\
a_{m1} & a_{m2} & \dots & a_{mn}
\end{bmatrix}
\begin{pmatrix}x_1\\x_2\\\vdots\\x_n\end{pmatrix}=
\begin{pmatrix}b_1\\b_2\\\vdots\\b_m\end{pmatrix}.
\end{align}
A \verb+\frac+ utas�t�s hat�konys�g�t egy �ltal�nos m�sodfok� tag �tviteli f�ggv�ny�n kereszt�l mutatjuk be, azaz
\begin{align}
W(s)=\frac{A}{1+2T\xi s+s^2T^2}.
\end{align}
A matematikai m�d minden szimb�lum�nak �s k�pess�g�nek a bemutat�s�ra term�szetesen itt nincs lehet�s�g, de gyors referenciak�nt hat�konyan haszn�lhat�k a k�vetkez� linkek:\\
\indent\url{http://www.artofproblemsolving.com/LaTeX/AoPS_L_GuideSym.php},\\
\indent\url{http://www.ctan.org/tex-archive/info/symbols/comprehensive/symbols-a4.pdf},\\
\indent\url{ftp://ftp.ams.org/pub/tex/doc/amsmath/short-math-guide.pdf}.\\
Ez pedig itt egy magyar�zat, hogy mi�rt �rdemes \verb+align+ k�rnyezetet haszn�lni:\\
\indent\url{http://texblog.net/latex-archive/maths/eqnarray-align-environment/}.

%----------------------------------------------------------------------------
\section{Irodalmi hivatkoz�sok}\label{sect:HowtoReference}
%----------------------------------------------------------------------------
Egy \LaTeX dokumentumban az irodalmi hivatkoz�sok defin�ci�j�nak k�t m�dja van. Az egyik a \verb+\thebibliograhy+ k�rnyezet haszn�lata a dokumentum v�g�n, az \verb+\end{document}+ lez�r�s el�tt.
\begin{lstlisting}[frame=single,float=!ht]
\begin{thebibliography}{9}

\bibitem{Lamport94} Leslie Lamport, \emph{\LaTeX: A Document Preparation System}. 
Addison Wesley, Massachusetts, 2nd Edition, 1994.

\end{thebibliography}
\end{lstlisting}

Ezek ut�n a dokumentumban a \verb+\cite{Lamport94}+ utas�t�ssal hivatkozhatunk a forr�sra. A fenti megad�s viszonylag k�tetlen, a szerz� maga form�zza az irodalomjegyz�ket. 

Egy sokkal professzion�lisabb m�dszer a BiB\TeX~haszn�lata, ez�rt ez a sablon is ezt t�mogatja. Ebben az esetben egy k�l�n sz�veges adatb�zisban defini�ljuk a forr�smunk�kat, �s egy k�l�n st�lusf�jl hat�rozza meg az irodalomjegyz�k kin�zet�t. Ez, �sszhangban azzal, hogy k�l�n form�tumkonvenci� hat�rozza meg a foly�irat-, a k�nyv-, a konferenciacikk- stb. hivatkoz�sok kin�zet�t az irodalomjegyz�kben (a sablon haszn�lata eset�n ezzel nem is kell foglalkoznia a hallgat�nak, de az eredm�nyt c�lszer� ellen�rizni). A felhaszn�lt hivatkoz�sok adatb�zisa egy \verb+.bib+ kiterjeszt�s� sz�veges f�jl, amelynek szerkezet�t a \listref{Bibtex} k�dr�szlet demonstr�lja. A forr�smunk�k bevitelekor a sor v�gi vessz�k k�l�n figyelmet ig�nyelnek, mert hi�nyuk a BiB\TeX-ford�t� hiba�zenet�t eredm�nyezi. A forr�smunk�kat t�pus szerinti kulcssz� vezeti be (\verb+@book+ k�nyv, \verb+@inproceedings+ konferenciakiadv�nyban megjelent cikk, \verb+@article+ foly�iratban megjelent cikk, \verb+@techreport+ valamelyik egyetem gondoz�s�ban megjelent m�szaki tanulm�ny, \verb+@manual+ m�szaki dokument�ci� eset�n stb.). Nemcsak a megjelen�s st�lusa, de a k�telez�en megadand� mez�k is t�pusr�l-t�pusra v�ltoznak. Egy j�l haszn�lhat� referencia a \url{http://en.wikipedia.org/wiki/BibTeX} oldalon tal�lhat�.
\begin{lstlisting}[frame=single,float=!ht,caption=P�lda sz�veges irodalomjegyz�k-adatb�zisra BiBTeX haszn�lata eset�n., label=listing:Bibtex]
@BOOK{Wettl04,
  author="Ferenc Wettl and Gyula Mayer and P�ter Szab�",
  title="\LaTeX~k�zik�nyv",
  publisher="Panem K�nyvkiad�",
  year=2004
}
@ARTICLE{Candy86,
  author ="James C. Candy",
  title  ="Decimation for Sigma Delta Modulation",
  journal="{IEEE} Trans.\ on Communications",
  volume =34,
  number =1,
  pages  ="72--76",
  month  =jan,
  year   =1986,
}
@INPROCEEDINGS{Lee87,
  author =       "Wai L. Lee and Charles G. Sodini",
  title =        "A Topology for Higher Order Interpolative Coders",
  booktitle =    "Proc.\ of the IEEE International Symposium on 
  Circuits and Systems",
  year =         1987,
  vol =          2,
  month =        may # "~4--7",
  address =      "Philadelphia, PA, USA",
  pages =        "459--462"
}
@PHDTHESIS{KissPhD,
  author =   "Peter Kiss",
  title =    "Adaptive Digital Compensation of Analog Circuit Imperfections 
  for Cascaded Delta-Sigma Analog-to-Digital Converters",
  school =   "Technical University of Timi\c{s}oara, Romania",
  month =    apr,
  year =     2000
}
@MANUAL{Schreier00,
  author = "Richard Schreier",
  title  = "The Delta-Sigma Toolbox v5.2",
  organization = "Oregon State University",
  year   = 2000,
  month  = jan,
  note   ="\newline URL: http://www.mathworks.com/matlabcentral/fileexchange/"
}
@MISC{DipPortal,
	author="Budapesti {M}�szaki �s {G}azdas�gtudom�nyi {E}gyetem 
	{V}illamosm�rn�ki �s {I}nformatikai {K}ar",
  title="{D}iplomaterv port�l (2011 febru�r 26.)",
  howpublished="\url{http://diplomaterv.vik.bme.hu/}",
}}
\end{lstlisting}

A st�lusf�jl egy \verb+.sty+ kiterjeszt�s� f�jl, de ezzel l�nyeg�ben nem kell foglalkozni, mert vannak be�p�tett st�lusok, amelyek j�l haszn�lhat�k. Ez a sablon a BiB\TeX-et haszn�lja, a hozz� tartoz� adatb�zisf�jl a \verb+mybib.bib+ f�jl. Megfigyelhet�, hogy az irodalomjegyz�ket a dokumentum v�g�re (a \verb+\end{document}+ utas�t�s el�) beillesztett \verb+\bibliography{mybib}+ utas�t�ssal hozhatjuk l�tre, a st�lus�t pedig ugyanitt a  \verb+\bibliographystyle{plain}+ utas�t�ssal adhatjuk meg. Ebben az esetben a \verb+plain+ el�re defini�lt st�lust haszn�ljuk (a sablonban is ezt �ll�tottuk be). A \verb+plain+ st�luson k�v�l term�szetesen sz�mtalan m�s el�re defini�lt st�lus is l�tezik. Mivel a \verb+.bib+ adatb�zisban ezeket megadtuk, a BiB\TeX-ford�t� is meg tudja k�l�nb�ztetni a szerz�t a c�mt�l �s a kiad�t�l, �s ez alapj�n automatikusan gener�l�dik az irodalomjegyz�k a st�lusf�jl �ltal meghat�rozott st�lusban.

Az egyes forr�smunk�kra a sz�vegb�l tov�bbra is a \verb+\cite+ paranccsal tudunk hivatkozni, �gy a \listref{Bibtex} k�dr�szlet eset�n a hivatkoz�sok rendre \verb+\cite{Wettl04}+, \verb+\cite{Candy86}+, \verb+\cite{Lee87}+, \verb+\cite{KissPhD}+, \verb+\cite{Schreirer00}+ �s \verb+\cite{DipPortal}+. Az irodalomjegyz�kben alap�rtelmez�sben csak azok a forr�smunk�k jelennek meg, amelyekre tal�lhat� hivatkoz�s a sz�vegben, �s ez �gy alapvet�en helyes is, hiszen olyan forr�smunk�kat nem illik az irodalomjegyz�kbe �rni, amelyekre nincs hivatkoz�s.

Mivel a ford�t�si folyamat sor�n t�bb l�p�sben old�dnak fel a szimb�lumok, ez�rt gyakran t�bbsz�r (TeXLive �s TeXnicCenter eset�n 2-3-szor) is le kell ford�tani a dokumentumot. Ilyenkor ez els� 1-2 ford�t�s esetleg szimb�lum-felold�sra vonatkoz� figyelmeztet� �zenettel z�rul. Ha hiba�zenettel z�rul b�rmelyik ford�t�s, akkor nincs �rtelme megism�telni, hanem a hib�t kell megkeresni. A \verb+.bib+ f�jl megv�ltoztat�skor sokszor nincs hat�sa a v�ltoztat�snak azonnal, mivel nem mindig fut �jra a BibTeX ford�t�. Ez�rt c�lszer� a v�ltoztat�s ut�n azt manu�lisan is lefuttatni (TeXnicCenter eset�n \verb+Build/BibTeX+).

Hogy a sz�vegbe �gyazott hivatkoz�sok kin�zet�t demonstr�ljuk, itt most sorban meghivatkozzuk a \cite{Wettl04}, \cite{Candy86}, \cite{Lee87}, \cite{KissPhD} �s az \cite{Schreier00} forr�smunk�t, valamint az \cite{DipPortal} weboldalt.

Megjegyzend�, hogy az �kezetes magyar bet�ket is tartalmaz� \verb+.bib+ f�jl az \verb+inputenc+ csomaggal bet�lt�tt \verb+latin2+ bet�k�szlet miatt ford�that�. Ugyanez a \verb+.bib+ f�jl hiba�zenettel fordul egy olyan dokumentumban, ami nem tartalmazza a \verb+\usepackage[latin2]{inputenc}+ sort. Speci�lis ig�ny eset�n az irodalmi adatb�zis �ltal�nosabb �rv�ny�v� tehet�, ha az �kezetes bet�ket speci�lis latex karakterekkel helyettes�tj�k a \verb+.bib+ f�jlban, pl. � helyett \verb+\'{a}+-t vagy � helyett \verb+\H{o}+-t �runk. 

Oldalt�r�s k�vetkezik (ld. forr�s).
\newpage

%----------------------------------------------------------------------------
\section{A dolgozat szerkezete �s a forr�sf�jlok}
%----------------------------------------------------------------------------
A diplomatervsablon (a kari ir�nyelvek szerint) az al�bbi f� fejezetekb�l �ll:
\begin{enumerate}
	\item 1 oldalas \emph{t�j�koztat�} a szakdolgozat/diplomaterv szerkezet�r�l (\verb+guideline.tex+), ami a v�gs� dolgozatb�l t�rlend�,
	\item \emph{feladatki�r�s} (\verb+project.tex+), a dolgozat nyomtatott verz�j�ban ennek a hely�re ker�l a tansz�k �ltal kiadott, a tansz�kvezet� �ltal al��rt feladatki�r�s, a dolgozat elektronikus verzi�j�ba pedig a feladatki�r�s egy�ltal�n ne ker�lj�n bele, azt k�l�n t�lti fel a tansz�k a diplomaterv-honlapra,
	\item \emph{c�moldal} (\verb+titlepage.tex+),
	\item \emph{tartalomjegyz�k} (\verb+diploma.tex+),
	\item a diplomatervez� \emph{nyilatkozat}a az �n�ll� munk�r�l (\verb+declaration.tex+),
	\item 1-2 oldalas tartalmi \emph{�sszefoglal�} magyarul �s angolul, illetve elk�sz�thet� m�g tov�bbi nyelveken is (\verb+abstract.tex+),
	\item \emph{bevezet�s}: a feladat �rtelmez�se, a tervez�s c�lja, a feladat indokolts�ga, a diplomaterv fel�p�t�s�nek r�vid �sszefoglal�sa (\verb+introduction.tex+),
	\item sorsz�mmal ell�tott \emph{fejezetek}: a feladatki�r�s pontos�t�sa �s r�szletes elemz�se, el�zm�nyek (irodalomkutat�s, hasonl� alkot�sok), az ezekb�l levonhat� k�vetkeztet�sek, a tervez�s r�szletes le�r�sa, a d�nt�si lehet�s�gek �rt�kel�se �s a v�lasztott megold�sok indokl�sa, a megtervezett m�szaki alkot�s �rt�kel�se, kritikai elemz�se, tov�bbfejleszt�si lehet�s�gek (\verb+chapter{1,2..n}.tex+),
	\item esetleges \emph{k�sz�netnyilv�n�t�s}ok (\verb+acknowledgement.tex+),
	\item r�szletes �s pontos \emph{irodalomjegyz�k} (ez a sablon eset�ben automatikusan gener�l�dik a \verb+diploma.tex+ f�jlban elhelyezett \verb+\bibliography+ utas�t�s hat�s�ra, a \sectref{HowtoReference}. fejezetben le�rtak szerint),
	\item \emph{f�ggel�kek} (\verb+appendices.tex+).
\end{enumerate}

A sablonban a fejezetek a \verb+diploma.tex+ f�jlba vannak beillesztve \verb+\include+ utas�t�sok seg�ts�g�vel. Lehet�s�g van arra, hogy csak az �ppen szerkeszt�s alatt �ll� \verb+.tex+ f�jlt ford�tsuk le, ezzel ler�vid�tve a ford�t�si folyamatot. Ezt a lehet�s�get az al�bbi k�dr�szlet biztos�tja a \verb+diploma.tex+ f�jlban.
\begin{lstlisting}[frame=single,float=!ht]
\includeonly{
	guideline,%
	project,%
	titlepage,%
	declaration,%
	abstract,%
	introduction,%
	chapter1,%
	chapter2,%
	chapter3,%
	acknowledgement,%
	appendices,%
}
\end{lstlisting}

Ha az al�bbi k�dr�szletben az egyes sorokat a \verb+%+ szimb�lummal kikommentezz�k, akkor a megfelel� \verb+.tex+ f�jl nem fordul le. Az oldalsz�mok �s a tartalomjegy�k term�szetesen csak akkor billennek helyre, ha a teljes dokumentumot leford�tjuk.

%----------------------------------------------------------------------------
\newpage
\section{Alapadatok megad�sa}
%----------------------------------------------------------------------------
A diplomaterv alapadatait (c�m, szerz�, konzulens, konzulens titulusa) a \verb+diploma.tex+ f�jlban lehet megadni az al�bbi k�dr�szlet m�dos�t�s�val.
\begin{lstlisting}[frame=single,float=!ht]
\newcommand{\vikszerzo}{B�dis-Szomor� Andr�s}
\newcommand{\vikkonzulens}{dr.~Konzulens Elem�r}
\newcommand{\vikcim}{Elektronikus terel�k}
\newcommand{\viktanszek}{M�r�stechnika �s Inform�ci�s Rendszerek Tansz�k}
\newcommand{\vikdoktipus}{Diplomaterv}
\newcommand{\vikdepartmentr}{B�dis-Szomor� Andr�s}
\end{lstlisting}

%----------------------------------------------------------------------------
\section{�j fejezet �r�sa}
%----------------------------------------------------------------------------
A f�fejezetek k�l�n \verb+chapter{1..n}.tex+ f�jlban foglalnak helyet. A sablonhoz 3 fejezet k�sz�lt. Tov�bbi f�fejezeteket �gy hozhatunk l�tre, ha �j \verb+chapter{i}.tex+ f�jlt k�sz�t�nk a fejezet sz�m�ra, �s a \verb+diploma.tex+ f�jlban, a \verb+\include+ �s \verb+\includeonly+ utas�t�sok argumentum�ba felvessz�k az �j \verb+.tex+ f�jl nev�t.







%----------------------------------------------------------------------------
\chapter*{Acknowledgement}\addcontentsline{toc}{chapter}{Acknowledgement}
%----------------------------------------------------------------------------

Fanni,Pali,Sch�nherz

Ez nem k�telez�, ak�r t�r�lhet� is. Ha a szerz� sz�ks�g�t �rzi, itt lehet k�sz�netet nyilv�n�tani azoknak, akik hozz�j�rultak munk�jukkal ahhoz, hogy a hallgat� a szakdolgozatban vagy diplomamunk�ban le�rt feladatokat sikeresen elv�gezze. A konzulensnek val� k�sz�netnyilv�n�t�s sem k�telez�, a konzulensnek hivatalosan is dolga, hogy a hallgat�t konzult�lja.
%\listoffigures\addcontentsline{toc}{chapter}{�br�k jegyz�ke}
%\listoftables\addcontentsline{toc}{chapter}{T�bl�zatok jegyz�ke}
\renewcommand{\bibname}{Bibliography}
\bibliography{mybib}
%TODO
\addcontentsline{toc}{chapter}{Bibliography}

\bibliographystyle{plain}

%----------------------------------------------------------------------------
\appendix
%----------------------------------------------------------------------------
\chapter*{Appendix}\addcontentsline{toc}{chapter}{Appendix}
\setcounter{chapter}{1}  % a fofejezet-szamlalo az angol ABC 6. betuje (F) lesz
\setcounter{equation}{0} % a fofejezet-szamlalo az angol ABC 6. betuje (F) lesz
\numberwithin{equation}{section}
\numberwithin{figure}{section}
\numberwithin{lstlisting}{section}
%\numberwithin{tabular}{section}
%----------------------------------------------------------------------------
\section{CAN fingerprint RUL calculation workflow:}
%----------------------------------------------------------------------------
\label{appendix:CANFPRULCalc}
%\begin{figure}[!ht]
%\centering
%\includegraphics[width=150mm, keepaspectratio]{figures/TeXnicCenter.png}
%\caption{A TeXnicCenter Windows alap� \LaTeX-szerkeszt�.} 
%\end{figure}
	\begin{figure}[!ht]
		\centering
		\includegraphics[width=150mm, angle =-90, keepaspectratio]{figures/CAN_fingerprint_RUL_calculation.png}
		\caption{Calculating RUL from cleaned data} 
	\end{figure}
%----------------------------------------------------------------------------
\clearpage\section{Electrical failure prediction preprocessing workflow}
%----------------------------------------------------------------------------
\label{appendix:FaultPredPreProc}
\begin{figure}[!ht]
		\centering
		\includegraphics[width=150mm, angle =-90,keepaspectratio]{figures/Electrical_fail_prediction_preprocessing_workflow.png}
		\caption{Cleaning and ordering step by step} 
\end{figure}

%A Pitagorasz-t�telb�l levezetve
%\begin{align}
%c^2=a^2+b^2=42.
%\end{align}
%A Faraday-indukci�s t�rv�nyb�l levezetve
%\begin{align}
%\rot E=-\frac{dB}{dt}\hspace{1cm}\longrightarrow \hspace{1cm}
%U_i=\oint\limits_\mathbf{L}{\mathbf{E}\mathbf{dl}}=-\frac{d}{dt}\int\limits_A{\mathbf{B}\mathbf{da}}=42.
%\end{align}
%----------------------------------------------------------------------------
\clearpage\section{Electrical failure prediction RUL calculation workflow}
%----------------------------------------------------------------------------
\label{appendix:FaultPredRULCalc}
		\#TODO explain red flags in short
		\begin{figure}[!ht]
		\centering
		\includegraphics[width=150mm, angle =-90,keepaspectratio]{figures/Electrical_failure_pred_RUL.png}
		\caption{Electrical failure prediction RUL calculation workflow} 
		\end{figure}





\label{page:last}
\end{document}
