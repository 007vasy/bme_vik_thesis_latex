%----------------------------------------------------------------------------
\chapter{Implementation}
%----------------------------------------------------------------------------
\section{CAN fingerprint data set processing method}
\subsection\noindent
This sections would describe the algoritmization and realization of the previously described process in the design chapter in R language.The reasons are simple, the R is one of the most common and convinient tools to process data sets and interpret them, with a fitted statistical or machine learning model.
%----------------------------------------------------------------------------
\subsection{Cleaning}
%----------------------------------------------------------------------------
		\subsubsection{Idea One}
		Implementation the IDEA One and problems with that
		#TODO use github and describe the algorithm
		#WHY
%----------------------------------------------------------------------------
		\subsubsection{Idea Two}
		Implementation the IDEA Two and problems with that
		#WHY
%----------------------------------------------------------------------------
		\subsubsection{Idea Synthesis}
		Implementation the IDEA Synthesis and problems with that and summary why is it the best solution
		#WHY
%----------------------------------------------------------------------------		
\subsection{Data processing and understanding}
		Reasons describe: Why is is useful to understand the given data in depth
%----------------------------------------------------------------------------
	\subsubsection{Exploratory data analysis}
		No pain, no gain. Why is is useful
		confirm expert knowledge
%
----------------------------------------------------------------------------
		\paragraph{Data summaries}
		Common Types and applied types
		confirm expert knowledge
		#WHY
%----------------------------------------------------------------------------		
		\paragraph{Visualization}
		Common Tool and applied tool
		\begin{figure}[!ht]
			\centering
			\includegraphics[width=150mm, keepaspectratio]{figures/gen_files/can_f/clusteringf.png}
			\caption{ToDo} 
			\end{figure}
		\begin{figure}[!ht]
			\centering
			\includegraphics[width=150mm, keepaspectratio]{figures/gen_files/can_f/CrashYvsSteeringangle.png}
			\caption{ToDo} 
			\end{figure}
		\begin{figure}[!ht]
			\centering
			\includegraphics[width=150mm, keepaspectratio]{figures/gen_files/can_f/Rplot.png}
			\caption{ToDo} 
			\end{figure}		
		\begin{figure}[!ht]
			\centering
			\includegraphics[width=150mm, keepaspectratio]{figures/gen_files/can_f/speedvstorquefacet.png}
			\caption{ToDo} 
			\end{figure}	
		\begin{figure}[!ht]
			\centering
			\includegraphics[width=150mm, keepaspectratio]{figures/gen_files/can_f/timevssteeringangle.png}
			\caption{ToDo} 
			\end{figure}	
		\begin{figure}[!ht]
			\centering
			\includegraphics[width=150mm, keepaspectratio]{figures/gen_files/can_f/torquespeedcat.png}
			\caption{ToDo} 
			\end{figure}		
		confirm expert knowledge
%----------------------------------------------------------------------------		
	\subsubsection{Unsupervised learning}
			\begin{figure}[!ht]
			\centering
			\includegraphics[width=150mm, keepaspectratio]{figures/gen_files/can_f/clustering.png}
			\caption{ToDo} 
			\end{figure}
	description and why some of that is used.
	confirm expert knowledge
	#WHY
%----------------------------------------------------------------------------	
\subsection{Calculating Attributes}
	from expert knowledge, why does the solution have worked out this way
	#WHY
%----------------------------------------------------------------------------	
\subsection{Calculating RUL (Remaining useful life)}
	model description in detail.
	#WHY
%----------------------------------------------------------------------------	
\subsection{Optimization}
Possible optimizations enumerate.
#WHY
\subsection{Results}
Estimating result, and what causing aleatory mistakes in the model. 
#WHY
\cite{GitHub_CAN_RUL}
%----------------------------------------------------------------------------
%----------------------------------------------------------------------------
\section{Electrical fail prediction data set processing method}
\subsection\noindent
This sections would describe the algoritmization and realization of the previously described process in the design chapter in Python language.
#WHY
%----------------------------------------------------------------------------
	\subsection{Cleaning}
Cleaning the data set in detail, why I made these choices with that In the graph target in mind
#WHY
		#TODO use github and describe the algorithm
%----------------------------------------------------------------------------
	\subsection{Exploring}
		\paragraph{Data summaries}
		Common Types and applied types
		confirm expert knowledge
		#WHY
%----------------------------------------------------------------------------		
		\paragraph{Visualization}
		Common Tool and applied tool
			\begin{figure}[!ht]
			\centering
			\includegraphics[width=150mm, keepaspectratio]{figures/gen_files/fault_pred_files/path.png}
			\caption{ToDo} 
			\end{figure}
		confirm expert knowledge
#WHY
%----------------------------------------------------------------------------
	\subsection{Calculating RUL}
model description in detail.
#WHY
%----------------------------------------------------------------------------
	\subsection{Optimization}
Possible optimizations enumerate.
#WHY
%----------------------------------------------------------------------------
	\subsection{Result}
Estimating result, and what causing aleatory mistakes in the model. 
#WHY
\cite{GitHub_FP_RUL}