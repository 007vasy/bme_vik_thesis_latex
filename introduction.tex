%----------------------------------------------------------------------------
\chapter*{Introduction}\addcontentsline{toc}{chapter}{Introduction}
\setcounter{chapter}{1}
%----------------------------------------------------------------------------
\section{Environment of the data sets}
%----------------------------------------------------------------------------
The interconnectivity and interoperability \cite{TalebArchInsSys} of sensors and actuators are transforming our environment, and has an enormous effect on all industries as well. Furthermore, we use more and more sensors, and connect them to each other and to computers to collect useful data and make predictions.

In this thesis I have developed several prediction methods for proactive future maintenance, to diminish human and financial cost.

The information that I had received to work with is a CAN-Bus data collected from the STILL Company's forklifts.

My main task was to prototype methods and reason for specific data types in order to get better results to develop the Industry 4.0 \cite{Indfourpointzero,} MANTIS project further. \cite{PaliHCsCPS}\cite{HCsCPS2}\cite{GartnerIoT}\cite{HuangCPS}\cite{Pali}\cite{HCsCPS}\cite{ProMain}

To achieve this object, I have worked with two kinds of data sets. These will be described in the following sections.
%----------------------------------------------------------------------------
\section{Datasets and Objectives}
%----------------------------------------------------------------------------
    \subsection{CAN fingerprint data set}
%----------------------------------------------------------------------------
\noindent
My task was to create a tidy format from the given "CAN fingerprint" data set with cleaning methods. Explore what can be predicted with no target value provided, to figure out requirements for better prognostication. Furthermore visualize the connection between the given variables. Moreover, the second phase data wave and small amount and low resolution target value I made a pilot prediction algorithm to iterate the ultimate solution searching process. 
\noindent
After a few iteration the calculation of the remaining useful life of the tires are achievable. 
%----------------------------------------------------------------------------
  \subsection{Electrical fail prediction data set}
%----------------------------------------------------------------------------
\paragraph\noindent
The objective is with this data set is come up with a calculation method to compute the remaining useful life, but because the data is represented in a different way, it's obvious for the need of a distinct approach.
\paragraph\noindent
In the influence of the L. Barabasi-Albert book \cite{BALNWSCBOOK}, the RUL computation algorithm is based on a graph representation of the available data. 
\paragraph\noindent
The representation is based on the idea of the realization of the possible states as nodes and the paths of the machines between the nodes in time as edges.
\paragraph\noindent
With the previously described data representation, the possibility to go into a specific fail state\footnote{one type of the nodes, the other is the warning states.} is a given timespan is calculable and that is the RUL itself.

\section{Chapter descriptions}
%----------------------------------------------------------------------------
		\paragraph{Technical background}
%----------------------------------------------------------------------------

Overview of the technical environment of the handling, processing and remaining useful life extracting from the given data sets.

%----------------------------------------------------------------------------
		\paragraph{Design}
%----------------------------------------------------------------------------

Brief description of the planning phase, explaining the reasons behind the planned solutions and assembling an abstract work-flow.

%----------------------------------------------------------------------------
		\paragraph{Implementation}
%----------------------------------------------------------------------------

Effectuation of the designed data handling work-flow, from the raw data sets to the final calculated goal value. Including the wielding of the unexpected obstacles and the solutions for them.

%----------------------------------------------------------------------------
		\paragraph{Verification}
%----------------------------------------------------------------------------

Evaluating the process of calculating the remaining useful life of the two given data set. Comparison of the different hardware and software solutions.

%----------------------------------------------------------------------------
		\paragraph{Conclusion}
%----------------------------------------------------------------------------

Appreciate the success of the whole project, contextualize the outcome in the real word. Estimating the difficulties, core principles, planning methods and fails.

Adumbration of the possible future improvement for the whole work-flow, from the data collection to the RUL calculation and interpretation and usage.

