%----------------------------------------------------------------------------
\chapter*{Introduction}\addcontentsline{toc}{chapter}{Introduction}
%----------------------------------------------------------------------------
\section{Environtment of the data sets}
%----------------------------------------------------------------------------
Nowadays, in the concerned field of science, the interconnectivity and interoperability are transforming our environment, and has an enormous effect on all industries as well.Furthermore, we use more and more sensors, and connect them to each other and to computers to collect useful data to predict the future.

In this thesis I have developed several prediction methods for proactive future maintenance, to diminish human and financial cost.

The information what I had received to work with is a CAN-Bus data collected from the STILL Company's forklifts.

My main task was to prototype methods and reason for specific data types for better results to develop further the Industry 4.0 MANTIS project.

To achive this object, I have worked with two kinds of data sets. Which will be described in the following sections.
%----------------------------------------------------------------------------
\section{My Solution}
%----------------------------------------------------------------------------
    \subsection{CAN fingerprint data set}
%----------------------------------------------------------------------------
My task was to create a tidy format from the given "CAN fingerprint" data set with cleaning methods, to explore what can de predicted with no goal value provided, to figure out requirements for better prognostication, to visualize the connection between the given variables. Moreover, the second phase data wave and small amount and low resolution goal value I made a pilot prediction algorithm to iterate the ultimate solution searching process. 
%----------------------------------------------------------------------------
  \subsection{Electrical fail prediction data set}
%----------------------------------------------------------------------------
2 txt and 1 xml file the desired outcome is to represent it in a graph.

\section{Chapter descriptions}
%----------------------------------------------------------------------------
		\paragraph{Technical background}
%----------------------------------------------------------------------------

Overview of the technical environment of the handling, processing and remaining useful life extracting from the given data sets.

%----------------------------------------------------------------------------
		\paragraph{Design}
%----------------------------------------------------------------------------

Brief description of the planning phase, explainin the reasons behind the planned solutions and assembling an abstact workflow.

%----------------------------------------------------------------------------
		\paragraph{Implementation}
%----------------------------------------------------------------------------

Effectuation of the designed data hangling workflow, from the raw data sets to the final calculated goal value. Including the wielding of the unexpected obstacles and the solutions for them.

%----------------------------------------------------------------------------
		\paragraph{Verification}
%----------------------------------------------------------------------------

Evaluating the process of calculating the remaining useful life of the two given data set. Comparsion of the different hardware and software solutions.

%----------------------------------------------------------------------------
		\paragraph{Conclusion}
%----------------------------------------------------------------------------

Appreciate the success of the whole project, contextise the outcome in the real word. Estimating the difficulties, core principles, planning methods and fails.

Adumbration of the possible future improvement for the whole workflow, from the data collection to the rul calculation and interpretation and useage.

