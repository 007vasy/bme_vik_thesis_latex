%----------------------------------------------------------------------------
\chapter{Introduction}
%----------------------------------------------------------------------------
\section{Environment of data sets}
%----------------------------------------------------------------------------
The interconnectivity and interoperability \cite{TalebArchInsSys} of sensors and actuators are transforming our environment, and have an enormous effects on the industry as well. Furthermore, we use more and more sensors, and connect them to each other and to computers in order to collect useful data and make predictions.

In this thesis I have developed several prediction methods for proactive maintenance to diminish human and financial resources.

The information I had received to work with is a CAN(Controller Area Network)-Bus data collected from the STILL company's forklifts.

My main task was to prototype methods and to reason for specific data types in order to receive better results developing the Industry 4.0 \cite{Indfourpointzero} MANTIS project further \cite{PaliHCsCPS,HCsCPS2,GartnerIoT,HuangCPS,PaliCPS,HCsCPS,ProMain}.

To achieve this object, I have worked with two kinds of data sets. These will be described in the following sections.
%----------------------------------------------------------------------------
\section{Datasets and objectives}
%----------------------------------------------------------------------------
    \subsection{CAN fingerprint data set}
%----------------------------------------------------------------------------
\noindent
My first task was to create a tidy format from the given "CAN fingerprint" data set with cleaning methods. Exploring what can be predicted without provided target value \hyphenation{improves} the identification of requirements for better prognostication. Furthermore, visualize the connection between the given variables. Moreover, with the second phase data wave and small amount low resolution target value, I made a pilot prediction algorithm to iterate the ultimate solution searching process. 
\noindent
After a few iterations, the calculation of the remaining useful life of the given elements \footnore{tires in this case}, are available. 
%----------------------------------------------------------------------------
  \subsection{Electrical fail prediction data set}
%----------------------------------------------------------------------------
\paragraph\noindent
The objective is with this data set is  to come up with a calculation method to compute the remaining useful life, but due to the data being represented in a different way, there is an obvious need for a distinct approach.
\paragraph\noindent
Under the influence of the L. Barabasi-Albert book Network Science \cite{BALNWSCBOOK}, the RUL (Remaining Useful Life) computation algorithm is based on graph visualization of the available data. 
\paragraph\noindent
The representation is based on the idea of the possible states' realization as nodes and the machines' state transitions in time as edges.
\paragraph\noindent
With the previously described data representation, the possibility to go into a specific failure state\footnote{one type of the nodes, the other is the warning states.} with a given timespan is calculable and that is the RUL itself.

\section{Organization of the thesis}
In order to get a better initial overview of this thesis, let me briefly summarize the purpose of the upcoming chapters.
%----------------------------------------------------------------------------
		\paragraph{Technical background:}
%----------------------------------------------------------------------------

Overview of technical environment of handling, processing and information on remaining useful life extracting from given data sets.

%----------------------------------------------------------------------------
		\paragraph{Design:}
%----------------------------------------------------------------------------

Brief description of planning phase, explaining reasons behind the planned solutions, and assembling an abstract work-flow.

%----------------------------------------------------------------------------
		\paragraph{Implementation:}
%----------------------------------------------------------------------------

Effectuation of designed data handling work-flow from raw data sets to the final calculated goal value. Description of some unexpected obstacles and the solutions for them.

%----------------------------------------------------------------------------
		\paragraph{Verification:}
%----------------------------------------------------------------------------

Process evaluation regarding the calculation of RUL on the given data sets. Comparison and estimation of the different hardware and software solutions.

%----------------------------------------------------------------------------
		\paragraph{Conclusion:}
%----------------------------------------------------------------------------
Appreciation the success of the whole project, conceptualization the outcome in real word. Estimation the difficulties, core principles, planning methods, and fails.

Suggestions on possible future improvements for the whole work-flow from data collection to RUL calculation, interpretation, and usage.

