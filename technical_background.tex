%----------------------------------------------------------------------------
\chapter{Technical backgroud}
%----------------------------------------------------------------------------
	\section{The MANTIS project}
%----------------------------------------------------------------------------
		\subsection{Overall picture about the MANTIS project}
%----------------------------------------------------------------------------
The MANTIS pilot project is a data-supported decision making system. The data comes from industrial sensors and on a popular IOT communication protokoll (MQTT) communicate with the given company cloud service, where the data tidying and prediction happens, provide future insights to the company's employees for premaintenence and with that cut human, financial and material resources.

%----------------------------------------------------------------------------
		\subsection{Datasets from STILL}
%----------------------------------------------------------------------------
			\subsubsection{CAN fingerprint dataset}
%----------------------------------------------------------------------------
				\noindent
As I already mentioned, the STILL forklifts use the CAN technology, for two kind on data collecting, in the CAN fingerpint dataset, the workers from the STILL had  connected a special CAN bus data collector to a few forklift and had done with them specific task and they have called it "fingerprints". The *.file names helped the understanding of the type of the road what the forklift go through:

\begin{itemize}[noitemsep]
    \item {800hTestDrive\_fast.mat}
    \item {800hTestDrive\_slow.mat}
    \item {FastTrack800h\_fast.mat}
 	\item {FastTrack800h\_slow.mat}
	\item {FastTrackEight\_wl\_fast.mat}
	\item {FastTrackEight\_wl\_slow.mat}
	\item {FastTrackEight\_wol\_fast.mat}
	\item {FastTrackEight\_wol\_slow.mat}
  	\item {FastTrackOval\_wl\_fast.mat}
	\item {FastTrackOval\_wl\_slow.mat}
	\item {FastTrackOval\_wol\_fast.mat}
	\item {FastTrackOval\_wol\_slow.mat}
	\item {Ramp\_wl\_fast.mat} 
	\item {Ramp\_wl\_slow.mat}
	\item {Ramp\_wol\_fast.mat}
	\item {Ramp\_wol\_slow.mat}
    \item {WorkCycle\_fast.mat}
    \item {WorkCycle\_slow.mat}
	\item {Shunt800h\_fast.mat}  	
  	\item {Shunt800h\_slow.mat}	
 	\item {Shunt\_fast.mat} 	
 	\item {Shunt\_slow.mat}
\end{itemize}

				\noindent
As it seems in the file names the workers made the truck more than one time with various speed and load.

But in this first iteration, STILL won't privided any goal value for tire abrosion. So from this first iteration i could only made explanatory data analysis to fuel further data collection and enough goal values in sustainable resolation and quality to train some predictive model on that data.
				\noindent
In the second iteration on data providing i get on 10.2017. more data in a quite similar fashion, but from working forklift, and a top of that with some additional tire measurement. The CAN-bus sensor data span about two month in time, but there was just a few measurement of the tire abrosion, every third week. This is nearly enough to make some prediction, but for further iterations it was neccessary.



%----------------------------------------------------------------------------
			\subsubsection{Fault prediction dataset}
%----------------------------------------------------------------------------
Brief solution description:Fault prediction
%----------------------------------------------------------------------------
			\subsubsection{Remaing Usefull life dataset}
%----------------------------------------------------------------------------
Brief solution description:RUL
%----------------------------------------------------------------------------
	\section{Data processing methods}
%----------------------------------------------------------------------------
%CAN fingerprint


				\noindent
If one wants to make educated guess about the future, one has to bring the data in tidy
%tidy reference
form, it advised and required for preparation to the most of the data mining methods.
				\noindent
In this case the *.mat files involved some so-called box-shorting to achieve the "tidy" state. In box-shorting one has to know the range of the shorting key (in this data's case the timestamp key's range) and with that, one can make a large data table with a lot of "NA"-s, but it's needed to orginize the multi frequency sensory data.

Once the tidying is done, comes the \textit{exploratory} data analysis to make sense about the data mass and come with ways to achieve the best prediction possible, and short out unnessesary columns, to make the future process faster.
%FAULTPRED?
%----------------------------------------------------------------------------
	\section{Developing tools}
%----------------------------------------------------------------------------
		\subsection{Hardware}
%----------------------------------------------------------------------------
			\subsubsection{Personal computer}
%----------------------------------------------------------------------------
My PC is a Dell Latitude E6320 64 bit with an Intel(R) Core(TM) i5-2520M CPU @ 2.50GHz processor.
%TODO reference
%----------------------------------------------------------------------------
			\subsubsection{Department Linux R server}
%----------------------------------------------------------------------------
The Department's super compoter called "batman" is Linux based R server and it's capadle of running R script's on enormous dataset's. It has two Intel(R) Xeon(R) CPU E5-2620 v2 @ 2.10GHz processors each of them has 6 cores and capable of running 12 Threads.
%TODO reference
%----------------------------------------------------------------------------
		\subsection{Software}
%----------------------------------------------------------------------------
			\subsubsection{Linux Ubuntu 17.04}
%----------------------------------------------------------------------------
The operation system in the title is an open source software, stable and convinent for an advadced user, and to go further it's easiest way to communicate with to Linux system via SSH.
%TODO ref
%----------------------------------------------------------------------------				
			\subsubsection{RStudio}
%----------------------------------------------------------------------------
RStudio is the most common tool for developing R projects, it has good package controlling application.
%TODO ref
%----------------------------------------------------------------------------	
			\subsubsection{Termius Android mobile phone application}
%----------------------------------------------------------------------------
The tidying and data preprocessing method is resource and time consuming, and for optimize it to the daily life i used this app with my 4G or sometimes just HSDPA internet acces to maintain an SSH connection to the Linux R server and supervise the and check the process progress time to time and hand the occasinal error ands bugs.
%TODO ref
%----------------------------------------------------------------------------
			\subsubsection{Git and GitHub}
%----------------------------------------------------------------------------
For version control reasons and open source commitment, i choose this common toolset to save my work as it progressed.
%TODO ref
%----------------------------------------------------------------------------
%			\subsubsection{PyCharm}
%----------------------------------------------------------------------------	
%----------------------------------------------------------------------------
		\subsection{Management}
%----------------------------------------------------------------------------
For productivity reasons i use the GTD methodology, it's a time trying 10 years old method. It provides clarity, focus and flexible planning for me. 
%TODO ref