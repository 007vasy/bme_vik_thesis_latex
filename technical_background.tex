%----------------------------------------------------------------------------
\chapter{Technical backgroud}
%----------------------------------------------------------------------------
	\section{The MANTIS project}
%----------------------------------------------------------------------------
		\subsection{Overall picture about the MANTIS project}
%----------------------------------------------------------------------------
Mantis project brief description

%----------------------------------------------------------------------------
		\subsection{Datasets from STILL}
%----------------------------------------------------------------------------
			\subsubsection{CAN fingerprint dataset}
%----------------------------------------------------------------------------
				\noindent
As I already mentioned, the STILL forklifts use the CAN technology, for two kind on data collecting, in the CAN fingerpint dataset, the workers from the STILL had  connected a special CAN bus data collector to a few forklift and had done with them specific task and they have called it "fingerprints". The *.file names helped the understanding of the type of the road what the forklift go through:

\begin{itemize}[noitemsep]
    \item {800hTestDrive\_fast.mat}
    \item {800hTestDrive\_slow.mat}
    \item {FastTrack800h\_fast.mat}
 	\item {FastTrack800h\_slow.mat}
	\item {FastTrackEight\_wl\_fast.mat}
	\item {FastTrackEight\_wl\_slow.mat}
	\item {FastTrackEight\_wol\_fast.mat}
	\item {FastTrackEight\_wol\_slow.mat}
  	\item {FastTrackOval\_wl\_fast.mat}
	\item {FastTrackOval\_wl\_slow.mat}
	\item {FastTrackOval\_wol\_fast.mat}
	\item {FastTrackOval\_wol\_slow.mat}
	\item {Ramp\_wl\_fast.mat} 
	\item {Ramp\_wl\_slow.mat}
	\item {Ramp\_wol\_fast.mat}
	\item {Ramp\_wol\_slow.mat}
    \item {WorkCycle\_fast.mat}
    \item {WorkCycle\_slow.mat}
	\item {Shunt800h\_fast.mat}  	
  	\item {Shunt800h\_slow.mat}	
 	\item {Shunt\_fast.mat} 	
 	\item {Shunt\_slow.mat}
\end{itemize}

				\noindent
As it seems in the file names the workers made the truck more than one time with various speed and load.

But in this first iteration, STILL won't privided any goal value for tire abrosion. So from this first iteration i could only made explanatory data analysis to fuel further data collection and enough goal values in sustainable resolation and quality to train some predictive model on that data.
				\noindent
In the second iteration on data providing i get on 10.2017. more data in a quite similar fashion, but from working forklift, and a top of that with some additional tire measurement. The CAN-bus sensor data span about two month in time, but there was just a few measurement of the tire abrosion, every third week. This is nearly enough to make some prediction, but for further iterations it was neccesery.

%----------------------------------------------------------------------------
			\subsubsection{Fault prediction dataset}
%----------------------------------------------------------------------------
Brief solution description:Fault prediction
%----------------------------------------------------------------------------
			\subsubsection{Remaing Usefull life dataset}
%----------------------------------------------------------------------------
Brief solution description:RUL
%----------------------------------------------------------------------------
	\section{Data processing methods}
%----------------------------------------------------------------------------
A TeXnicCenter és a LEd kizárólag szerkesztõprogram 



%----------------------------------------------------------------------------
	\section{Developing tools}
%----------------------------------------------------------------------------
Linux operáció

TODO SUBSECTION
TODO RSTUDIO
TODO PYCharm